\subsection{Convergencia puntual y uniforme}

\begin{defi}[Convergencia puntual]
Sean $(E,d)$ y $(E',d')$ espacios métricos y sea
$(f_n)_{n\in\N}$ una sucesión de funciones $f_n : E \to E'$.
Decimos que $(f_n)$ \emph{converge puntualmente} a una función
$f : E \to E'$ si
\[
\forall x \in E:\quad \lim_{n\to\infty} d'\bigl(f_n(x),f(x)\bigr) = 0.
\]
Equivalentemente,
\[
\forall x \in E,\ \forall \varepsilon>0\ \exists N\in\N\ \forall n\ge N:
\ d'\bigl(f_n(x),f(x)\bigr) < \varepsilon.
\]
En este caso escribimos $f_n(x) \to f(x)$ puntualmente, o simplemente
$f_n \to f$ puntualmente en $E$.
\end{defi}

\begin{defi}[Convergencia uniforme]
Con la notación anterior, diremos que $(f_n)$ \emph{converge uniformemente}
a $f$ en $E$ si
\[
\forall \varepsilon>0\ \exists N\in\N\ \forall n\ge N\ \forall x\in E:
\ d'\bigl(f_n(x),f(x)\bigr) < \varepsilon.
\]
En este caso escribimos $f_n \rightrightarrows f$ en $E$.
\end{defi}

\subsection{Límite uniforme de funciones continuas}

\begin{teo}
Sea $(E,d)$, $(E',d')$ espacios métricos y sea
$(f_n)_{n\in\N}$ una sucesión de funciones continuas
$f_n : E \to E'$, que converge uniformemente a
$f : E \to E'$. Entonces $f$ es continua.
\end{teo}

\begin{proof}
Sea $x_0 \in E$ y sea $\varepsilon>0$ dado.  
Como $f_n \rightrightarrows f$, existe $N \in \N$ tal que
\[
d'\bigl(f_n(x),f(x)\bigr) < \frac{\varepsilon}{3}
\quad \forall x\in E,\ \forall n\ge N.
\]
En particular,
\[
d'\bigl(f_N(x_0),f(x_0)\bigr) < \frac{\varepsilon}{3}.
\]

Como $f_N$ es continua en $x_0$, existe $\delta>0$ tal que
\[
d(x,x_0) < \delta \Rightarrow
d'\bigl(f_N(x),f_N(x_0)\bigr) < \frac{\varepsilon}{3}.
\]

Tomemos ahora un $x\in E$ con $d(x,x_0)<\delta$. Entonces
\[
\begin{aligned}
d'\bigl(f(x),f(x_0)\bigr)
&\le d'\bigl(f(x),f_N(x)\bigr)
   + d'\bigl(f_N(x),f_N(x_0)\bigr)
   + d'\bigl(f_N(x_0),f(x_0)\bigr) \\
&< \frac{\varepsilon}{3} + \frac{\varepsilon}{3} + \frac{\varepsilon}{3}
= \varepsilon.
\end{aligned}
\]
Como $\varepsilon>0$ era arbitrario, esto prueba que $f$ es continua en
$x_0$. Dado que $x_0$ era un punto cualquiera de $E$, $f$ es continua
en todo $E$.
\end{proof}

\subsection{Pasaje al límite bajo el signo integral}

\begin{prop}
Sea $[a,b] \subset \R$ con $a<b$ y sean $f_n,f : [a,b]\to\R$ funciones
continuas tales que $f_n \rightrightarrows f$ en $[a,b]$. Entonces
\[
\lim_{n\to\infty} \int_a^b f_n(t)\,dt
= \int_a^b f(t)\,dt.
\]
\end{prop}

\begin{proof}
Sea $\varepsilon>0$ arbitrario. Como $f_n \rightrightarrows f$ en $[a,b]$,
por definición de convergencia uniforme existe $N\in\N$ tal que
\[
\forall n \ge N\ \forall x\in[a,b]:
\ |f_n(x)-f(x)| < \frac{\varepsilon}{b-a}.
\]

Fijemos $n \ge N$. Entonces, para todo $t\in[a,b]$,
\[
|f_n(t)-f(t)| < \frac{\varepsilon}{b-a}.
\]
Integrando en el intervalo $[a,b]$ y usando la desigualdad
triangular para integrales, obtenemos
\[
\begin{aligned}
\left|\int_a^b f_n(t)\,dt - \int_a^b f(t)\,dt\right|
&= \left|\int_a^b (f_n(t)-f(t))\,dt\right| \\
&\le \int_a^b |f_n(t)-f(t)|\,dt \\
&\le \int_a^b \frac{\varepsilon}{b-a}\,dt \\
&= \frac{\varepsilon}{b-a}\,(b-a) = \varepsilon.
\end{aligned}
\]

Hemos probado que
\[
\forall \varepsilon>0\ \exists N\in\N\ \forall n\ge N:
\left|\int_a^b f_n(t)\,dt - \int_a^b f(t)\,dt\right| < \varepsilon,
\]
lo cual es precisamente
\[
\lim_{n\to\infty} \int_a^b f_n(t)\,dt
= \int_a^b f(t)\,dt.
\]
\end{proof}

\subsection{Convergencia de derivadas}

\begin{prop}
Sean $f_n : [a,b]\to\R$ funciones de clase $C^1$ en $[a,b]$, tales que
\begin{itemize}
    \item $f_n \to f$ puntualmente en $[a,b]$;
    \item $f_n' \rightrightarrows g$ en $[a,b]$.
\end{itemize}
Entonces $f$ es derivable en $[a,b]$ y
\[
f' = g.
\]
\end{prop}

\begin{proof}
Sea $x_0 \in [a,b]$ fijo. Por el Teorema Fundamental del Cálculo aplicado
a cada $f_n$, para todo $x\in[a,b]$ se cumple
\[
f_n(x) - f_n(x_0) = \int_{x_0}^x f_n'(t)\,dt.
\]

Tomando límite cuando $n\to\infty$ en ambos lados:

- Por la convergencia puntual $f_n(x)\to f(x)$ y $f_n(x_0)\to f(x_0)$,
  el lado izquierdo converge a $f(x)-f(x_0)$.

- Por la proposición anterior (pasaje al límite bajo el integral) y la
  convergencia uniforme de $f_n'$ a $g$, el lado derecho converge a
  \(
  \displaystyle \int_{x_0}^x g(t)\,dt.
  \)

Por lo tanto,
\[
f(x) - f(x_0) = \int_{x_0}^x g(t)\,dt
\quad\text{para todo } x\in[a,b].
\]

Definamos
\[
F(x) := f(x_0) + \int_{x_0}^x g(t)\,dt.
\]
La función $F$ es de clase $C^1$ en $[a,b]$ y satisface $F' = g$.
Además, la igualdad anterior muestra que $f(x)=F(x)$ para todo $x$.
Luego $f$ es derivable y $f' = g$.
\end{proof}

\subsection{Sucesiones uniformemente de Cauchy}

\begin{defi}[Sucesión uniformemente de Cauchy]
Sea $(E,d)$ un espacio métrico, $A\subseteq E$ y
$(E',d')$ otro espacio métrico. Una sucesión
$(f_n)_{n\in\N}$ de funciones $f_n : A \to E'$ se dice
\emph{uniformemente de Cauchy} si
\[
\forall \varepsilon>0\ \exists n_0\in\N\ \forall m,n\ge n_0\
\forall x\in A:\ d'\bigl(f_n(x),f_m(x)\bigr) < \varepsilon.
\]
\end{defi}

\begin{teo}
Sea $A\subseteq E$ y $(f_n)_{n\in\N}$ una sucesión de funciones
$f_n : A \to \R$ uniformemente de Cauchy. Entonces existe una función
$f : A \to \R$ tal que $f_n \rightrightarrows f$ en $A$.
\end{teo}

\begin{proof}
Fijemos $x\in A$. Consideremos la sucesión numérica
\[
\bigl(f_n(x)\bigr)_{n\in\N} \subset \R.
\]
De la definición de sucesión uniformemente de Cauchy se deduce en
particular que, para todo $\varepsilon>0$, existe $n_0$ tal que
\[
|f_n(x) - f_m(x)| < \varepsilon
\quad\text{para todo } m,n\ge n_0.
\]
Es decir, para cada $x\in A$, la sucesión $(f_n(x))$ es de Cauchy en $\R$.
Como $\R$ es completo, existe el límite
\[
f(x) := \lim_{n\to\infty} f_n(x) \in \R.
\]
Así definimos una función $f : A \to \R$.

Resta ver que $f_n \rightrightarrows f$ en $A$. Sea $\varepsilon>0$.
Por ser $(f_n)$ uniformemente de Cauchy, existe $n_0 \in \N$ tal que
\[
|f_n(x) - f_m(x)| < \varepsilon
\quad\forall m,n\ge n_0,\ \forall x\in A.
\]

Fijemos $n\ge n_0$ y $x\in A$ arbitrarios. Tomando el límite cuando
$m\to\infty$ en la desigualdad anterior, obtenemos
\[
|f_n(x) - f(x)| \le \varepsilon,
\]
ya que $f_m(x)\to f(x)$ para cada $x$.

Como la cota $\varepsilon$ es independiente de $x$ y vale para todo
$n\ge n_0$, concluimos que
\[
\sup_{x\in A} |f_n(x) - f(x)| \le \varepsilon
\quad\text{para todo } n\ge n_0.
\]
Esto es precisamente $f_n \rightrightarrows f$ en $A$.
\end{proof}