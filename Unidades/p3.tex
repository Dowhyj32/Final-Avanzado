\subsection{Conjuntos abiertos}

\begin{defi}
Sea $(E,d)$ un espacio métrico. Un subconjunto $U \subseteq E$ se dice
\emph{abierto} si
\[
\forall x \in U\ \exists r > 0 \ \text{tal que} \ B(x,r) \subseteq U,
\]
donde
\[
B(x,r) = \{y \in E : d(x,y) < r\}
\]
es la bola abierta de centro $x$ y radio $r$.
\end{defi}

\begin{prop}
Sea $(E,d)$ un espacio métrico, $x_0 \in E$ y $r>0$.
Entonces la bola abierta $B(x_0,r)$ es un conjunto abierto.
\end{prop}

\begin{proof}
Debemos ver que para todo punto $x \in B(x_0,r)$ existe un radio
$\varepsilon > 0$ tal que
\[
B(x,\varepsilon) \subseteq B(x_0,r).
\]

Sea $x \in B(x_0,r)$. Por definición de bola abierta,
\[
d(x,x_0) < r.
\]
Definimos
\[
\varepsilon = r - d(x,x_0).
\]
Entonces $\varepsilon > 0$ porque $d(x,x_0) < r$.

Tomemos ahora un punto cualquiera $y \in B(x,\varepsilon)$; es decir,
\[
d(x,y) < \varepsilon.
\]
Aplicando la desigualdad triangular, obtenemos
\[
d(y,x_0) \le d(y,x) + d(x,x_0).
\]
Sustituyendo las cotas anteriores,
\[
d(y,x_0) < \varepsilon + d(x,x_0)
= \bigl(r - d(x,x_0)\bigr) + d(x,x_0) = r.
\]
Por lo tanto, $d(y,x_0) < r$, lo que significa que $y \in B(x_0,r)$.

Como $y$ fue elegido arbitrariamente en $B(x,\varepsilon)$, hemos probado
\[
B(x,\varepsilon) \subseteq B(x_0,r).
\]
Y esto vale para todo $x \in B(x_0,r)$, con el $\varepsilon$ definido
como arriba. Por definición de conjunto abierto, $B(x_0,r)$ es abierto.
\end{proof}

\begin{defi}
Sea $(E,d)$ un espacio métrico y sea $A \subseteq E$.
El \emph{interior} de $A$ es el conjunto
\[
A^\circ = \{\,x \in A : \exists r>0 \text{ tal que } B(x,r) \subseteq A\,\}.
\]
Equivalente: $A^\circ$ es el conjunto de los puntos interiores de $A$.
\end{defi}

\begin{prop}
Sea $(E,d)$ un espacio métrico y $A,A_1,A_2 \subseteq E$.
Se tienen las siguientes propiedades:
\begin{enumerate}[label=(\roman*)]
    \item $A^\circ \subseteq A$.
    \item Si $A_1 \subseteq A_2$, entonces $A_1^\circ \subseteq A_2^\circ$.
    \item $A^\circ$ es un conjunto abierto.
    \item Si $G$ es abierto y $G \subseteq A$, entonces $G \subseteq A^\circ$.
\end{enumerate}
\end{prop}

\begin{proof}
(i) Por definición,
\[
A^\circ = \{x \in A : \exists r>0 \text{ tal que } B(x,r) \subseteq A\}.
\]
En particular, todo $x \in A^\circ$ pertenece a $A$, luego
$A^\circ \subseteq A$.

\medskip

(ii) Supongamos que $A_1 \subseteq A_2$ y sea $x \in A_1^\circ$.
Entonces, por definición de interior, existe $r>0$ tal que
\[
B(x,r) \subseteq A_1.
\]
Como $A_1 \subseteq A_2$, se tiene también
\[
B(x,r) \subseteq A_2,
\]
de modo que $x$ es punto interior de $A_2$, es decir, $x \in A_2^\circ$.
Hemos probado que $A_1^\circ \subseteq A_2^\circ$.

\medskip

(iii) Queremos ver que $A^\circ$ es abierto. Sea $x \in A^\circ$.
Entonces existe $r>0$ tal que
\[
B(x,r) \subseteq A.
\]
Por la proposición ya demostrada, $B(x,r)$ es un conjunto abierto.
En particular, como $x \in B(x,r)$ y $B(x,r)$ es abierto, existe
$\varepsilon>0$ tal que
\[
B(x,\varepsilon) \subseteq B(x,r).
\]
Como además $B(x,r) \subseteq A$, obtenemos
\[
B(x,\varepsilon) \subseteq A.
\]
Por definición de interior, esto implica que $x \in A^\circ$ y, de hecho,
para todo $y \in B(x,\varepsilon)$ se cumple $y \in A^\circ$.
Por lo tanto, $B(x,\varepsilon) \subseteq A^\circ$.

Hemos visto que para todo $x \in A^\circ$ existe $\varepsilon>0$ tal que
$B(x,\varepsilon) \subseteq A^\circ$, lo cual significa que $A^\circ$
es abierto.

\medskip

(iv) Sea $G$ un conjunto abierto tal que $G \subseteq A$, y sea
$x \in G$. Como $G$ es abierto, existe $r>0$ tal que
\[
B(x,r) \subseteq G.
\]
De la inclusión $G \subseteq A$ se deduce
\[
B(x,r) \subseteq A.
\]
Luego $x$ es punto interior de $A$, es decir, $x \in A^\circ$.
Como $x$ era un punto arbitrario de $G$, concluimos que $G \subseteq A^\circ$.
\end{proof}

\begin{prop}
Sea $(E,d)$ un espacio métrico. Entonces se verifican:
\begin{enumerate}[label=(\roman*)]
    \item La unión arbitraria de conjuntos abiertos es un conjunto abierto.
    \item La intersección finita de conjuntos abiertos es un conjunto abierto.
\end{enumerate}
\end{prop}

\begin{proof}
(i) Sea $(G_i)_{i\in I}$ una familia cualquiera de conjuntos abiertos en $E$
(indexada por un conjunto $I$ no necesariamente numerable) y definamos
\[
G = \bigcup_{i\in I} G_i.
\]
Queremos ver que $G$ es abierto.

Sea $x \in G$. Entonces existe algún índice $i_0 \in I$ tal que
$x \in G_{i_0}$. Como $G_{i_0}$ es abierto, existe $r>0$ tal que
\[
B(x,r) \subseteq G_{i_0}.
\]
De aquí se deduce
\[
B(x,r) \subseteq G_{i_0} \subseteq G.
\]
Por lo tanto, para todo $x \in G$ hemos encontrado un radio $r>0$ con
$B(x,r) \subseteq G$. Por definición, $G$ es abierto.

\medskip

(ii) Sea $G_1,\dots,G_n$ una familia finita de conjuntos abiertos en $E$
y definamos
\[
H = \bigcap_{k=1}^n G_k.
\]
Probemos que $H$ es abierto.

Tomemos $x \in H$. Entonces $x \in G_k$ para todo $k=1,\dots,n$.
Como cada $G_k$ es abierto, existe $r_k > 0$ tal que
\[
B(x,r_k) \subseteq G_k
\quad \text{para cada } k=1,\dots,n.
\]
Definimos
\[
r = \min\{r_1,\dots,r_n\}.
\]
Entonces $r > 0$ y, para todo $k$,
\[
B(x,r) \subseteq B(x,r_k) \subseteq G_k.
\]
De aquí se sigue
\[
B(x,r) \subseteq \bigcap_{k=1}^n G_k = H.
\]
Por lo tanto, para cada $x \in H$ existe $r>0$ tal que
$B(x,r) \subseteq H$, lo que muestra que $H$ es abierto.
\end{proof}

\begin{obs}
El resultado sobre intersección de abiertos es, en general, válido sólo
para intersecciones finitas. Una intersección infinita de abiertos puede
no ser abierta.

Por ejemplo, en $(\R,d_{\text{eucl}})$, para cada $n \in \N$ el conjunto
\[
G_n = \left(-\frac{1}{n}, \frac{1}{n}\right)
\]
es un abierto. Consideremos la intersección
\[
\bigcap_{n\in\N} G_n
= \bigcap_{n\in\N} \left(-\frac{1}{n}, \frac{1}{n}\right).
\]
Es fácil ver que
\[
\bigcap_{n\in\N} \left(-\frac{1}{n}, \frac{1}{n}\right) = \{0\}.
\]
El conjunto $\{0\}$ no es abierto en $\R$ con la métrica usual, ya que si
tomamos cualquier $r>0$, la bola $B(0,r) = (-r,r)$ contiene puntos distintos
de $0$, de modo que nunca se cumple $B(0,r) \subseteq \{0\}$.
Por lo tanto, la intersección numerable de los abiertos $G_n$ no es abierta.
\end{obs}

\subsection{Conjuntos cerrados}

\begin{defi}
Sea $(E,d)$ un espacio métrico y $A \subseteq E$.
Un punto $x \in E$ se llama \emph{punto de adherencia} (o
\emph{punto de clausura}) de $A$ si
\[
\forall r > 0 \quad B(x,r) \cap A \ne \varnothing.
\]
\end{defi}

\begin{defi}
Sea $(E,d)$ un espacio métrico y $A \subseteq E$.
La \emph{clausura} (o \emph{adherencia}) de $A$ es el conjunto
\[
\overline{A}
= \{x \in E : x \text{ es punto de adherencia de } A\}.
\]
\end{defi}

\begin{defi}
Sea $(E,d)$ un espacio métrico. Un subconjunto $F \subseteq E$ se dice
\emph{cerrado} si su complemento $E \setminus F$ es un conjunto abierto.
\end{defi}

\begin{prop}
Sea $(E,d)$ un espacio métrico, $x_0 \in E$ y $r>0$.
Entonces la bola cerrada
\[
\overline{B}(x_0,r) = \{x \in E : d(x,x_0) \le r\}
\]
es un conjunto cerrado.
\end{prop}

\begin{proof}
Por definición, $\overline{B}(x_0,r)$ es cerrada si su complemento
$E \setminus \overline{B}(x_0,r)$ es abierto. Sea
\[
x \in E \setminus \overline{B}(x_0,r).
\]
Entonces $d(x,x_0) > r$. Definimos
\[
\varepsilon = \frac{d(x,x_0) - r}{2}.
\]
Como $d(x,x_0) > r$, se tiene $\varepsilon > 0$.

Mostremos que
\[
B(x,\varepsilon) \subseteq E \setminus \overline{B}(x_0,r).
\]
Sea $y \in B(x,\varepsilon)$, es decir,
\[
d(x,y) < \varepsilon.
\]
Por la desigualdad triangular,
\[
d(y,x_0) \ge d(x,x_0) - d(x,y).
\]
Luego
\[
d(y,x_0) > d(x,x_0) - \varepsilon
= d(x,x_0) - \frac{d(x,x_0) - r}{2}
= \frac{d(x,x_0) + r}{2}.
\]
Como $d(x,x_0) > r$, se tiene
\[
\frac{d(x,x_0) + r}{2} > \frac{r + r}{2} = r,
\]
de modo que $d(y,x_0) > r$. En particular, $d(y,x_0) \not\le r$, así que
$y \notin \overline{B}(x_0,r)$.

Hemos probado que todo $y \in B(x,\varepsilon)$ pertenece al complemento
de la bola cerrada:
\[
B(x,\varepsilon) \subseteq E \setminus \overline{B}(x_0,r).
\]
Por lo tanto, para cada $x$ del complemento existe un $\varepsilon>0$
tal que la bola $B(x,\varepsilon)$ está contenida en dicho complemento.
Esto significa que $E \setminus \overline{B}(x_0,r)$ es abierto.

Concluimos que $\overline{B}(x_0,r)$ es cerrado.
\end{proof}


\begin{prop}[Caracterización de la clausura mediante cerrados]
Sea $(E,d)$ un espacio métrico y $A \subseteq E$.
Entonces:
\begin{enumerate}[label=(\roman*)]
    \item $A \subseteq \overline{A}$.
    \item $\overline{A}$ es un conjunto cerrado.
    \item Si $F$ es un conjunto cerrado con $A \subseteq F$, entonces
    $\overline{A} \subseteq F$.
\end{enumerate}
En particular, $\overline{A}$ es el menor conjunto cerrado que contiene a $A$,
y se puede escribir
\[
\overline{A} = \bigcap\{F \subseteq E : F \text{ es cerrado y } A \subseteq F\}.
\]
\end{prop}

\begin{proof}
(i) Sea $x \in A$. Entonces, para cualquier $r>0$, se tiene
$x \in B(x,r)$, y por lo tanto $B(x,r) \cap A \ne \varnothing$.
Es decir, $x es$ punto de adherencia de $A$, y por definición
$x \in \overline{A}$. Luego $A \subseteq \overline{A}$.

\medskip

(ii) Probemos que $\overline{A}$ es cerrado mostrando que
$E \setminus \overline{A}$ es abierto.

Sea $x \in E \setminus \overline{A}$. Entonces $x$ no es punto
de adherencia de $A$, es decir, existe $r>0$ tal que
\[
B(x,r) \cap A = \varnothing.
\]
Mostraremos que $B(x,r/2) \subseteq E \setminus \overline{A}$.

Sea $y \in B(x,r/2)$, de modo que $d(x,y) < r/2$.
Tomemos $s = r/2$. Consideremos un punto $z \in B(y,s)$;
entonces $d(y,z) < s = r/2$. Por la desigualdad triangular,
\[
d(x,z) \le d(x,y) + d(y,z) < \frac{r}{2} + \frac{r}{2} = r,
\]
luego $z \in B(x,r)$. Como $B(x,r) \cap A = \varnothing$,
se sigue que $z \notin A$. En consecuencia,
\[
B(y,s) \cap A = \varnothing.
\]
Esto significa que $y$ no es punto de adherencia de $A$,
es decir, $y \notin \overline{A}$.

Como $y$ fue un punto arbitrario de $B(x,r/2)$, hemos probado
que $B(x,r/2) \subseteq E \setminus \overline{A}$. Por lo tanto,
$E \setminus \overline{A}$ es abierto, y en consecuencia
$\overline{A}$ es cerrado.

\medskip

(iii) Sea $F$ un conjunto cerrado tal que $A \subseteq F$.
Debemos probar que $\overline{A} \subseteq F$.

Sea $x \in E \setminus F$. Como $F$ es cerrado, su complemento
$E \setminus F$ es abierto. Entonces existe $r>0$ tal que
\[
B(x,r) \subseteq E \setminus F.
\]
Como $A \subseteq F$, se tiene $A \cap (E \setminus F) = \varnothing$,
y en particular
\[
B(x,r) \cap A = \varnothing.
\]
Esto muestra que $x$ no es punto de adherencia de $A$, es decir,
$x \notin \overline{A}$. Hemos probado
\[
E \setminus F \subseteq E \setminus \overline{A}.
\]
Tomando complementos,
\[
\overline{A} \subseteq F.
\]
Esto vale para todo conjunto cerrado $F$ que contiene a $A$, con lo cual
queda demostrado que $\overline{A}$ es el menor cerrado que contiene a $A$,
y en particular
\[
\overline{A} = \bigcap\{F \subseteq E : F \text{ es cerrado y } A \subseteq F\}.
\]
\end{proof}

\begin{prop}
Un conjunto $F \subseteq E$ es cerrado si y sólo si
\[
F = \overline{F}.
\]
\end{prop}

\begin{proof}
($\Rightarrow$) Si $F$ es cerrado y $F$ contiene a $F$, por la proposición
anterior, $\overline{F} \subseteq F$. Por otra parte, siempre se cumple
$F \subseteq \overline{F}$. De aquí se deduce $F = \overline{F}$.

($\Leftarrow$) Si $F = \overline{F}$, como sabemos que $\overline{F}$ es
cerrado, se sigue que $F$ es cerrado.
\end{proof}

\begin{prop}[Caracterización secuencial de la clausura]
Sea $(E,d)$ un espacio métrico, $A \subseteq E$ y $x \in E$.
Entonces las siguientes afirmaciones son equivalentes:
\begin{enumerate}[label=(\roman*)]
    \item $x \in \overline{A}$.
    \item Existe una sucesión $(a_n)_{n\in\N}$ con $a_n \in A$ para todo $n$ tal que
    \[
    \lim_{n\to\infty} a_n = x.
    \]
\end{enumerate}
\end{prop}

\begin{proof}
$(i) \Rightarrow (ii)$) Supongamos que $x \in \overline{A}$. Por definición
de clausura, esto significa que para todo $r>0$ se cumple
\[
B(x,r) \cap A \ne \varnothing.
\]

Para cada $n \in \N$, consideremos el radio $r_n = \frac{1}{n} > 0$.
Como $B(x,r_n) \cap A \ne \varnothing$, podemos elegir un punto
\[
a_n \in B\Bigl(x,\frac{1}{n}\Bigr) \cap A.
\]
Así definimos una sucesión $(a_n)_{n\in\N}$ de puntos de $A$ que satisface
\[
d(a_n,x) < \frac{1}{n} \quad \text{para todo } n \in \N.
\]

Veamos que $a_n \to x$. Sea $\varepsilon > 0$. Elegimos $N \in \N$ tal que
\[
\frac{1}{N} < \varepsilon.
\]
Si $n \ge N$, entonces
\[
d(a_n,x) < \frac{1}{n} \le \frac{1}{N} < \varepsilon.
\]
Por lo tanto,
\[
\forall \varepsilon>0\ \exists N \in \N\ \forall n \ge N:\ d(a_n,x) < \varepsilon,
\]
lo cual significa que $a_n \to x$. Esto prueba (ii).

\medskip

$(ii) \Rightarrow (i)$) Recíprocamente, supongamos que existe una sucesión
$(a_n)_{n\in\N}$ con $a_n \in A$ para todo $n$ tal que $a_n \to x$.
Debemos probar que $x \in \overline{A}$, es decir, que
\[
\forall r>0 \quad B(x,r) \cap A \ne \varnothing.
\]

Sea $r>0$. Como $a_n \to x$, por definición de límite existe $N \in \N$
tal que, para todo $n \ge N$,
\[
d(a_n,x) < r.
\]
En particular, tomando $n = N$, tenemos $a_N \in B(x,r)$.
Como además $a_N \in A$ por construcción de la sucesión, se obtiene
\[
a_N \in B(x,r) \cap A,
\]
y por lo tanto $B(x,r) \cap A \ne \varnothing$.

Como esto vale para todo $r>0$, concluimos que $x$ es punto de adherencia
de $A$, es decir, $x \in \overline{A}$.
\end{proof}

\begin{prop}
Sea $(E,d)$ un espacio métrico y $F \subseteq E$.
Entonces $F$ es cerrado si y sólo si se verifica:
\[
\text{si } (x_n)_{n\in\N} \subseteq F \text{ y } x_n \to x \text{ en } E,
\text{ entonces } x \in F.
\]
\end{prop}

\begin{proof}
($\Rightarrow$) Supongamos que $F$ es cerrado y sea $(x_n)$ una sucesión
de puntos de $F$ tal que $x_n \to x$ en $E$. Supongamos, por absurdo,
que $x \notin F$. Entonces $x \in E \setminus F$.

Como $E \setminus F$ es abierto (porque $F$ es cerrado), existe $r>0$
tal que
\[
B(x,r) \subseteq E \setminus F.
\]
Por la convergencia de $(x_n)$ a $x$, sabemos que
\[
\forall \varepsilon > 0 \ \exists N \in \N \ \forall n \ge N:
\ d(x_n,x) < \varepsilon.
\]
Tomamos $\varepsilon = r$. Entonces existe $N \in \N$ tal que
si $n \ge N$, se cumple $d(x_n,x) < r$, es decir, $x_n \in B(x,r)$.
Pero $B(x,r) \subseteq E \setminus F$, luego $x_n \notin F$ para
todo $n \ge N$. Esto contradice el hecho de que $x_n \in F$ para
todos los $n$, por hipótesis. Por lo tanto, debe ser $x \in F$.

\medskip

($\Leftarrow$) Supongamos ahora que se cumple la propiedad secuencial:
toda sucesión de puntos de $F$ que converge en $E$ tiene su límite en $F$.
Queremos ver que $F$ es cerrado, es decir, que $E \setminus F$ es abierto.

Tomemos un punto $x \in E \setminus F$. Supongamos, por absurdo, que
$E \setminus F$ no es abierto. Entonces no existe ningún $r>0$ tal que
$B(x,r) \subseteq E \setminus F$. En particular, para todo $n \in \N$,
el radio $r = \tfrac{1}{n}$ no sirve, es decir:
\[
B\Bigl(x,\frac{1}{n}\Bigr) \not\subseteq E \setminus F.
\]
Esto significa que para cada $n \in \N$ existe algún punto
\[
x_n \in B\Bigl(x,\frac{1}{n}\Bigr) \cap F.
\]
Por construcción, $x_n \in F$ y, además,
\[
d(x_n,x) < \frac{1}{n} \quad \text{para todo } n \in \N.
\]
De aquí se deduce que $x_n \to x$ en el espacio métrico $(E,d)$.

Pero, por hipótesis secuencial, el límite de cualquier sucesión de puntos
de $F$ que converge en $E$ debe pertenecer a $F$, de modo que $x \in F$.
Esto contradice que $x \in E \setminus F$.

Por lo tanto, nuestra suposición era falsa: para cada $x \in E \setminus F$
debe existir $r>0$ tal que $B(x,r) \subseteq E \setminus F$, lo que muestra
que $E \setminus F$ es abierto. En consecuencia, $F$ es cerrado.
\end{proof}

\begin{prop}
Sea $(E,d)$ un espacio métrico. Entonces se verifican:
\begin{enumerate}[label=(\roman*)]
    \item $\varnothing$ y $E$ son conjuntos cerrados.
    \item La intersección arbitraria de conjuntos cerrados es un conjunto cerrado.
    \item La unión finita de conjuntos cerrados es un conjunto cerrado.
\end{enumerate}
\end{prop}

\begin{proof}
(i) Notemos que
\[
E \setminus \varnothing = E
\quad\text{y}\quad
E \setminus E = \varnothing.
\]
Como ya sabemos que $E$ y $\varnothing$ son abiertos, por definición
de conjunto cerrado se sigue que $\varnothing$ y $E$ son cerrados.

\medskip

(ii) Sea $(F_i)_{i\in I}$ una familia cualquiera de conjuntos cerrados
en $E$, indexada por un conjunto $I$ arbitrario, y consideremos
\[
F = \bigcap_{i\in I} F_i.
\]
Queremos probar que $F$ es cerrado.

Usamos la relación entre complementos, uniones e intersecciones:
\[
E \setminus F
= E \setminus \bigg(\bigcap_{i\in I} F_i\bigg)
= \bigcup_{i\in I} (E \setminus F_i).
\]
Cada $F_i$ es cerrado, así que $E \setminus F_i$ es abierto para todo $i$.
Por lo tanto, la unión $\bigcup_{i\in I} (E \setminus F_i)$ es abierta
(por la propiedad ya demostrada para uniones arbitrarias de abiertos).
En consecuencia, $E \setminus F$ es abierto, lo que implica que $F$
es cerrado.

\medskip

(iii) Sean $F_1,\dots,F_n$ conjuntos cerrados en $E$ y consideremos
\[
H = \bigcup_{k=1}^n F_k.
\]
Entonces
\[
E \setminus H
= E \setminus \bigg(\bigcup_{k=1}^n F_k\bigg)
= \bigcap_{k=1}^n (E \setminus F_k).
\]
Como cada $F_k$ es cerrado, $E \setminus F_k$ es abierto para todo $k$.
La intersección finita de abiertos es abierta, por lo tanto
$\bigcap_{k=1}^n (E \setminus F_k)$ es abierta. De aquí se deduce
que $E \setminus H$ es abierto y, por definición, $H$ es cerrado.
\end{proof}

\subsection{Puntos de acumulación y frontera}

\begin{defi}
Sea $(E,d)$ un espacio métrico y $A \subseteq E$.
Un punto $x \in E$ se llama \emph{punto de acumulación} (o
\emph{punto límite}) de $A$ si
\[
\forall r>0 \quad \bigl(B(x,r) \setminus \{x\}\bigr) \cap A \neq \varnothing.
\]
Denotamos por $A'$ al conjunto de todos los puntos de acumulación de $A$.
\end{defi}

\begin{prop}[Caracterización secuencial de puntos de acumulación]
Sea $(E,d)$ un espacio métrico, $A \subseteq E$ y $x \in E$.
Entonces las siguientes afirmaciones son equivalentes:
\begin{enumerate}[label=(\roman*)]
    \item $x$ es punto de acumulación de $A$, es decir $x \in A'$.
    \item Existe una sucesión $(a_n)_{n\in\N}$ con $a_n \in A \setminus \{x\}$
    para todo $n$ tal que
    \[
    \lim_{n\to\infty} a_n = x.
    \]
\end{enumerate}
\end{prop}

\begin{proof}
$(i) \Rightarrow (ii)$) Supongamos que $x \in A'$. Entonces, por definición,
para todo $r>0$ se cumple
\[
\bigl(B(x,r) \setminus \{x\}\bigr) \cap A \neq \varnothing.
\]
Para cada $n \in \N$, tomamos $r_n = \frac{1}{n}$ y elegimos
\[
a_n \in \bigl(B(x,r_n) \setminus \{x\}\bigr) \cap A.
\]
Entonces $a_n \in A \setminus \{x\}$ y $d(a_n,x) < \frac{1}{n}$ para todo
$n \in \N$.

Sea $\varepsilon>0$. Elegimos $N \in \N$ tal que
\(
\frac{1}{N} < \varepsilon.
\)
Si $n \ge N$, entonces
\[
d(a_n,x) < \frac{1}{n} \le \frac{1}{N} < \varepsilon.
\]
Por definición de límite, $a_n \to x$.

\medskip

$(ii) \Rightarrow (i)$) Recíprocamente, supongamos que existe una sucesión
$(a_n)$ con $a_n \in A \setminus \{x\}$ para todo $n$ y $a_n \to x$.
Sea $r>0$. Como $a_n \to x$, existe $N \in \N$ tal que, para todo $n \ge N$,
\[
d(a_n,x) < r.
\]
En particular, $a_N \in B(x,r)$, y como $a_N \neq x$ se tiene
$a_N \in B(x,r) \setminus \{x\}$. Además, $a_N \in A$.

Por lo tanto,
\[
\bigl(B(x,r) \setminus \{x\}\bigr) \cap A \neq \varnothing
\]
para todo $r>0$, lo que significa que $x$ es punto de acumulación de $A$,
es decir, $x \in A'$.
\end{proof}

\begin{teo}
Sea $A \subseteq E$ un subconjunto de un espacio métrico $(E,d)$.
Entonces
\[
\overline{A} = A \cup A'.
\]
\end{teo}

\begin{proof}
Primero probamos la inclusión $\overline{A} \subseteq A \cup A'$.
Sea $x \in \overline{A}$. Si $x \in A$, ya está.
Supongamos que $x \notin A$. Como $x \in \overline{A}$, por definición de
clausura se cumple
\[
\forall r>0 \quad B(x,r) \cap A \neq \varnothing.
\]
Pero como $x \notin A$, esto implica automáticamente que
\[
\forall r>0 \quad \bigl(B(x,r) \setminus \{x\}\bigr) \cap A \neq \varnothing,
\]
es decir, $x$ es punto de acumulación de $A$, $x \in A'$. En cualquiera de
los dos casos, $x \in A \cup A'$. Por lo tanto
\[
\overline{A} \subseteq A \cup A'.
\]

Ahora probamos la inclusión inversa $A \cup A' \subseteq \overline{A}$.
Si $x \in A$, ya vimos que siempre se cumple $A \subseteq \overline{A}$,
así que $x \in \overline{A}$.

Si $x \in A'$, entonces para todo $r>0$ se tiene
\[
\bigl(B(x,r) \setminus \{x\}\bigr) \cap A \neq \varnothing.
\]
En particular, esto implica $B(x,r) \cap A \neq \varnothing$ para todo $r>0$,
de modo que $x$ es punto de adherencia de $A$, es decir, $x \in \overline{A}$.

En ambos casos $x \in \overline{A}$, por lo que
\[
A \cup A' \subseteq \overline{A}.
\]

Concluimos que $\overline{A} = A \cup A'$.
\end{proof}

\begin{cor}
Sea $A \subseteq E$. Entonces $A$ es cerrado si y sólo si
\[
A' \subseteq A.
\]
\end{cor}

\begin{proof}
($\Rightarrow$) Si $A$ es cerrado, por definición $A = \overline{A}$.
Por el teorema anterior,
\[
\overline{A} = A \cup A'.
\]
Luego
\[
A = A \cup A'.
\]
Esto sólo puede ocurrir si $A' \subseteq A$.

\medskip

($\Leftarrow$) Recíprocamente, supongamos que $A' \subseteq A$.
Del teorema anterior tenemos
\[
\overline{A} = A \cup A'.
\]
Como $A' \subseteq A$, se obtiene
\[
\overline{A} = A.
\]
Pero un conjunto es cerrado si y sólo si coincide con su clausura, por lo que
$A$ es cerrado.
\end{proof}

\begin{defi}
Sea $(E,d)$ un espacio métrico y $A \subseteq E$.
Un punto $x \in E$ se llama \emph{punto frontera} (o \emph{punto de borde})
de $A$ si para todo $r>0$ se cumple
\[
B(x,r) \cap A \neq \varnothing
\quad\text{y}\quad
B(x,r) \cap (E \setminus A) \neq \varnothing.
\]
El conjunto de todos los puntos frontera de $A$ se denota por $\partial A$
y se llama \emph{frontera} (o \emph{borde}) de $A$.
\end{defi}

\begin{prop}
Para todo $A \subseteq E$ se verifica
\[
\partial A = \overline{A} \setminus A^\circ.
\]
\end{prop}

\begin{proof}
Sea $x \in \partial A$. Entonces, por definición de punto frontera,
para todo $r>0$,
\[
B(x,r) \cap A \neq \varnothing
\quad\text{y}\quad
B(x,r) \cap (E \setminus A) \neq \varnothing.
\]
En particular, la primera condición implica que $x$ es punto de adherencia
de $A$, es decir $x \in \overline{A}$. Por otra parte, la segunda condición
impide que exista algún $r>0$ con $B(x,r) \subseteq A$, luego $x$ no es
punto interior de $A$, es decir $x \notin A^\circ$. Por lo tanto
$x \in \overline{A} \setminus A^\circ$.

Recíprocamente, sea $x \in \overline{A} \setminus A^\circ$. Entonces
$x \in \overline{A}$, de modo que para todo $r>0$,
\[
B(x,r) \cap A \neq \varnothing.
\]
Además, $x \notin A^\circ$, es decir, no existe ningún $r>0$ tal que
$B(x,r) \subseteq A$. Por lo tanto, para todo $r>0$ se tiene
\[
B(x,r) \not\subseteq A,
\]
lo que significa que para todo $r>0$ existe $y \in B(x,r)$ tal que
$y \notin A$, es decir,
\[
B(x,r) \cap (E \setminus A) \neq \varnothing.
\]
Juntando ambas condiciones, $x$ es punto frontera de $A$, es decir
$x \in \partial A$.

Hemos probado ambas inclusiones, así que $\partial A = \overline{A} \setminus A^\circ$.
\end{proof}

\begin{prop}
Sea $A \subseteq E$. Entonces:
\begin{enumerate}[label=(\roman*)]
    \item $\partial A$ es un conjunto cerrado.
    \item Se tiene la descomposición
    \[
    E = A^\circ \,\dot\cup\, \partial A \,\dot\cup\, (E \setminus \overline{A}),
    \]
    donde las uniones son disjuntas.
    \item Se verifica
    \[
    \overline{A} = A \cup \partial A.
    \]
\end{enumerate}
\end{prop}

\begin{proof}
(i) Por la proposición anterior,
\[
\partial A = \overline{A} \setminus A^\circ
= \overline{A} \cap (E \setminus A^\circ).
\]
Sabemos que $\overline{A}$ es cerrado. Como $A^\circ$ es abierto,
su complemento $E \setminus A^\circ$ es cerrado. La intersección de
conjuntos cerrados es cerrada, luego $\partial A$ es cerrado.

\medskip

(ii) Por definición de interior, clausura y frontera, se tiene
\[
A^\circ \subseteq A \subseteq \overline{A}.
\]
Además, por la proposición anterior,
\[
\partial A = \overline{A} \setminus A^\circ.
\]
Entonces
\[
\overline{A} = A^\circ \,\dot\cup\, \partial A.
\]
Por otra parte, el complemento de $\overline{A}$ es
$E \setminus \overline{A}$, y es abierto. Así, todo punto de $E$
pertenece o bien a $\overline{A}$ o bien a $E \setminus \overline{A}$,
y dentro de $\overline{A}$ está exactamente en $A^\circ$ o en
$\partial A$. Esto da la descomposición
\[
E = A^\circ \,\dot\cup\, \partial A \,\dot\cup\, (E \setminus \overline{A}).
\]

\medskip

(iii) De la igualdad del teorema anterior
\(
\overline{A} = A \cup A'
\)
y del hecho de que $\partial A \subseteq \overline{A}$, se deduce
en particular que
\[
A \subseteq \overline{A}
\quad\text{y}\quad
\partial A \subseteq \overline{A}.
\]
Por lo tanto,
\[
A \cup \partial A \subseteq \overline{A}.
\]

Recíprocamente, sea $x \in \overline{A}$. Si $x \in A$, ya está.
Si $x \notin A$, como $x \in \overline{A}$, para todo $r>0$ se cumple
$B(x,r) \cap A \neq \varnothing$. Además, como $x \notin A$, se tiene
$x \in E \setminus A$, luego para todo $r>0$ se cumple
$x \in B(x,r) \cap (E \setminus A)$, o sea
\[
B(x,r) \cap (E \setminus A) \neq \varnothing.
\]
Por definición, esto significa que $x \in \partial A$.
En cualquier caso, $x \in A \cup \partial A$, con lo cual
\[
\overline{A} \subseteq A \cup \partial A.
\]
Concluimos que $\overline{A} = A \cup \partial A$.
\end{proof}

\subsection{Métricas equivalentes}

Sea $E$ un conjunto y $d_1,d_2$ dos métricas en $E$.

\begin{defi}
Decimos que las métricas $d_1$ y $d_2$ en $E$ son
\emph{equivalentes} si inducen los mismos conjuntos abiertos, es decir,
si para todo $U \subseteq E$ se cumple:
\[
U \text{ es abierto en } (E,d_1)
\quad \Longleftrightarrow \quad
U \text{ es abierto en } (E,d_2).
\]
\end{defi}

\begin{prop}[Caracterización secuencial]
Sean $d_1,d_2$ dos métricas en $E$. Son equivalentes si y sólo si
para toda sucesión $(x_n)_{n\in\N}$ en $E$ y todo $x \in E$ se verifica
\[
x_n \to x \text{ en } (E,d_1)
\quad \Longleftrightarrow \quad
x_n \to x \text{ en } (E,d_2).
\]
\end{prop}

\begin{proof}
($\Rightarrow$) Supongamos que $d_1$ y $d_2$ son equivalentes, es decir,
tienen los mismos conjuntos abiertos.

Recordemos que, en un espacio métrico, una sucesión $(x_n)$ converge a $x$
si y sólo si se verifica la siguiente propiedad topológica:
\[
\forall U \text{ abierto con } x \in U,\ \exists N \in \N\ \forall n \ge N:
\ x_n \in U.
\]

Sea $(x_n)$ una sucesión en $E$ y $x \in E$.
Supongamos que $x_n \to x$ en $(E,d_1)$.
Entonces, para todo abierto $U$ de $(E,d_1)$ con $x \in U$, existe
$N$ tal que, si $n \ge N$, se cumple $x_n \in U$.

Como $d_1$ y $d_2$ tienen los mismos abiertos, ese mismo conjunto $U$
es abierto también en $(E,d_2)$. La condición anterior es exactamente
la definición de convergencia de $(x_n)$ a $x$ en $(E,d_2)$. Por lo tanto,
$x_n \to x$ en $(E,d_2)$.

La implicación recíproca se demuestra igual, intercambiando el rol de
$d_1$ y $d_2$.

\medskip

($\Leftarrow$) Supongamos ahora que las sucesiones convergentes son las
mismas para $d_1$ y $d_2$.

Queremos ver que los conjuntos abiertos también coinciden.
Sea $U \subseteq E$ abierto en $(E,d_1)$, y probemos que es abierto en
$(E,d_2)$.

Para eso, basta ver que si $(x_n)$ es una sucesión en $E \setminus U$
que converge a algún $x \in E$ respecto de $d_2$, entonces
$x \in E \setminus U$ (es decir, $x \notin U$). En efecto, esta propiedad
caracteriza a los cerrados, y al tomar complementos caracteriza a los
abiertos.

Sea entonces $(x_n)$ una sucesión con $x_n \in E \setminus U$ para todo $n$
y $x_n \to x$ en $(E,d_2)$. Por hipótesis, las sucesiones convergentes son
las mismas en ambas métricas, así que también $x_n \to x$ en $(E,d_1)$.

Como $U$ es abierto en $(E,d_1)$, su complemento $E \setminus U$ es cerrado
en $(E,d_1)$. Por la caracterización secuencial de cerrados,
si una sucesión de puntos de $E \setminus U$ converge (en $d_1$),
su límite debe pertenecer a $E \setminus U$. En particular,
$x \in E \setminus U$, es decir, $x \notin U$.

Hemos probado que el complemento de $U$ es cerrado también con la métrica
$d_2$, luego $U$ es abierto en $(E,d_2)$. El mismo argumento, cambiando
el rol de $d_1$ y $d_2$, muestra la implicación inversa. Por lo tanto,
los abiertos de $d_1$ coinciden con los de $d_2$, y las métricas son
equivalentes.
\end{proof}

\begin{defi}
Decimos que dos métricas $d_1,d_2$ en $E$ son \emph{fuertemente
equivalentes} si existen constantes $c_1,c_2 > 0$ tales que
\[
c_1\, d_1(x,y) \le d_2(x,y) \le c_2\, d_1(x,y)
\quad \text{para todo } x,y \in E.
\]
\end{defi}

\begin{prop}
Si $d_1$ y $d_2$ son fuertemente equivalentes, entonces son
equivalentes (en el sentido de que inducen los mismos conjuntos abiertos).
\end{prop}

\begin{proof}
Sea $U$ un abierto de $(E,d_1)$ y tomemos $x \in U$.
Por definición de abierto, existe $r>0$ tal que
\[
B_{d_1}(x,r) \subseteq U,
\]
donde $B_{d_1}(x,r) = \{y : d_1(x,y) < r\}$.

Consideremos ahora la bola en la métrica $d_2$
\[
B_{d_2}\bigl(x, c_1 r\bigr)
= \{y \in E : d_2(x,y) < c_1 r\}.
\]
Si $y \in B_{d_2}(x,c_1 r)$, entonces $d_2(x,y) < c_1 r$, y usando
la desigualdad $c_1 d_1(x,y) \le d_2(x,y)$ obtenemos
\[
c_1 d_1(x,y) \le d_2(x,y) < c_1 r
\quad\Longrightarrow\quad
d_1(x,y) < r.
\]
Luego $y \in B_{d_1}(x,r)$, y por lo tanto
\[
B_{d_2}(x,c_1 r) \subseteq B_{d_1}(x,r) \subseteq U.
\]
Hemos mostrado que para todo $x \in U$ existe un radio $c_1 r>0$ tal que
la bola $d_2$-abierta correspondiente está contenida en $U$, de modo que
$U$ es abierto en $(E,d_2)$.

La implicación recíproca (todo abierto $d_2$-abierto es $d_1$-abierto)
se prueba exactamente igual usando la otra desigualdad
$d_2(x,y) \le c_2 d_1(x,y)$. Concluimos que $d_1$ y $d_2$ tienen los mismos
abiertos, es decir, son métricas equivalentes.
\end{proof}

\begin{obs}
En espacios de dimensión finita (por ejemplo, $\R^n$), todas las normas
son fuertemente equivalentes. En particular, las métricas inducidas
por las normas $d_1,d_2,d_\infty$ en $\R^n$ son equivalentes, y por lo tanto
tienen los mismos abiertos, los mismos cerrados, las mismas sucesiones
convergentes, etc.
\end{obs}

\begin{teo}
Sean $d$ y $d'$ dos métricas sobre un mismo conjunto $E$.
Las métricas $d$ y $d'$ son (topológicamente) equivalentes si y sólo si
para todo $x \in E$ y todo $r>0$ existen $r_1,r_2>0$ tales que
\[
B_{d'}(x,r_1) \subseteq B_d(x,r)
\quad\text{y}\quad
B_d(x,r_2) \subseteq B_{d'}(x,r),
\]
donde
\[
B_d(x,r) = \{y \in E : d(x,y) < r\},
\qquad
B_{d'}(x,r) = \{y \in E : d'(x,y) < r\}.
\]
\end{teo}

\begin{proof}
($\Rightarrow$) Supongamos que $d$ y $d'$ son métricas equivalentes, es decir,
inducen los mismos conjuntos abiertos.

Sea $x \in E$ y $r>0$ arbitrarios. Entonces $B_d(x,r)$ es un abierto en
$(E,d)$; como los abiertos de $(E,d)$ y $(E,d')$ coinciden, $B_d(x,r)$
es también abierto en $(E,d')$.

Por definición de abierto en la métrica $d'$, existe $r_1>0$ tal que
\[
B_{d'}(x,r_1) \subseteq B_d(x,r).
\]

Para obtener la otra inclusión, usamos el mismo argumento intercambiando
el rol de $d$ y $d'$: como $B_{d'}(x,r)$ es abierto en $(E,d')$ y las
familias de abiertos coinciden, $B_{d'}(x,r)$ es abierto en $(E,d)$.
Luego existe $r_2>0$ tal que
\[
B_d(x,r_2) \subseteq B_{d'}(x,r).
\]
Así se verifica la condición del enunciado.

\medskip

($\Leftarrow$) Recíprocamente, supongamos que para todo $x \in E$ y
todo $r>0$ existen $r_1,r_2>0$ tales que
\[
B_{d'}(x,r_1) \subseteq B_d(x,r)
\quad\text{y}\quad
B_d(x,r_2) \subseteq B_{d'}(x,r).
\]

Queremos probar que los abiertos de $(E,d)$ y $(E,d')$ coinciden.

Sea $U \subseteq E$ un conjunto abierto en $(E,d)$. Sea $x \in U$.
Como $U$ es abierto para $d$, existe $r>0$ tal que
\[
B_d(x,r) \subseteq U.
\]
Por hipótesis, existe $r_2>0$ tal que
\[
B_d(x,r_2) \subseteq B_{d'}(x,r).
\]
Aplicando de nuevo la hipótesis, pero ahora a la bola $d'$-abierta
$B_{d'}(x,r)$, podemos elegir $r_3>0$ con
\[
B_{d'}(x,r_3) \subseteq B_d(x,r_2).
\]
Por lo tanto,
\[
B_{d'}(x,r_3) \subseteq B_d(x,r_2) \subseteq B_d(x,r) \subseteq U.
\]
Hemos encontrado, para cada $x \in U$, un radio $r_3>0$ tal que
$B_{d'}(x,r_3) \subseteq U$, lo que significa que $U$ es abierto en
$(E,d')$.

El argumento recíproco (si $U$ es abierto en $(E,d')$ entonces lo es
en $(E,d)$) se obtiene exactamente igual, usando de nuevo la hipótesis
para pasar de bolas $d'$-abiertas a bolas $d$-abiertas. Concluimos que
las familias de conjuntos abiertos de $(E,d)$ y $(E,d')$ coinciden, es
decir, $d$ y $d'$ son métricas equivalentes.
\end{proof}

\subsection{Sucesiones de Cauchy y espacios métricos completos}

\begin{defi}
Sea $(E,d)$ un espacio métrico y $A \subseteq E$.
Decimos que $A$ es \emph{acotado} si existen $x \in E$ y $r>0$ tales que
\[
A \subset B(x,r),
\]
donde $B(x,r) = \{y \in E : d(x,y) < r\}$ es la bola abierta de centro $x$
y radio $r$.
\end{defi}

\begin{defi}
Sea $(E,d)$ un espacio métrico y $(x_n)_{n\in\N} \subseteq E$ una sucesión.
Decimos que $(x_n)$ es \emph{acotada} si existen $x \in E$ y $r>0$ tales que
\[
x_n \in B(x,r) \quad \text{para todo } n \in \N.
\]
Equivalente: el conjunto $\{x_n : n \in \N\}$ es acotado en $E$.
\end{defi}

\begin{defi}
Sea $(E,d)$ un espacio métrico y $(x_n)_{n\in\N} \subseteq E$ una sucesión.
Decimos que $(x_n)$ es una \emph{sucesión de Cauchy} si
\[
\forall \varepsilon>0\ \exists n_0 \in \N\ \text{tal que}\ \forall n,m \ge n_0:
\ d(x_n,x_m) < \varepsilon.
\]
\end{defi}

\begin{teo}
Sea $(E,d)$ un espacio métrico y $(x_n)_{n\in\N} \subseteq E$. Entonces:
\begin{enumerate}[label=(\arabic*)]
    \item Si $(x_n)$ es de Cauchy, entonces es acotada.
    \item Si $(x_n)$ es convergente en $E$, entonces es de Cauchy.
    \item Si $(x_n)$ es de Cauchy y tiene alguna subsucesión convergente,
    entonces $(x_n)$ es convergente (en $E$) y converge al mismo límite
    que esa subsucesión.
\end{enumerate}
\end{teo}

\begin{proof}
(1) Supongamos que $(x_n)$ es de Cauchy. Tomamos $\varepsilon = 1$ en la
definición. Entonces existe $n_0 \in \N$ tal que, si $n,m \ge n_0$, se cumple
\[
d(x_n,x_m) < 1.
\]
En particular, para todo $n \ge n_0$,
\[
d(x_n,x_{n_0}) < 1.
\]

Sea ahora
\[
M = \max\bigl\{d(x_1,x_{n_0}), d(x_2,x_{n_0}), \dots, d(x_{n_0-1},x_{n_0}), 1\bigr\}
\]
(si $n_0=1$, simplemente tomamos $M=1$). Entonces $M>0$ y, para todo
$n \in \N$,
\[
d(x_n,x_{n_0}) \le M.
\]
Por lo tanto,
\[
x_n \in B(x_{n_0},M) \quad \text{para todo } n \in \N,
\]
lo que muestra que la sucesión $(x_n)$ es acotada.

\medskip

(2) Supongamos que $(x_n)$ converge en $E$ a algún $x \in E$; es decir,
\[
\forall \varepsilon>0\ \exists n_0 \in \N\ \forall n \ge n_0:
\ d(x_n,x) < \varepsilon/2.
\]
(Escribimos $\varepsilon/2$ para simplificar las cuentas que siguen.)

Sea ahora $\varepsilon>0$ fijo. Tomamos $n_0$ como arriba.
Si $n,m \ge n_0$, por la desigualdad triangular:
\[
d(x_n,x_m) \le d(x_n,x) + d(x,x_m) < \frac{\varepsilon}{2} + \frac{\varepsilon}{2}
= \varepsilon.
\]
Hemos mostrado que para todo $\varepsilon>0$ existe $n_0$ tal que,
si $n,m \ge n_0$, se cumple $d(x_n,x_m) < \varepsilon$; es decir,
$(x_n)$ es de Cauchy.

\medskip

(3) Supongamos que $(x_n)$ es de Cauchy y que existe una subsucesión
$(x_{n_k})_{k\in\N}$ tal que $x_{n_k} \to x \in E$.

Debemos probar que $x_n \to x$.

Sea $\varepsilon>0$. Como $(x_{n_k})$ converge a $x$, existe $K \in \N$ tal que,
si $k \ge K$,
\[
d(x_{n_k},x) < \varepsilon/2.
\]
Como $(x_n)$ es de Cauchy, existe $n_0 \in \N$ tal que, si $n,m \ge n_0$,
\[
d(x_n,x_m) < \varepsilon/2.
\]

Definimos
\[
N = \max\{n_0, n_K\}.
\]
Sea ahora $n \ge N$. Entonces $n \ge n_0$ y $n_K \ge n_0$, por lo que
aplicando la propiedad de Cauchy con $m = n_K$ obtenemos:
\[
d(x_n,x_{n_K}) < \varepsilon/2.
\]
Por otra parte, como $n_K \ge K$, tenemos
\[
d(x_{n_K},x) < \varepsilon/2.
\]

Usando la desigualdad triangular,
\[
d(x_n,x) \le d(x_n,x_{n_K}) + d(x_{n_K},x)
< \frac{\varepsilon}{2} + \frac{\varepsilon}{2} = \varepsilon.
\]

Hemos probado que
\[
\forall \varepsilon>0\ \exists N \in \N\ \forall n \ge N:\ d(x_n,x) < \varepsilon,
\]
lo cual significa que $x_n \to x$ en $(E,d)$.
\end{proof}

\begin{defi}
Un espacio métrico $(E,d)$ se dice \emph{completo} si toda sucesión de
Cauchy en $E$ es convergente a un punto de $E$.
\end{defi}

% (Opcional, pero muy usado en ejercicios)
\begin{prop}
Sea $(E,d)$ un espacio métrico completo y sea $F \subseteq E$ un subconjunto
cerrado. Entonces el subespacio métrico $(F,d)$ es completo.
\end{prop}

\begin{proof}
Sea $(x_n)_{n\in\N}$ una sucesión de Cauchy en $F$ (con la métrica $d$
restringida). Como $F \subseteq E$, también es una sucesión de Cauchy en $E$.
Dado que $(E,d)$ es completo, existe $x \in E$ tal que $x_n \to x$ en $E$.

Como $F$ es cerrado en $E$ y todos los $x_n$ están en $F$, por la
caracterización secuencial de cerrados se tiene necesariamente $x \in F$.
Por lo tanto, $(x_n)$ converge a un punto de $F$, y así $(F,d)$ es completo.
\end{proof}
