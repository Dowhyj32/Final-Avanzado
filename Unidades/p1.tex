\subsection{Axioma de completitud}
Dado $A \subset\mathbb{R}$ no vacío y acotado superiormente, existe $\sup A$.\\
Dado $A \subset\mathbb{R}$ no vacío y acotado inferiormente, existe $\inf A$.

\subsection{Infimo}
Sea $A \subset\mathbb{R}$, no vacío y acotado inferiormente.
\[
i = \inf A \Longleftrightarrow
\begin{cases}
i \le a & \text{para todo } a \in A,\\[4pt]
\forall \varepsilon > 0 \ \exists a \in A \text{ tal que } i \le a < i + \varepsilon.
\end{cases}
\]
\begin{proof}
Supongamos que $i = \inf A$.

(i) Como $i$ es el ínfimo de $A$, por definición $i$ es cota inferior de $A$.
Es decir, para todo $a \in A$ se cumple $i \le a$.

(ii) Sea $\varepsilon > 0$. Supongamos, buscando una contradicción, que no existe
$a \in A$ tal que $i \le a < i + \varepsilon$. Entonces, para todo $a \in A$
se cumple $a \ge i + \varepsilon$, de modo que $i + \varepsilon$ es una cota
inferior de $A$. Como además $\varepsilon>0$, tenemos $i + \varepsilon > i$,
lo que contradice que $i$ es la mayor de las cotas inferiores de $A$.
Por lo tanto, debe existir $a \in A$ tal que $i \le a < i + \varepsilon$.

Recíprocamente, supongamos que se verifican:
(i) $i \le a$ para todo $a \in A$;
(ii) para todo $\varepsilon>0$ existe $a \in A$ tal que $i \le a < i + \varepsilon$.

De (i) se sigue que $i$ es cota inferior de $A$.
Sea $i'$ otra cota inferior de $A$. Queremos ver que $i' \le i$.
Supongamos, buscando una contradicción, que $i' > i$.
Sea $\varepsilon = i' - i > 0$. Por (ii) existe $a \in A$ tal que
$i \le a < i + \varepsilon$. Como $i + \varepsilon = i'$, obtenemos
$a < i'$, lo cual contradice que $i'$ es cota inferior de $A$.
Por lo tanto $i' \le i$ y, en consecuencia, $i = \inf A$.
\end{proof}


\subsection{Supremo}
Sea $A \subset\mathbb{R}$, no vacío y acotado superiormente.
\[
s = \sup A \Longleftrightarrow
\begin{cases}
s \ge a & \text{para todo } a \in A,\\[4pt]
\forall \varepsilon > 0 \ \exists a \in A \text{ tal que } s - \varepsilon < a \le s.
\end{cases}
\]
\begin{proof}
Supongamos que $s = \sup A$.

(i) Como $s$ es el supremo de $A$, por definición $s$ es cota superior de $A$.
Es decir, para todo $a \in A$ se cumple $a \le s$.

(ii) Sea $\varepsilon>0$. Supongamos, buscando una contradicción, que no existe
$a \in A$ tal que $s-\varepsilon < a \le s$. Entonces, para todo $a \in A$
se cumple $a \le s-\varepsilon$, y por lo tanto $s-\varepsilon$ es una cota
superior de $A$. Como además $\varepsilon>0$, tenemos $s-\varepsilon < s$,
lo que contradice que $s$ es la menor de las cotas superiores de $A$.
Por lo tanto, debe existir $a \in A$ tal que $s-\varepsilon < a \le s$.

Supongamos que se cumplen:
\\
(i) $a \le s$ para todo $a \in A$\\
(ii) para todo $\varepsilon>0$ existe $a \in A$ tal que $s-\varepsilon < a \le s$.

Entonces (i) dice que $s$ es cota superior de $A$.
Sea ahora $s'$ otra cota superior de $A$. Queremos ver que $s \le s'$.
Supongamos, buscando una contradicción, que $s' < s$.
Sea $\varepsilon = s - s' > 0$. Por (ii) existe $a \in A$ tal que
$s-\varepsilon < a \le s$. Como $s-\varepsilon = s'$, obtenemos $s' < a$,
lo cual contradice que $s'$ es cota superior de $A$.
Por lo tanto $s \le s'$ y, en consecuencia, $s = \sup A$.
\end{proof}

\subsection{Principio de Arquímedes}

\subsubsection*{Versión 1}
\[
\forall x \in \R \ \exists n \in \N \text{ tal que } x \le n.
\]

\begin{proof}[Demostración]
Supongamos por el absurdo que el conjunto de los naturales $\N$ está acotado superiormente.
Como $\N$ es no vacío, por el axioma de completitud existe $s = \sup \N$.

Tomamos $\varepsilon = 1$. Por la propiedad caracterizadora del supremo, existe
$n \in \N$ tal que
\[
s - 1 < n \le s.
\]
De $s - 1 < n$ se sigue que $n + 1 > s$. Como $n \in \N$, también $n+1 \in \N$,
y por lo tanto hemos encontrado un número natural estrictamente mayor que $s$,
lo que contradice que $s$ sea cota superior de $\N$.

Esta contradicción muestra que $\N$ no está acotado superiormente, es decir,
para todo $M \in \R$ existe $n \in \N$ tal que $n > M$. En particular, dado
$x \in \R$, tomando $M = x$ obtenemos un $n \in \N$ con $n \ge x$, que es
justamente lo que afirma la versión 1.
\end{proof}


\subsubsection*{Versión 2}
\[
\forall y > 0 \ \exists n \in \N \text{ tal que } 0 < \frac{1}{n} < y.
\]

\begin{proof}[Demostración]
Sea $y > 0$. Por la Versión 1 del principio de Arquímedes aplicada a $x = 1/y$,
existe $n \in \N$ tal que
\[
n > \frac{1}{y}.
\]
Como $n > 0$, al invertir la desigualdad obtenemos
\[
0 < \frac{1}{n} < y.
\]
Por lo tanto, para todo $y > 0$ existe $n \in \N$ tal que $0 < \dfrac{1}{n} < y$,
como queríamos demostrar.
\end{proof}

\subsection{Sucesiones}

\begin{prop}
Sea $(a_n)_{n\in\N}$ una sucesión real y sea $l \in \R$ tal que
\[
\lim_{n\to\infty} a_n = l.
\]
Entonces la sucesión $(a_n)$ está acotada.
\end{prop}

\begin{proof}
Por hipótesis, $\lim_{n\to\infty} a_n = l$. Esto significa que
\[
\forall \varepsilon > 0 \ \exists N \in \N \ \text{tal que} \ \forall n \ge N:
\ |a_n - l| < \varepsilon.
\]

Tomamos ahora $\varepsilon = 1$. Entonces existe $N \in \N$ tal que para todo
$n \ge N$ se cumple
\[
|a_n - l| < 1.
\]
Por la desigualdad triangular,
\[
|a_n| = |(a_n - l) + l| \le |a_n - l| + |l| < 1 + |l|
\]
para todo $n \ge N$. Es decir, para todos los índices grandes,
\[
|a_n| \le |l| + 1.
\]

Consideremos ahora los primeros términos de la sucesión:
\[
a_1, a_2, \dots, a_{N-1}.
\]
Se trata de un conjunto finito de números reales, por lo que el conjunto
\[
\{|a_1|, |a_2|, \dots, |a_{N-1}|\}
\]
tiene un máximo. Sea
\[
M_0 = \max\{|a_1|, |a_2|, \dots, |a_{N-1}|\}
\]
(en el caso $N = 1$ podemos tomar, por ejemplo, $M_0 = 0$).

Definimos ahora
\[
M = \max\{M_0,\ |l| + 1\}.
\]

Entonces:
\\- Si $n < N$, se cumple $|a_n| \le M_0 \le M$.
\\- Si $n \ge N$, se cumple $|a_n| \le |l| + 1 \le M$.
\\
En ambos casos obtenemos
\[
|a_n| \le M \quad \text{para todo } n \in \N.
\]
Por lo tanto, la sucesión $(a_n)$ está acotada.
\end{proof}

\begin{prop}
Sea $(a_n)_{n\in\N}$ una sucesión real monótona creciente y acotada superiormente.
Sea
\[
s = \sup\{a_n : n \in \N\}.
\]
Entonces
\[
\lim_{n\to\infty} a_n = s.
\]
\end{prop}

\begin{proof}
Como $(a_n)$ está acotada superiormente, el conjunto
\[
A = \{a_n : n \in \N\}
\]
tiene supremo $s \in \R$.

Queremos probar que $\lim_{n\to\infty} a_n = s$, es decir,
\[
\forall \varepsilon > 0 \ \exists N \in \N \ \forall n \ge N:
\ |a_n - s| < \varepsilon.
\]

Sea entonces $\varepsilon > 0$ arbitrario. Por la propiedad
caracterizadora del supremo aplicada al conjunto $A$, existe
$N \in \N$ tal que
\[
s - \varepsilon < a_N \le s.
\]

Como la sucesión $(a_n)$ es monótona creciente, se cumple
\[
a_N \le a_n \le s \quad \text{para todo } n \ge N.
\]

De $a_n \le s$ obtenemos $a_n - s \le 0$, luego
\[
|a_n - s| = s - a_n.
\]
Además, de $a_N \le a_n$ se sigue
\[
s - a_n \le s - a_N.
\]
Juntando estas desigualdades,
\[
|a_n - s|
= s - a_n
\le s - a_N
< s - (s - \varepsilon)
= \varepsilon
\]
para todo $n \ge N$.

Por lo tanto, para todo $\varepsilon>0$ encontramos $N \in \N$ tal que
para todo $n \ge N$ se cumple $|a_n - s|<\varepsilon$, y esto significa
exactamente que $\lim_{n\to\infty} a_n = s$.
\end{proof}

\begin{prop}
Sea $(a_n)_{n\in\N}$ una sucesión real monótona decreciente y acotada inferiormente.
Sea
\[
i = \inf\{a_n : n \in \N\}.
\]
Entonces
\[
\lim_{n\to\infty} a_n = i.
\]
\end{prop}

\begin{proof}
Como $(a_n)$ está acotada inferiormente, el conjunto
\[
A = \{a_n : n \in \N\}
\]
tiene ínfimo $i \in \R$.

Queremos probar que $\lim_{n\to\infty} a_n = i$, es decir,
\[
\forall \varepsilon > 0 \ \exists N \in \N \ \forall n \ge N:
\ |a_n - i| < \varepsilon.
\]

Sea $\varepsilon>0$ arbitrario. Por la propiedad caracterizadora del
ínfimo aplicada al conjunto $A$, existe $N \in \N$ tal que
\[
i \le a_N < i + \varepsilon.
\]

Como la sucesión $(a_n)$ es monótona decreciente, se cumple
\[
i \le a_n \le a_N \quad \text{para todo } n \ge N.
\]

De $a_n \ge i$ se obtiene $a_n - i \ge 0$, luego
\[
|a_n - i| = a_n - i.
\]
Además, de $a_n \le a_N$ se sigue
\[
a_n - i \le a_N - i.
\]
Por lo tanto,
\[
|a_n - i|
= a_n - i
\le a_N - i
< (i + \varepsilon) - i
= \varepsilon
\]
para todo $n \ge N$.

Entonces, para todo $\varepsilon>0$ encontramos $N \in \N$ tal que
para todo $n \ge N$ se cumple $|a_n - i|<\varepsilon$, lo cual prueba
que $\lim_{n\to\infty} a_n = i$.
\end{proof}

\begin{prop}[Equivalencia del supremo]
Sea $A \subset \R$ un conjunto no vacío y acotado superiormente, y sea $s \in \R$.
Entonces
\[
s = \sup A
\]
si y sólo si se cumplen:
\begin{enumerate}[label=(\roman*)]
    \item $s$ es cota superior de $A$, es decir, $a \le s$ para todo $a \in A$;
    \item existe una sucesión $(a_n)_{n\in\N}$ con $a_n \in A$ para todo $n$, tal que
    \[
    \lim_{n\to\infty} a_n = s.
    \]
\end{enumerate}
\end{prop}

\begin{proof}
Supongamos primero que $s = \sup A$. Entonces, por definición de supremo,
$s$ es cota superior de $A$, con lo cual se cumple (i).

Nos queda probar (ii). Usamos la caracterización del supremo que ya vimos:
para todo $\varepsilon > 0$ existe $a \in A$ tal que
\[
s - \varepsilon < a \le s.
\]
Para cada $n \in \N$, aplicamos esta propiedad con $\varepsilon = \tfrac{1}{n}$.
Obtenemos así un elemento $a_n \in A$ tal que
\[
s - \frac{1}{n} < a_n \le s.
\]
Esto define una sucesión $(a_n)_{n\in\N}$ de elementos de $A$.

Veamos ahora que $\lim_{n\to\infty} a_n = s$. Sea $\varepsilon > 0$. Elegimos
$N \in \N$ tal que $\frac{1}{N} < \varepsilon$. Entonces, si $n \ge N$,
se tiene $\frac{1}{n} \le \frac{1}{N} < \varepsilon$, y por la construcción
de $(a_n)$ se cumple
\[
s - \frac{1}{n} < a_n \le s.
\]
De aquí, $s - \varepsilon = s - \frac{1}{n} < a_n \le s$, luego
\[
0 \le s - a_n < \varepsilon,
\]
lo que implica
\[
|a_n - s| = s - a_n < \varepsilon.
\]
Por lo tanto,
\[
\forall \varepsilon > 0\ \exists N \in \N\ \forall n \ge N:\ |a_n - s| < \varepsilon,
\]
es decir, $\lim_{n\to\infty} a_n = s$. Esto prueba (ii).

\medskip

Recíprocamente, supongamos que se cumplen (i) y (ii). Entonces $s$ es cota
superior de $A$. Para ver que $s = \sup A$, basta probar que $s$ es la menor
de las cotas superiores de $A$. Sea $M$ otra cota superior de $A$. Queremos
ver que $s \le M$.

Supongamos, buscando una contradicción, que $s > M$. Definimos
\[
\varepsilon = \frac{s - M}{2} > 0.
\]
Como $\lim_{n\to\infty} a_n = s$, existe $N \in \N$ tal que, para todo
$n \ge N$, se cumple
\[
|a_n - s| < \varepsilon.
\]
En particular, para $n \ge N$ tenemos
\[
a_n > s - \varepsilon = s - \frac{s - M}{2}
= \frac{2s - s + M}{2}
= \frac{s + M}{2}.
\]
Pero, como $s > M$, se tiene
\[
\frac{s + M}{2} > M,
\]
de modo que
\[
a_n > \frac{s + M}{2} > M
\]
para todo $n \ge N$. Sin embargo, $a_n \in A$ y $M$ es cota superior de $A$,
luego debería cumplirse $a_n \le M$ para todo $n$, lo que contradice la
desigualdad anterior.

Esta contradicción muestra que no puede ocurrir $s > M$, por lo que necesariamente
$s \le M$. Como $M$ era una cota superior cualquiera de $A$, concluimos que
$s$ es la menor de las cotas superiores de $A$, es decir, $s = \sup A$.

Queda así demostrada la equivalencia.
\end{proof}

\begin{prop}
Sea $(a_n)_{n\in\N}$ una sucesión real y sea $(a_{n_j})_{j\in\N}$ una subsucesión de $(a_n)$,
donde $(n_j)_{j\in\N}$ es una sucesión estrictamente creciente de números naturales.
Si
\[
\lim_{n\to\infty} a_n = l,
\]
entonces
\[
\lim_{j\to\infty} a_{n_j} = l.
\]
\end{prop}

\begin{proof}
Supongamos que $\lim_{n\to\infty} a_n = l$. Por definición de límite, esto significa que
\[
\forall \varepsilon > 0 \ \exists N \in \N \ \text{tal que} \ \forall n \ge N:
\ |a_n - l| < \varepsilon.
\]

Sea ahora $\varepsilon > 0$ fijo. Por la hipótesis anterior, existe $N \in \N$ tal que
\[
n \ge N \ \Rightarrow\ |a_n - l| < \varepsilon.
\]

Como $(n_j)_{j\in\N}$ es una sucesión estrictamente creciente de números naturales,
podemos elegir $J \in \N$ tal que
\[
n_J \ge N.
\]
Entonces, para todo $j \ge J$ se tiene
\[
n_j \ge n_J \ge N.
\]
Aplicando la propiedad de arriba a $n = n_j$, obtenemos
\[
|a_{n_j} - l| < \varepsilon \quad \text{para todo } j \ge J.
\]

Por lo tanto,
\[
\forall \varepsilon > 0 \ \exists J \in \N \ \forall j \ge J:\ |a_{n_j} - l| < \varepsilon,
\]
lo cual significa exactamente que
\[
\lim_{j\to\infty} a_{n_j} = l.
\]
\end{proof}