\subsection{Conjuntos nulos}

\begin{defi}[Conjunto nulo]
Sea $A \subseteq \R$. Decimos que $A$ es un \emph{conjunto nulo}
si para todo $\varepsilon > 0$ existen intervalos abiertos
contables $(U_n)_{n \in \N}$ tales que
\[
A \subseteq \bigcup_{n\in\N} U_n
\quad\text{y}\quad
\sum_{n\in\N} \operatorname{long}(U_n) < \varepsilon.
\]
\end{defi}

\subsection{$\sigma$-álgebras y conjuntos medibles de Lebesgue}

\begin{defi}[$\sigma$-álgebra]
Sea $X$ un conjunto y sea $\mathcal{A} \subseteq \mathcal{P}(X)$
una familia de subconjuntos de $X$. Decimos que $\mathcal{A}$ es
una \emph{$\sigma$-álgebra} si se verifica:
\begin{enumerate}[label=(\roman*)]
    \item $X \in \mathcal{A}$;
    \item si $A \in \mathcal{A}$, entonces $A^c = X \setminus A \in \mathcal{A}$;
    \item si $(A_n)_{n\in\N} \subseteq \mathcal{A}$, entonces
    \[
    \bigcup_{n\in\N} A_n \in \mathcal{A}
    \quad\text{y}\quad
    \bigcap_{n\in\N} A_n \in \mathcal{A}.
    \]
\end{enumerate}
\end{defi}

\begin{defi}[Conjuntos medibles de Lebesgue]
Sea $\mathcal{M}$ la $\sigma$-álgebra generada por los intervalos abiertos
y los conjuntos nulos de $\R$. A $\mathcal{M}$ la llamamos
\emph{$\sigma$-álgebra de conjuntos medibles de Lebesgue} en $\R$.

Si $I$ es un intervalo de $\R$, denotamos por $\mathcal{M}(I)$ a la
$\sigma$-álgebra de subconjuntos medibles de Lebesgue de $I$.
\end{defi}

\subsection{Medida de Lebesgue}

\begin{teo}[Existencia y unicidad de la medida de Lebesgue]
Existe una única función
\[
\mu : \mathcal{M} \longrightarrow [0,+\infty]
\]
tal que:
\begin{enumerate}[label=(\roman*)]
    \item si $A = (a,b)$ es un intervalo abierto acotado, entonces
    \[
    \mu(A) = b-a;
    \]
    \item si $(A_n)_{n\in\N} \subseteq \mathcal{M}$, entonces
    \[
    \mu\Bigl(\bigcup_{n\in\N} A_n\Bigr)
    \le \sum_{n\in\N} \mu(A_n);
    \]
    \item si, además, los $A_n$ son dos a dos disjuntos, entonces
    \[
    \mu\Bigl(\bigcup_{n\in\N} A_n\Bigr)
    = \sum_{n\in\N} \mu(A_n);
    \]
    \item para todo $A \in \mathcal{M}$ se cumple la propiedad de
    \emph{regularidad exterior}:
    \[
    \mu(A) = \inf\{\mu(U) : A \subseteq U,\ U \text{ abierto}\}.
    \]
\end{enumerate}
La función $\mu$ se llama \emph{medida de Lebesgue}.
\end{teo}

\subsection{Propiedades básicas de la medida de Lebesgue}

En esta subsección trabajamos, salvo aclaración en contrario,
en el intervalo $I = [0,1]$ con la medida de Lebesgue, y escribimos
$\mathcal{M}(I)$ para la $\sigma$-álgebra de subconjuntos medibles de $I$.

\begin{teo}[Propiedades básicas]
Sea $\mu : \mathcal{M}(I) \to [0,+\infty]$ la medida de Lebesgue. Entonces:
\begin{enumerate}[label=(\roman*)]
    \item \emph{Monotonía:} si $A,B \in \mathcal{M}(I)$ y $A \subseteq B$,
    entonces
    \[
    \mu(A) \le \mu(B).
    \]
    \item \emph{Conjuntos nulos:} si $A \subseteq \R$ es un conjunto nulo,
    entonces $A \in \mathcal{M}$ y $\mu(A) = 0$. Recíprocamente,
    si $A \in \mathcal{M}$ y $\mu(A) = 0$, entonces $A$ es un conjunto nulo.
    \item \emph{Invariancia por traslaciones:} dados $A \in \mathcal{M}$
    y $c \in \R$, se tiene $A+c := \{x+c : x\in A\} \in \mathcal{M}$ y
    \[
    \mu(A+c) = \mu(A).
    \]
\end{enumerate}
\end{teo}

\begin{proof}[Idea de la demostración]
La monotonía se obtiene escribiendo $B$ como unión disjunta de $A$ y
$B \setminus A$ y usando la $\sigma$-aditividad. Las afirmaciones sobre
conjuntos nulos se deducen de la relación entre definición de conjunto
nulo y la regularidad exterior. La invariancia por traslaciones se
prueba primero en intervalos (donde es obvia) y luego se extiende a
$\mathcal{M}$ usando que ésta es la $\sigma$-álgebra generada por
intervalos y conjuntos nulos, y que la traslación preserva nulos.
\end{proof}

\begin{prop}
Sea $I = [0,1]$ y sea $\mu$ la medida de Lebesgue en $\mathcal{M}(I)$.
Si $A,B \in \mathcal{M}(I)$, entonces:
\begin{enumerate}[label=(\roman*)]
    \item $A \setminus B \in \mathcal{M}(I)$;
    \item
    \[
        \mu(A \cup B) = \mu(A \setminus B) + \mu(B).
    \]
\end{enumerate}
En particular,
\[
\mu(A^c) = 1 - \mu(A), \qquad A^c = I \setminus A.
\]
\end{prop}

\begin{proof}
(i) Como $\mathcal{M}(I)$ es una $\sigma$-álgebra,
es cerrada por complementos e intersecciones. Observamos que
\[
A \setminus B = A \cap B^c,
\]
por lo que $A \setminus B \in \mathcal{M}(I)$.

\medskip

(ii) Notamos que
\[
A \cup B = (A \setminus B) \cup B
\]
y que $(A \setminus B)$ y $B$ son disjuntos. Por $\sigma$-aditividad,
\[
\mu(A \cup B) = \mu(A \setminus B) + \mu(B).
\]

La igualdad $\mu(A^c) = 1 - \mu(A)$ se obtiene aplicando esta fórmula
con $A$ y $A^c$ observando que $I = A \cup A^c$ y $\mu(I)=1$.
\end{proof}

\subsection{Regularidad de la medida de Lebesgue}

\begin{prop}[Regularidad exterior]
Sea $A \in \mathcal{M}(I)$. Entonces
\[
\mu(A) = \inf\{\mu(U) : A \subseteq U,\ U \text{ abierto en } I\}.
\]
\end{prop}

\begin{proof}[Esbozo]
Esta propiedad forma parte de la construcción misma de la medida de Lebesgue
(en el teorema de existencia). La desigualdad
\[
\mu(A) \le \inf\{\mu(U)\}
\]
se sigue de la monotonía: si $A \subseteq U$, entonces $\mu(A) \le \mu(U)$.
En la construcción de la medida se verifica además que para todo
$\varepsilon>0$ existe un abierto $U \supseteq A$ tal que
$\mu(U) < \mu(A) + \varepsilon$, lo que da la igualdad.
\end{proof}

\begin{prop}[Regularidad interior]
Sea $A \in \mathcal{M}(I)$. Entonces
\[
\mu(A) =
\sup\{\mu(F) : F \subseteq A,\ F \text{ cerrado en } I\}.
\]
\end{prop}

\begin{proof}
Por monotonía, si $F \subseteq A$ entonces $\mu(F) \le \mu(A)$,
de donde
\[
\sup\{\mu(F) : F \subseteq A,\ F \text{ cerrado}\} \le \mu(A).
\]

Para la otra desigualdad, sea $\varepsilon>0$.  
Por regularidad exterior aplicada a $A^c$, existe un abierto
$U \supseteq A^c$ tal que
\[
\mu(U) < \mu(A^c) + \varepsilon.
\]
Tomando complementos en $I$, el conjunto
\[
F := U^c = I \setminus U
\]
es cerrado y satisface $F \subseteq A$.

Además, por la proposición anterior,
\[
\mu(F) = \mu(I) - \mu(U).
\]
Como $\mu(I)=1$ y $\mu(A^c) = 1 - \mu(A)$, obtenemos
\[
\mu(F) = 1 - \mu(U)
> 1 - (\mu(A^c) + \varepsilon)
= \mu(A) - \varepsilon.
\]
Por lo tanto, para todo $\varepsilon>0$ existe un cerrado $F \subseteq A$
tal que $\mu(F) > \mu(A) - \varepsilon$, lo que implica
\[
\mu(A) \le
\sup\{\mu(F) : F \subseteq A,\ F \text{ cerrado}\}.
\]
\end{proof}

\begin{prop}[Regularidad fuerte]
Sea $A \in \mathcal{M}(I)$ y $\varepsilon>0$. Entonces existen
un cerrado $C$ y un abierto $U$ tales que
\[
C \subseteq A \subseteq U
\quad\text{y}\quad
\mu(A) - \varepsilon < \mu(C) \le \mu(A) \le \mu(U) < \mu(A) + \varepsilon.
\]
Además, $U$ puede elegirse como una unión numerable de intervalos
abiertos dos a dos disjuntos.
\end{prop}

\begin{proof}
Sea $\varepsilon>0$.  
Por regularidad interior, existe un cerrado $C \subseteq A$ tal que
\[
\mu(C) > \mu(A) - \frac{\varepsilon}{2}.
\]
Por regularidad exterior, existe un abierto $U \supseteq A$ tal que
\[
\mu(U) < \mu(A) + \frac{\varepsilon}{2}.
\]
De aquí se obtiene
\[
\mu(A) - \varepsilon
< \mu(A) - \frac{\varepsilon}{2}
< \mu(C) \le \mu(A) \le \mu(U)
< \mu(A) + \frac{\varepsilon}{2}
< \mu(A) + \varepsilon.
\]

El hecho de que cualquier abierto $U \subseteq I$ puede escribirse
como unión numerable de intervalos abiertos dos a dos disjuntos es
un resultado clásico de análisis real (se prueba usando que $U$ es
una unión numerable de componentes conexas, que en $\R$ son intervalos).
Aplicándolo a este $U$, se obtiene la última afirmación.
\end{proof}

\subsection{Continuidad de la medida}

\begin{teo}[Continuidad de la medida]
Sea $\{A_n\}_{n\in\N} \subseteq \mathcal{M}(I)$. Entonces:
\begin{enumerate}[label=(\roman*)]
    \item (Continuidad desde abajo) Si
    \[
    A_1 \subseteq A_2 \subseteq \cdots \subseteq A_n \subseteq \cdots
    \]
    y $A = \bigcup_{n\in\N} A_n$, entonces
    \[
    \mu(A) = \lim_{n\to\infty} \mu(A_n).
    \]
    \item (Continuidad desde arriba) Si
    \[
    B_1 \supseteq B_2 \supseteq \cdots \supseteq B_n \supseteq \cdots
    \]
    y $B = \bigcap_{n\in\N} B_n$, con $\mu(B_1)<\infty$, entonces
    \[
    \mu(B) = \lim_{n\to\infty} \mu(B_n).
    \]
\end{enumerate}
\end{teo}

\begin{proof}
(i) Definimos
\[
C_1 = A_1, \qquad
C_n = A_n \setminus A_{n-1} \quad (n \ge 2).
\]
Entonces los conjuntos $C_n$ son dos a dos disjuntos y
\[
A = \bigcup_{n\in\N} A_n = \bigcup_{n\in\N} C_n.
\]
Por $\sigma$-aditividad,
\[
\mu(A) = \sum_{n=1}^\infty \mu(C_n).
\]
Además, para cada $n$,
\[
\mu(A_n) = \mu\Bigl(\bigcup_{k=1}^n C_k\Bigr)
= \sum_{k=1}^n \mu(C_k).
\]
La sucesión de sumas parciales converge a la suma infinita, luego
\[
\lim_{n\to\infty} \mu(A_n)
= \sum_{k=1}^\infty \mu(C_k)
= \mu(A).
\]

\medskip

(ii) Definimos
\[
A_n = B_1 \setminus B_n \quad (n\in\N).
\]
Entonces $A_1 \subseteq A_2 \subseteq \cdots$ y
\[
\bigcup_{n\in\N} A_n = B_1 \setminus \bigcap_{n\in\N} B_n
= B_1 \setminus B.
\]
Aplicando (i) a la familia $(A_n)$,
\[
\mu(B_1 \setminus B)
= \lim_{n\to\infty} \mu(A_n)
= \lim_{n\to\infty} \bigl(\mu(B_1) - \mu(B_n)\bigr),
\]
donde usamos que $A_n = B_1 \setminus B_n$ y $\mu(B_1)<\infty$.
Entonces
\[
\mu(B_1) - \mu(B)
= \lim_{n\to\infty} \bigl(\mu(B_1) - \mu(B_n)\bigr).
\]
Restando $\mu(B_1)$ en ambos lados se obtiene
\[
\mu(B) = \lim_{n\to\infty} \mu(B_n).
\]
\end{proof}