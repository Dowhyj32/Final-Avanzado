Sea $(X,\mathcal A)$ un espacio medible (es decir, $X$ es un conjunto y
$\mathcal A$ una $\sigma$-álgebra de subconjuntos de $X$).

\begin{defi}[Función medible real]
Una función $f : X \to \R$ se llama \emph{(Lebesgue) medible} si para
todo $a \in \R$ se cumple
\[
\{x \in X : f(x) < a\} \in \mathcal A.
\]
\end{defi}

\begin{teo}[Caracterizaciones de función medible]
Sea $(X,\mathcal A)$ un espacio medible y $f : X \to \R$ una función.
Son equivalentes las siguientes afirmaciones:
\begin{enumerate}[label=(\roman*)]
    \item Para todo $a \in \R$ se tiene
    \[
    \{x \in X : f(x) < a\} \in \mathcal A.
    \]
    \item Para todo $a \in \R$ se tiene
    \[
    \{x \in X : f(x) \le a\} \in \mathcal A.
    \]
    \item Para todo $a \in \R$ se tiene
    \[
    \{x \in X : f(x) > a\} \in \mathcal A.
    \]
    \item Para todo $a \in \R$ se tiene
    \[
    \{x \in X : f(x) \ge a\} \in \mathcal A.
    \]
\end{enumerate}
En particular, cualquiera de estas condiciones puede tomarse como
definición de función medible.
\end{teo}

\begin{proof}
$(i) \Rightarrow (ii)$.
Sea $a \in \R$. Mostramos que
\[
\{f \le a\} = \bigcap_{n=1}^{\infty} \{f < a + \tfrac{1}{n}\}.
\]

Primero, si $x \in \{f \le a\}$, entonces $f(x) \le a < a + \tfrac{1}{n}$
para todo $n \in \N$, y por lo tanto $x \in \{f < a + 1/n\}$ para todo $n$.
Esto prueba la inclusión
\[
\{f \le a\} \subseteq \bigcap_{n=1}^{\infty} \{f < a + \tfrac{1}{n}\}.
\]

Recíprocamente, sea $x$ tal que $x \in \{f < a + 1/n\}$ para todo $n$.
Entonces
\[
f(x) < a + \frac{1}{n} \quad \text{para todo } n \in \N.
\]
Supongamos, por absurdo, que $f(x) > a$. Entonces $f(x) - a > 0$ y
podemos definir
\[
\varepsilon = \frac{f(x) - a}{2} > 0.
\]
Tomamos $n$ suficientemente grande tal que $\tfrac{1}{n} < \varepsilon$.
Entonces
\[
a + \frac{1}{n} < a + \varepsilon = \frac{a + f(x)}{2} < f(x),
\]
lo cual contradice $f(x) < a + 1/n$. Por lo tanto no puede ser
$f(x) > a$, y forzosamente $f(x) \le a$, es decir $x \in \{f \le a\}$.

Concluimos la igualdad de conjuntos. Cada conjunto
$\{f < a + 1/n\}$ es medible por hipótesis (i), y las intersecciones
numerables de conjuntos medibles pertenecen a $\mathcal A$. Por lo tanto
$\{f \le a\}$ es medible. Así, (ii) se cumple.

\medskip

$(ii) \Rightarrow (iii)$.
Sea $a \in \R$. Observamos que
\[
\{f > a\} = X \setminus \{f \le a\}.
\]
En efecto, si $f(x) > a$ entonces $f(x)$ no puede satisfacer
$f(x) \le a$, y viceversa. Como $\mathcal A$ es una $\sigma$-álgebra,
es estable por complementos; de la medibilidad de $\{f \le a\}$ se
deduce la medibilidad de $\{f > a\}$. Luego (iii) se verifica.

\medskip

$(iii) \Rightarrow (iv)$.
Sea $a \in \R$. Mostramos que
\[
\{f \ge a\}
= \bigcap_{n=1}^{\infty} \{f > a - \tfrac{1}{n}\}.
\]

Si $x \in \{f \ge a\}$, entonces $f(x) \ge a > a - \tfrac{1}{n}$ para
todo $n$, y en particular $x \in \{f > a - 1/n\}$ para todo $n$.
Esto prueba la inclusión
\[
\{f \ge a\} \subseteq \bigcap_{n=1}^{\infty} \{f > a - \tfrac{1}{n}\}.
\]

Para la inclusión recíproca, sea $x$ tal que
$x \in \{f > a - 1/n\}$ para todo $n$, es decir,
\[
f(x) > a - \frac{1}{n} \quad \forall n \in \N.
\]
Supongamos, por absurdo, que $f(x) < a$. Entonces $a - f(x) > 0$ y
podemos definir
\[
\varepsilon = \frac{a - f(x)}{2} > 0.
\]
Elegimos $n$ suficientemente grande tal que $\tfrac{1}{n} < \varepsilon$.
Entonces
\[
a - \frac{1}{n} > a - \varepsilon = \frac{a + f(x)}{2} > f(x),
\]
lo cual contradice $f(x) > a - 1/n$. Por lo tanto no puede ser
$f(x) < a$, y debe cumplirse $f(x) \ge a$, es decir $x \in \{f \ge a\}$.

Hemos probado la igualdad. Cada conjunto $\{f > a - 1/n\}$ es medible
por hipótesis (iii), y la intersección numerable de medibles también
lo es; por lo tanto $\{f \ge a\}$ es medible. Se verifica (iv).

\medskip

$(iv) \Rightarrow (i)$.
Sea $a \in \R$. Notamos que
\[
\{f < a\} = X \setminus \{f \ge a\}.
\]
En efecto, si $f(x) < a$ entonces no puede ser $f(x)\ge a$, y si
$f(x)\ge a$ entonces no puede ser $f(x)<a$. Si $\{f \ge a\}$ es medible
por (iv), su complemento también pertenece a $\mathcal A$, de modo que
$\{f < a\}$ es medible.

Con esto cerramos el ciclo
\[
(i) \Rightarrow (ii) \Rightarrow (iii) \Rightarrow (iv) \Rightarrow (i),
\]
y las cuatro condiciones son equivalentes.
\end{proof}