\section{Unidad 5: Compacidad}

\subsection{Definiciones básicas}

\begin{defi}[Conjunto compacto]
Sea $(E,d)$ un espacio métrico y $K \subseteq E$.
Decimos que $K$ es \emph{compacto} si para toda familia de abiertos
$\{G_i\}_{i \in I}$ de $E$ tal que
\[
K \subseteq \bigcup_{i \in I} G_i,
\]
existe un subíndice finito $i_1,\dots,i_n \in I$ con
\[
K \subseteq \bigcup_{k=1}^n G_{i_k}.
\]
Es decir, todo recubrimiento abierto de $K$ admite un subcubrimiento
finito.
\end{defi}

\begin{defi}[Compacidad secuencial]
Sea $(E,d)$ un espacio métrico y $K \subseteq E$.
Decimos que $K$ es \emph{secuencialmente compacto} si para toda sucesión
$(x_n)_{n\in\N}$ con $x_n \in K$ para todo $n$, existe una subsucesión
$(x_{n_j})_{j\in\N}$ y un punto $x \in K$ tales que
\[
x_{n_j} \xrightarrow[j\to\infty]{} x.
\]
\end{defi}

\begin{defi}[Conjunto totalmente acotado]
Sea $(E,d)$ un espacio métrico y $A \subseteq E$.
Decimos que $A$ es \emph{totalmente acotado} (T.T.A.) si
\[
\forall \varepsilon>0\ \exists x_1,\dots,x_{n(\varepsilon)} \in E
\ \text{tales que}\ 
A \subseteq \bigcup_{k=1}^{n(\varepsilon)} B(x_k,\varepsilon),
\]
donde $B(x_k,\varepsilon)=\{y\in E : d(x_k,y)<\varepsilon\}$.
\end{defi}

\begin{defi}[Punto de acumulación]
Sea $(E,d)$ un espacio métrico y $A \subseteq E$.
Un punto $x \in E$ se llama \emph{punto de acumulación} (o punto límite)
de $A$ si
\[
\forall r>0 \quad \bigl(B(x,r)\setminus\{x\}\bigr) \cap A \neq \varnothing.
\]
Denotamos por $A'$ al conjunto de puntos de acumulación de $A$.
\end{defi}

\begin{prop}[Caracterización secuencial de puntos de acumulación]
Sea $A \subseteq E$ y $x \in E$. Entonces
\[
x \in A' \iff \exists (a_n)_{n\in\N} \subseteq A\setminus\{x\}
\text{ tal que } a_n \to x.
\]
\end{prop}

\subsection{Compacidad y puntos de acumulación}

\begin{teo}[Equivalencias de compacidad en espacios métricos]
Sea $(E,d)$ un espacio métrico y $K \subseteq E$. Son equivalentes:
\begin{enumerate}[label=(\roman*)]
    \item $K$ es compacto.
    \item $K$ es secuencialmente compacto.
    \item Todo subconjunto infinito $A \subseteq K$ tiene al menos un punto
    de acumulación en $K$, es decir, $A' \cap K \neq \varnothing$.
\end{enumerate}
\end{teo}

\begin{proof}
$(i) \Rightarrow (ii)$.
Sea $(x_n)$ una sucesión en $K$. Si el conjunto de valores
$A=\{x_n:n\in\N\}$ es finito, alguna de sus constantes se repite
infinitas veces y obtenemos una subsucesión constante, luego convergente.

Si $A$ es infinito, como $K$ es compacto, también lo es $\overline{A}$,
y por tanto $A$ tiene un punto de acumulación $x\in K$.
Por la caracterización secuencial de puntos de acumulación, existe una
subsucesión $(x_{n_j})$ con $x_{n_j}\to x\in K$.
En ambos casos, $(x_n)$ admite una subsucesión convergente en $K$.

\medskip

$(ii) \Rightarrow (iii)$.
Sea $A\subseteq K$ infinito. Elegimos una sucesión de puntos distintos
$(x_n)\subseteq A$ (por ejemplo, una enumeración inyectiva de $A$).
Por (ii), existe una subsucesión $(x_{n_j})$ y un punto $x\in K$ tales
que $x_{n_j}\to x$. Como todos los $x_{n_j}\in A$, la caracterización
secuencial de puntos de acumulación dice que $x\in A'$. Por tanto
$A'\cap K\neq \varnothing$.

\medskip

$(iii) \Rightarrow (ii)$.
Sea $(x_n)$ una sucesión en $K$ y sea $A=\{x_n:n\in\N\}\subseteq K$.
Si $A$ es finito, como antes obtenemos una subsucesión constante y
convergente.

Si $A$ es infinito, por (iii) existe $x\in A'\cap K$. Entonces, por la
caracterización secuencial de puntos de acumulación, existe una
subsucesión $(x_{n_j})$ de $(x_n)$ tal que $x_{n_j}\to x\in K$.
En todo caso, $(x_n)$ admite una subsucesión convergente en $K$.

\medskip

$(ii) \Rightarrow (i)$.
Probamos la contrarrecíproca. Supongamos que $K$ no es compacto.
Entonces existe un recubrimiento abierto $\{G_i\}_{i\in I}$ de $K$
que no admite subcubrimiento finito.

Construimos inductivamente una sucesión $(x_n)$ en $K$ del siguiente modo:

- Elegimos $x_1\in K$ y un índice $i_1$ con $x_1\in G_{i_1}$.
- Supuesto elegido $x_1,\dots,x_n$ con índices $i_1,\dots,i_n$ tales
  que $x_k\in G_{i_k}$, como
  \(
    K \nsubseteq \bigcup_{k=1}^n G_{i_k}
  \)
  existe
  \(
    x_{n+1}\in K \setminus \bigcup_{k=1}^n G_{i_k}.
  \)
  Elegimos $i_{n+1}$ con $x_{n+1}\in G_{i_{n+1}}$.

Entonces cada $x_n$ pertenece a un abierto nuevo de la familia, distinto
de los anteriores. Sea $(x_{n_j})$ una subsucesión cualquiera.
Si fuera convergente, digamos $x_{n_j}\to x\in K$, algún abierto $G_i$
contendría a $x$ y a todos los términos suficientemente avanzados de
la subsucesión; en particular, ese solo $G_i$ cubriría casi todos los
$x_n$, y un número finito de abiertos de la familia cubriría $K$,
contradiciendo la elección de $\{G_i\}$ sin subcubrimiento finito.
Luego $(x_n)$ no posee subsucesión convergente, y $K$ no es secuencialmente
compacto.

La contrarrecíproca muestra que (ii) implica (i).
\end{proof}

\subsection{Compacidad, completitud y total acotación}

\begin{teo}
Sea $(E,d)$ un espacio métrico y $K \subseteq E$. Entonces:
\begin{enumerate}[label=(\roman*)]
    \item Si $K$ es compacto, entonces $K$ es completo y totalmente acotado.
    \item Si $K$ es completo y totalmente acotado, entonces $K$ es compacto.
\end{enumerate}
En particular,
\[
K \text{ compacto} \iff
K \text{ completo y totalmente acotado}.
\]
\end{teo}

\begin{proof}
(i) Si $K$ es compacto, por el teorema anterior es secuencialmente
compacto. Toda sucesión de Cauchy en $K$ admite una subsucesión convergente;
un argumento estándar muestra que la sucesión completa converge al mismo
límite, que pertenece a $K$, de modo que $K$ es completo.

Para ver que $K$ es T.T.A., sea $\varepsilon>0$ y consideremos la familia
de bolas $\{B(x,\varepsilon):x\in K\}$, que es un recubrimiento abierto
de $K$. Por compacidad, existe un subcubrimiento finito
$K\subset \bigcup_{k=1}^n B(x_k,\varepsilon)$, que es precisamente la
definición de T.T.A.

\medskip

(ii) Supongamos que $K$ es completo y totalmente acotado.
Sea $(x_n)$ una sucesión en $K$. Por total acotación, usando el clásico
método de “subsucesiones encajadas”, se puede extraer una subsucesión
$(x_{n_j})$ que resulta ser de Cauchy. Como $K$ es completo, esa
subsucesión converge en $K$. Así, toda sucesión en $K$ tiene una
subsucesión convergente en $K$, es decir, $K$ es secuencialmente compacto.
Por el teorema anterior, $K$ es compacto.
\end{proof}

\begin{prop}
Si $(E,d)$ es un espacio métrico y $K \subseteq E$ es compacto, entonces
$K$ es cerrado y acotado.
\end{prop}

\begin{proof}
Ya vimos que todo compacto es totalmente acotado, luego es acotado.

Resta ver que es cerrado. Sea $\overline{K}$ la clausura de $K$.
Si $x\in \overline{K}$, existe una sucesión $(x_n)\subseteq K$
tal que $x_n\to x$. Como $K$ es compacto, es secuencialmente compacto,
por lo que $(x_n)$ admite una subsucesión convergente con límite en $K$;
la unicidad del límite en espacios métricos fuerza que ese límite sea $x$,
por lo que $x\in K$. Se obtiene $\overline{K}\subseteq K$, y como siempre
$K\subseteq\overline{K}$, concluimos $\overline{K}=K$, es decir, $K$ es
cerrado.
\end{proof}

\begin{teo}[Heine--Borel en $\R^m$]
Sea $K \subseteq \R^m$ con la métrica euclídea usual. Entonces
\[
K \text{ es compacto} \iff K \text{ es cerrado y acotado}.
\]
\end{teo}

\subsection{Propiedades estructurales de compactos y T.T.A.}

\begin{prop}[Subconjunto cerrado de un compacto]
Sea $(E,d)$ un espacio métrico y $K \subseteq E$ compacto.
Si $F \subseteq K$ es cerrado (en $E$), entonces $F$ es compacto.
\end{prop}

\begin{proof}
Sea $\{G_\alpha\}_{\alpha\in A}$ un recubrimiento abierto de $F$ en $E$.
Como $F$ es cerrado, $E\setminus F$ es abierto. Entonces
\[
\{G_\alpha : \alpha\in A\} \cup \{E\setminus F\}
\]
es un recubrimiento abierto de $K$. Por compacidad, existe una subfamilia
finita que recubre $K$, y por lo tanto $F$ queda cubierto por una subfamilia
finita de los $G_\alpha$. Así, $F$ es compacto.
\end{proof}

\begin{cor}
La intersección arbitraria de subconjuntos compactos de un espacio
métrico es un conjunto compacto (en particular, $\varnothing$ es compacto).
\end{cor}

\begin{proof}
Si $\{K_i\}_{i\in I}$ son compactos, cada uno es cerrado en $E$.
La intersección $K = \bigcap_{i\in I} K_i$ es cerrada y satisface
$K\subseteq K_{i_0}$ para cualquier índice fijo $i_0$. Como $K$ es un
subconjunto cerrado de $K_{i_0}$, por la proposición anterior $K$ es compacto.
El caso $K=\varnothing$ también es compacto por definición.
\end{proof}

\begin{prop}[Propiedades de conjuntos totalmente acotados]
Sea $(E,d)$ un espacio métrico.
\begin{enumerate}[label=(\roman*)]
    \item Si $A \subseteq B \subseteq E$ y $B$ es T.T.A., entonces $A$ es T.T.A.
    \item Si $A$ es T.T.A., entonces $\overline{A}$ es T.T.A.
    \item Si $A$ y $B$ son T.T.A., entonces $A \cup B$ es T.T.A.
    (en general, la unión finita de conjuntos T.T.A. es T.T.A.).
    \item Si $A$ es T.T.A., entonces $A$ es acotado.
\end{enumerate}
\end{prop}

\begin{proof}
(i) y (iii) se obtienen directamente de la definición observando que
un recubrimiento finito de $B$ o de $A$ y $B$ provee uno de $A$ o
de $A\cup B$.

(ii) Dado $\varepsilon>0$, se cubre $A$ con finitas bolas
$B(x_k,\varepsilon/2)$; se comprueba que la clausura de cada una de ellas
queda incluida en $B(x_k,\varepsilon)$, y se deduce que
$\overline{A}$ queda cubierta por la unión finita
\(\bigcup_k B(x_k,\varepsilon)\).

(iv) Tomando $\varepsilon=1$, se ve que $A$ queda contenido en la unión
finita de bolas de radio $1$, por lo que todo punto de $A$ está a distancia
acotada de cualquiera de sus centros; esto da una cota global para $A$.
\end{proof}

\begin{prop}[Imagen de T.T.A. por función uniformemente continua]
Sean $(E,d)$ y $(E',d')$ espacios métricos, $A \subseteq E$
totalmente acotado y $f : E \to E'$ uniformemente continua.
Entonces $f(A)$ es totalmente acotado en $(E',d')$.
\end{prop}

\begin{proof}
Sea $\varepsilon>0$. Por uniformidad de $f$, existe $\delta>0$ tal que
\[
d(x,y)<\delta \Rightarrow d'(f(x),f(y))<\varepsilon
\quad \forall x,y\in E.
\]
Como $A$ es T.T.A., existen $x_1,\dots,x_n$ con
\[
A \subseteq \bigcup_{k=1}^n B(x_k,\delta).
\]
Entonces
\[
f(A) \subseteq \bigcup_{k=1}^n f(B(x_k,\delta))
\subseteq \bigcup_{k=1}^n B_{d'}(f(x_k),\varepsilon).
\]
Hemos cubierto $f(A)$ con finit­as bolas de radio $\varepsilon$; como
$\varepsilon$ era arbitrario, $f(A)$ es T.T.A.
\end{proof}

\subsection{Funciones continuas sobre compactos}

\begin{teo}[Imagen continua de un compacto]
Sean $(E,d)$ y $(E',d')$ espacios métricos y sea
$f : E \to E'$ continua. Si $K \subseteq E$ es compacto, entonces
$f(K)$ es compacto en $E'$.
\end{teo}

\begin{proof}
Sea $\{V_\alpha\}_{\alpha\in A}$ un recubrimiento abierto de $f(K)$ en $E'$.
Para cada $\alpha$ definimos $U_\alpha = f^{-1}(V_\alpha)$, que es abierto
en $E$ por continuidad de $f$. Además,
\[
K \subseteq \bigcup_{\alpha\in A} U_\alpha,
\]
pues si $x\in K$ entonces $f(x)\in f(K)\subseteq \bigcup V_\alpha$.

Como $K$ es compacto, existe un subcubrimiento finito
$K \subseteq \bigcup_{j=1}^n U_{\alpha_j}$. Aplicando $f$,
\[
f(K) \subseteq \bigcup_{j=1}^n f(U_{\alpha_j})
\subseteq \bigcup_{j=1}^n V_{\alpha_j},
\]
que es un subcubrimiento finito de $f(K)$. Así, $f(K)$ es compacto.
\end{proof}

\begin{cor}[Teorema de Weierstrass]
Sea $K \subseteq E$ compacto y $f : K \to \R$ continua.
Entonces:
\begin{enumerate}[label=(\roman*)]
    \item $f$ es acotada en $K$: existe $c>0$ tal que
    $|f(x)|\le c$ para todo $x\in K$;
    \item $f$ alcanza su máximo y su mínimo en $K$, es decir,
    existen $x_{\min},x_{\max}\in K$ con
    \[
    f(x_{\min}) \le f(x) \le f(x_{\max}) \quad \forall x\in K.
    \]
\end{enumerate}
\end{cor}

\begin{proof}
Por el teorema anterior, $f(K)$ es compacto en $\R$. Todo compacto
no vacío de $\R$ es cerrado y acotado, de modo que admite supremo e
ínfimo que pertenecen al conjunto; eso da el máximo y el mínimo.
La acotación se deduce de la existencia de supremo y de ínfimo.
\end{proof}

\begin{teo}[Continuidad uniforme en compactos]
Sean $(E,d)$ y $(E',d')$ espacios métricos y sea
$f : E \to E'$ continua. Si $E$ es compacto, entonces $f$ es
uniformemente continua.
\end{teo}

\begin{proof}
Sea $\varepsilon>0$. Para cada $x\in E$, por continuidad de $f$ en $x$
existe $\delta_x>0$ tal que
\[
d(x,y)<\delta_x \Rightarrow d'(f(x),f(y))<\varepsilon
\quad \forall y\in E.
\]
La familia de bolas $B(x,\delta_x)$ (o $B(x,\delta_x/2)$) es un
cubrimiento abierto de $E$. Por compacidad, existe un subcubrimiento
finito
\[
E \subseteq \bigcup_{k=1}^n B\bigl(x_k,\delta_{x_k}\bigr).
\]
Definimos
\[
\delta = \min_{1\le k\le n} \delta_{x_k} > 0.
\]

Si $x,y\in E$ satisfacen $d(x,y)<\delta$, entonces ambos pertenecen a
alguna de las bolas del subcubrimiento (por ejemplo, a la de centro
$x_k$ que contiene a $x$), y como en esa bola vale la condición
de continuidad con $\delta_{x_k}\ge\delta$, obtenemos
$d'(f(x),f(y))<\varepsilon$. Por lo tanto,
\[
\forall \varepsilon>0\ \exists \delta>0\ \forall x,y\in E:
d(x,y)<\delta \Rightarrow d'(f(x),f(y))<\varepsilon,
\]
y $f$ es uniformemente continua.
\end{proof}

\begin{cor}[Funciones continuas sobre compacto son cerradas]
Sean $(E,d)$ y $(E',d')$ espacios métricos, $E$ compacto y
$f : E \to E'$ continua. Entonces $f$ es una aplicación cerrada:
si $G \subseteq E$ es cerrado, entonces $f(G)$ es cerrado en $E'$.
\end{cor}

\begin{proof}
Si $G$ es cerrado en $E$, entonces $G$ es compacto (cerrado en un
compacto). Por el teorema de la imagen de un compacto, $f(G)$ es
compacto en $E'$. En un espacio métrico, todo compacto es cerrado,
así que $f(G)$ es cerrado.
\end{proof}

\begin{cor}[Homeomorfismos desde compactos]
Sean $(E,d)$ y $(E',d')$ espacios métricos, con $E$ compacto, y sea
$f : E \to E'$ continua y biyectiva. Entonces $f$ es un homeomorfismo,
es decir, $f^{-1} : E' \to E$ es continua.
\end{cor}

\begin{proof}
Del corolario anterior, $f$ es una aplicación cerrada.  
Sea $F' \subseteq E'$ cerrado. Como $f$ es biyectiva, se tiene
\[
f^{-1}(F') \subseteq E.
\]
Además
\[
E' \setminus F' \text{ es abierto} \Rightarrow
f^{-1}(E'\setminus F') = E \setminus f^{-1}(F')
\text{ es abierto en }E,
\]
por continuidad de $f$. Por lo tanto $f^{-1}(F')$ es cerrado en $E$.
La preimagen por $f^{-1}$ de todo cerrado de $E$ es cerrada, lo que
equivale a la continuidad de $f^{-1}$. Luego $f$ es un homeomorfismo.
\end{proof}
