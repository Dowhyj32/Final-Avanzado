\documentclass[11pt,a4paper]{article}

\usepackage[spanish]{babel}
\usepackage[utf8]{inputenc}
\usepackage[T1]{fontenc}
\usepackage{amsmath, amssymb, amsthm}
\usepackage{enumitem}
\usepackage{mathrsfs}

% Atajos
\newcommand{\R}{\mathbb{R}}
\newcommand{\N}{\mathbb{N}}
\newcommand{\Q}{\mathbb{Q}}
\newcommand{\eps}{\varepsilon}

% Entornos de teorema
\theoremstyle{plain}
\newtheorem{teo}{Teorema}[section]
\newtheorem{prop}[teo]{Proposición}
\newtheorem{lema}[teo]{Lema}
\newtheorem{cor}[teo]{Corolario}
\theoremstyle{definition}
\newtheorem{defi}[teo]{Definición}
\newtheorem{ej}[teo]{Ejemplo}
\theoremstyle{remark}
\newtheorem*{obs}{Observación}

\begin{document}

\title{Demostraciones Análisis Avanzado}
\author{Lucas Dowhyj}
\date{}
\maketitle

\tableofcontents

\section{Unidad 1: Infimo, supremo y sucesiones}
\subsection{Axioma de completitud}
Dado $A \subset\mathbb{R}$ no vacío y acotado superiormente, existe $\sup A$.\\
Dado $A \subset\mathbb{R}$ no vacío y acotado inferiormente, existe $\inf A$.

\subsection{Infimo}
Sea $A \subset\mathbb{R}$, no vacío y acotado inferiormente.
\[
i = \inf A \Longleftrightarrow
\begin{cases}
i \le a & \text{para todo } a \in A,\\[4pt]
\forall \varepsilon > 0 \ \exists a \in A \text{ tal que } i \le a < i + \varepsilon.
\end{cases}
\]
\begin{proof}
Supongamos que $i = \inf A$.

(i) Como $i$ es el ínfimo de $A$, por definición $i$ es cota inferior de $A$.
Es decir, para todo $a \in A$ se cumple $i \le a$.

(ii) Sea $\varepsilon > 0$. Supongamos, buscando una contradicción, que no existe
$a \in A$ tal que $i \le a < i + \varepsilon$. Entonces, para todo $a \in A$
se cumple $a \ge i + \varepsilon$, de modo que $i + \varepsilon$ es una cota
inferior de $A$. Como además $\varepsilon>0$, tenemos $i + \varepsilon > i$,
lo que contradice que $i$ es la mayor de las cotas inferiores de $A$.
Por lo tanto, debe existir $a \in A$ tal que $i \le a < i + \varepsilon$.

Recíprocamente, supongamos que se verifican:
(i) $i \le a$ para todo $a \in A$;
(ii) para todo $\varepsilon>0$ existe $a \in A$ tal que $i \le a < i + \varepsilon$.

De (i) se sigue que $i$ es cota inferior de $A$.
Sea $i'$ otra cota inferior de $A$. Queremos ver que $i' \le i$.
Supongamos, buscando una contradicción, que $i' > i$.
Sea $\varepsilon = i' - i > 0$. Por (ii) existe $a \in A$ tal que
$i \le a < i + \varepsilon$. Como $i + \varepsilon = i'$, obtenemos
$a < i'$, lo cual contradice que $i'$ es cota inferior de $A$.
Por lo tanto $i' \le i$ y, en consecuencia, $i = \inf A$.
\end{proof}


\subsection{Supremo}
Sea $A \subset\mathbb{R}$, no vacío y acotado superiormente.
\[
s = \sup A \Longleftrightarrow
\begin{cases}
s \ge a & \text{para todo } a \in A,\\[4pt]
\forall \varepsilon > 0 \ \exists a \in A \text{ tal que } s - \varepsilon < a \le s.
\end{cases}
\]
\begin{proof}
Supongamos que $s = \sup A$.

(i) Como $s$ es el supremo de $A$, por definición $s$ es cota superior de $A$.
Es decir, para todo $a \in A$ se cumple $a \le s$.

(ii) Sea $\varepsilon>0$. Supongamos, buscando una contradicción, que no existe
$a \in A$ tal que $s-\varepsilon < a \le s$. Entonces, para todo $a \in A$
se cumple $a \le s-\varepsilon$, y por lo tanto $s-\varepsilon$ es una cota
superior de $A$. Como además $\varepsilon>0$, tenemos $s-\varepsilon < s$,
lo que contradice que $s$ es la menor de las cotas superiores de $A$.
Por lo tanto, debe existir $a \in A$ tal que $s-\varepsilon < a \le s$.

Supongamos que se cumplen:
\\
(i) $a \le s$ para todo $a \in A$\\
(ii) para todo $\varepsilon>0$ existe $a \in A$ tal que $s-\varepsilon < a \le s$.

Entonces (i) dice que $s$ es cota superior de $A$.
Sea ahora $s'$ otra cota superior de $A$. Queremos ver que $s \le s'$.
Supongamos, buscando una contradicción, que $s' < s$.
Sea $\varepsilon = s - s' > 0$. Por (ii) existe $a \in A$ tal que
$s-\varepsilon < a \le s$. Como $s-\varepsilon = s'$, obtenemos $s' < a$,
lo cual contradice que $s'$ es cota superior de $A$.
Por lo tanto $s \le s'$ y, en consecuencia, $s = \sup A$.
\end{proof}

\subsection{Principio de Arquímedes}

\subsubsection*{Versión 1}
\[
\forall x \in \R \ \exists n \in \N \text{ tal que } x \le n.
\]

\begin{proof}[Demostración]
Supongamos por el absurdo que el conjunto de los naturales $\N$ está acotado superiormente.
Como $\N$ es no vacío, por el axioma de completitud existe $s = \sup \N$.

Tomamos $\varepsilon = 1$. Por la propiedad caracterizadora del supremo, existe
$n \in \N$ tal que
\[
s - 1 < n \le s.
\]
De $s - 1 < n$ se sigue que $n + 1 > s$. Como $n \in \N$, también $n+1 \in \N$,
y por lo tanto hemos encontrado un número natural estrictamente mayor que $s$,
lo que contradice que $s$ sea cota superior de $\N$.

Esta contradicción muestra que $\N$ no está acotado superiormente, es decir,
para todo $M \in \R$ existe $n \in \N$ tal que $n > M$. En particular, dado
$x \in \R$, tomando $M = x$ obtenemos un $n \in \N$ con $n \ge x$, que es
justamente lo que afirma la versión 1.
\end{proof}


\subsubsection*{Versión 2}
\[
\forall y > 0 \ \exists n \in \N \text{ tal que } 0 < \frac{1}{n} < y.
\]

\begin{proof}[Demostración]
Sea $y > 0$. Por la Versión 1 del principio de Arquímedes aplicada a $x = 1/y$,
existe $n \in \N$ tal que
\[
n > \frac{1}{y}.
\]
Como $n > 0$, al invertir la desigualdad obtenemos
\[
0 < \frac{1}{n} < y.
\]
Por lo tanto, para todo $y > 0$ existe $n \in \N$ tal que $0 < \dfrac{1}{n} < y$,
como queríamos demostrar.
\end{proof}

\subsection{Sucesiones}

\begin{prop}
Sea $(a_n)_{n\in\N}$ una sucesión real y sea $l \in \R$ tal que
\[
\lim_{n\to\infty} a_n = l.
\]
Entonces la sucesión $(a_n)$ está acotada.
\end{prop}

\begin{proof}
Por hipótesis, $\lim_{n\to\infty} a_n = l$. Esto significa que
\[
\forall \varepsilon > 0 \ \exists N \in \N \ \text{tal que} \ \forall n \ge N:
\ |a_n - l| < \varepsilon.
\]

Tomamos ahora $\varepsilon = 1$. Entonces existe $N \in \N$ tal que para todo
$n \ge N$ se cumple
\[
|a_n - l| < 1.
\]
Por la desigualdad triangular,
\[
|a_n| = |(a_n - l) + l| \le |a_n - l| + |l| < 1 + |l|
\]
para todo $n \ge N$. Es decir, para todos los índices grandes,
\[
|a_n| \le |l| + 1.
\]

Consideremos ahora los primeros términos de la sucesión:
\[
a_1, a_2, \dots, a_{N-1}.
\]
Se trata de un conjunto finito de números reales, por lo que el conjunto
\[
\{|a_1|, |a_2|, \dots, |a_{N-1}|\}
\]
tiene un máximo. Sea
\[
M_0 = \max\{|a_1|, |a_2|, \dots, |a_{N-1}|\}
\]
(en el caso $N = 1$ podemos tomar, por ejemplo, $M_0 = 0$).

Definimos ahora
\[
M = \max\{M_0,\ |l| + 1\}.
\]

Entonces:
\\- Si $n < N$, se cumple $|a_n| \le M_0 \le M$.
\\- Si $n \ge N$, se cumple $|a_n| \le |l| + 1 \le M$.
\\
En ambos casos obtenemos
\[
|a_n| \le M \quad \text{para todo } n \in \N.
\]
Por lo tanto, la sucesión $(a_n)$ está acotada.
\end{proof}

\begin{prop}
Sea $(a_n)_{n\in\N}$ una sucesión real monótona creciente y acotada superiormente.
Sea
\[
s = \sup\{a_n : n \in \N\}.
\]
Entonces
\[
\lim_{n\to\infty} a_n = s.
\]
\end{prop}

\begin{proof}
Como $(a_n)$ está acotada superiormente, el conjunto
\[
A = \{a_n : n \in \N\}
\]
tiene supremo $s \in \R$.

Queremos probar que $\lim_{n\to\infty} a_n = s$, es decir,
\[
\forall \varepsilon > 0 \ \exists N \in \N \ \forall n \ge N:
\ |a_n - s| < \varepsilon.
\]

Sea entonces $\varepsilon > 0$ arbitrario. Por la propiedad
caracterizadora del supremo aplicada al conjunto $A$, existe
$N \in \N$ tal que
\[
s - \varepsilon < a_N \le s.
\]

Como la sucesión $(a_n)$ es monótona creciente, se cumple
\[
a_N \le a_n \le s \quad \text{para todo } n \ge N.
\]

De $a_n \le s$ obtenemos $a_n - s \le 0$, luego
\[
|a_n - s| = s - a_n.
\]
Además, de $a_N \le a_n$ se sigue
\[
s - a_n \le s - a_N.
\]
Juntando estas desigualdades,
\[
|a_n - s|
= s - a_n
\le s - a_N
< s - (s - \varepsilon)
= \varepsilon
\]
para todo $n \ge N$.

Por lo tanto, para todo $\varepsilon>0$ encontramos $N \in \N$ tal que
para todo $n \ge N$ se cumple $|a_n - s|<\varepsilon$, y esto significa
exactamente que $\lim_{n\to\infty} a_n = s$.
\end{proof}

\begin{prop}
Sea $(a_n)_{n\in\N}$ una sucesión real monótona decreciente y acotada inferiormente.
Sea
\[
i = \inf\{a_n : n \in \N\}.
\]
Entonces
\[
\lim_{n\to\infty} a_n = i.
\]
\end{prop}

\begin{proof}
Como $(a_n)$ está acotada inferiormente, el conjunto
\[
A = \{a_n : n \in \N\}
\]
tiene ínfimo $i \in \R$.

Queremos probar que $\lim_{n\to\infty} a_n = i$, es decir,
\[
\forall \varepsilon > 0 \ \exists N \in \N \ \forall n \ge N:
\ |a_n - i| < \varepsilon.
\]

Sea $\varepsilon>0$ arbitrario. Por la propiedad caracterizadora del
ínfimo aplicada al conjunto $A$, existe $N \in \N$ tal que
\[
i \le a_N < i + \varepsilon.
\]

Como la sucesión $(a_n)$ es monótona decreciente, se cumple
\[
i \le a_n \le a_N \quad \text{para todo } n \ge N.
\]

De $a_n \ge i$ se obtiene $a_n - i \ge 0$, luego
\[
|a_n - i| = a_n - i.
\]
Además, de $a_n \le a_N$ se sigue
\[
a_n - i \le a_N - i.
\]
Por lo tanto,
\[
|a_n - i|
= a_n - i
\le a_N - i
< (i + \varepsilon) - i
= \varepsilon
\]
para todo $n \ge N$.

Entonces, para todo $\varepsilon>0$ encontramos $N \in \N$ tal que
para todo $n \ge N$ se cumple $|a_n - i|<\varepsilon$, lo cual prueba
que $\lim_{n\to\infty} a_n = i$.
\end{proof}

\begin{prop}[Equivalencia del supremo]
Sea $A \subset \R$ un conjunto no vacío y acotado superiormente, y sea $s \in \R$.
Entonces
\[
s = \sup A
\]
si y sólo si se cumplen:
\begin{enumerate}[label=(\roman*)]
    \item $s$ es cota superior de $A$, es decir, $a \le s$ para todo $a \in A$;
    \item existe una sucesión $(a_n)_{n\in\N}$ con $a_n \in A$ para todo $n$, tal que
    \[
    \lim_{n\to\infty} a_n = s.
    \]
\end{enumerate}
\end{prop}

\begin{proof}
Supongamos primero que $s = \sup A$. Entonces, por definición de supremo,
$s$ es cota superior de $A$, con lo cual se cumple (i).

Nos queda probar (ii). Usamos la caracterización del supremo que ya vimos:
para todo $\varepsilon > 0$ existe $a \in A$ tal que
\[
s - \varepsilon < a \le s.
\]
Para cada $n \in \N$, aplicamos esta propiedad con $\varepsilon = \tfrac{1}{n}$.
Obtenemos así un elemento $a_n \in A$ tal que
\[
s - \frac{1}{n} < a_n \le s.
\]
Esto define una sucesión $(a_n)_{n\in\N}$ de elementos de $A$.

Veamos ahora que $\lim_{n\to\infty} a_n = s$. Sea $\varepsilon > 0$. Elegimos
$N \in \N$ tal que $\frac{1}{N} < \varepsilon$. Entonces, si $n \ge N$,
se tiene $\frac{1}{n} \le \frac{1}{N} < \varepsilon$, y por la construcción
de $(a_n)$ se cumple
\[
s - \frac{1}{n} < a_n \le s.
\]
De aquí, $s - \varepsilon = s - \frac{1}{n} < a_n \le s$, luego
\[
0 \le s - a_n < \varepsilon,
\]
lo que implica
\[
|a_n - s| = s - a_n < \varepsilon.
\]
Por lo tanto,
\[
\forall \varepsilon > 0\ \exists N \in \N\ \forall n \ge N:\ |a_n - s| < \varepsilon,
\]
es decir, $\lim_{n\to\infty} a_n = s$. Esto prueba (ii).

\medskip

Recíprocamente, supongamos que se cumplen (i) y (ii). Entonces $s$ es cota
superior de $A$. Para ver que $s = \sup A$, basta probar que $s$ es la menor
de las cotas superiores de $A$. Sea $M$ otra cota superior de $A$. Queremos
ver que $s \le M$.

Supongamos, buscando una contradicción, que $s > M$. Definimos
\[
\varepsilon = \frac{s - M}{2} > 0.
\]
Como $\lim_{n\to\infty} a_n = s$, existe $N \in \N$ tal que, para todo
$n \ge N$, se cumple
\[
|a_n - s| < \varepsilon.
\]
En particular, para $n \ge N$ tenemos
\[
a_n > s - \varepsilon = s - \frac{s - M}{2}
= \frac{2s - s + M}{2}
= \frac{s + M}{2}.
\]
Pero, como $s > M$, se tiene
\[
\frac{s + M}{2} > M,
\]
de modo que
\[
a_n > \frac{s + M}{2} > M
\]
para todo $n \ge N$. Sin embargo, $a_n \in A$ y $M$ es cota superior de $A$,
luego debería cumplirse $a_n \le M$ para todo $n$, lo que contradice la
desigualdad anterior.

Esta contradicción muestra que no puede ocurrir $s > M$, por lo que necesariamente
$s \le M$. Como $M$ era una cota superior cualquiera de $A$, concluimos que
$s$ es la menor de las cotas superiores de $A$, es decir, $s = \sup A$.

Queda así demostrada la equivalencia.
\end{proof}

\begin{prop}
Sea $(a_n)_{n\in\N}$ una sucesión real y sea $(a_{n_j})_{j\in\N}$ una subsucesión de $(a_n)$,
donde $(n_j)_{j\in\N}$ es una sucesión estrictamente creciente de números naturales.
Si
\[
\lim_{n\to\infty} a_n = l,
\]
entonces
\[
\lim_{j\to\infty} a_{n_j} = l.
\]
\end{prop}

\begin{proof}
Supongamos que $\lim_{n\to\infty} a_n = l$. Por definición de límite, esto significa que
\[
\forall \varepsilon > 0 \ \exists N \in \N \ \text{tal que} \ \forall n \ge N:
\ |a_n - l| < \varepsilon.
\]

Sea ahora $\varepsilon > 0$ fijo. Por la hipótesis anterior, existe $N \in \N$ tal que
\[
n \ge N \ \Rightarrow\ |a_n - l| < \varepsilon.
\]

Como $(n_j)_{j\in\N}$ es una sucesión estrictamente creciente de números naturales,
podemos elegir $J \in \N$ tal que
\[
n_J \ge N.
\]
Entonces, para todo $j \ge J$ se tiene
\[
n_j \ge n_J \ge N.
\]
Aplicando la propiedad de arriba a $n = n_j$, obtenemos
\[
|a_{n_j} - l| < \varepsilon \quad \text{para todo } j \ge J.
\]

Por lo tanto,
\[
\forall \varepsilon > 0 \ \exists J \in \N \ \forall j \ge J:\ |a_{n_j} - l| < \varepsilon,
\]
lo cual significa exactamente que
\[
\lim_{j\to\infty} a_{n_j} = l.
\]
\end{proof}
\section{Unidad 2: Cardinalidad}
\subsection{Conjuntos coordinables y cardinal}

\begin{defi}
Sean $X$ e $Y$ dos conjuntos. Decimos que son \emph{coordinables}
(o \emph{equipotentes}, o que tienen el mismo cardinal) si existe
una función biyectiva $f : X \to Y$. En este caso escribimos
\[
X \sim Y.
\]
\end{defi}

\begin{prop}
La relación $\sim$ es una relación de equivalencia en la clase de
todos los conjuntos.
\end{prop}

\begin{proof}
Debemos probar que la relación $\sim$ es reflexiva, simétrica
y transitiva.

\medskip

\noindent\textbf{Reflexividad.}
Sea $X$ un conjunto cualquiera. Consideramos la función identidad
\[
\operatorname{id}_X : X \to X, \quad \operatorname{id}_X(x) = x.
\]
La función identidad es inyectiva (si $\operatorname{id}_X(x)
= \operatorname{id}_X(y)$, entonces $x = y$) y sobreyectiva
(para todo $x \in X$ existe $x \in X$ tal que
$\operatorname{id}_X(x) = x$). Luego es biyectiva, y por la
definición de $\sim$ se tiene $X \sim X$. Por lo tanto, $\sim$
es reflexiva.

\medskip

\noindent\textbf{Simetría.}
Sean $X$ e $Y$ conjuntos tales que $X \sim Y$. Por definición,
existe una biyección $f : X \to Y$. Toda función biyectiva tiene
inversa $f^{-1} : Y \to X$, y dicha inversa es también biyectiva.
Por lo tanto, existe una biyección de $Y$ en $X$, es decir,
$Y \sim X$. Luego, $\sim$ es simétrica.

\medskip

\noindent\textbf{Transitividad.}
Sean $X, Y, Z$ conjuntos tales que $X \sim Y$ y $Y \sim Z$.
Entonces existen biyecciones
\[
f : X \to Y,
\qquad
g : Y \to Z.
\]
Consideramos la composición
\[
g \circ f : X \to Z, \quad (g \circ f)(x) = g(f(x)).
\]
Como composición de funciones biyectivas, $g \circ f$ es también
biyectiva: la composición de funciones inyectivas es inyectiva y
la composición de funciones sobreyectivas es sobreyectiva. En
consecuencia, existe una biyección de $X$ en $Z$, es decir,
$X \sim Z$. Esto muestra que $\sim$ es transitiva.

\medskip

Como $\sim$ es reflexiva, simétrica y transitiva, concluimos que
$\sim$ es una relación de equivalencia.
\end{proof}

\begin{defi}
Definimos el \emph{cardinal} de un conjunto $X$ como la clase de
equivalencia de los conjuntos coordinables con $X$:
\[
\#X = \operatorname{Card}(X)
:= \{ Y \mid X \sim Y \}.
\]
A algunos cardinales les damos nombres especiales:
\begin{itemize}
    \item $\#\N = \aleph_0$ (cardinal numerable),
    \item $\#\R = \mathfrak{c}$ (el \emph{continuo}),
    \item $\#\{1,2,\dots,n\} = n$ para $n \in \N$.
\end{itemize}
\end{defi}

\begin{defi}
Para $n \in \N$, llamamos
\[
I_n = \{1,2,\dots,n\}
\]
al \emph{intervalo inicial} del conjunto $\N$ de los números naturales.
\end{defi}

\begin{defi}
Un conjunto $A$ es \emph{finito} si existe $n \in \N$ tal que
\[
A \sim I_n.
\]
\end{defi}

\begin{defi}
Un conjunto $A$ es \emph{infinito} si no es finito.
\end{defi}

\begin{defi}
Un conjunto $A$ es \emph{numerable} si $A \sim \N$.
Equivalente y simbólicamente, si
\[
\#A = \aleph_0.
\]
\end{defi}
\begin{defi}
Decimos que un conjunto $A$ es \emph{a lo sumo numerable}
(o \emph{contable}) si es finito o numerable. Es decir,
$A$ es a lo sumo numerable si cumple
\[
A \sim I_n \quad \text{para algún } n \in \N
\quad \text{o bien} \quad
A \sim \N.
\]
\end{defi}

\begin{prop}
Sea $A$ un conjunto numerable y sea $B \subseteq A$.
Entonces $B$ es a lo sumo numerable.
\end{prop}

\begin{proof}
Si $B$ es finito, por definición ya es a lo sumo numerable y no hay nada que
probar. Supongamos entonces que $B$ es infinito. Veremos que en ese caso
$B$ es numerable.

Como $A$ es numerable, por definición existe una biyección
\[
f : \N \longrightarrow A.
\]
Consideremos la sucesión $(f(1), f(2), f(3),\dots)$ de elementos de $A$ y
vamos a “extraer” de ella una enumeración de los elementos de $B$.

Definimos, por inducción, una sucesión estrictamente creciente
$(n_k)_{k\in\N}$ de números naturales de la siguiente manera.

En primer lugar, como $B$ es infinito, en particular es no vacío y existe
algún $b_1 \in B$. Como $f$ es sobreyectiva, existe $n_1 \in \N$ tal que
$f(n_1) = b_1$. Además, podemos elegir $n_1$ como el mínimo de los
naturales $n$ que satisfacen $f(n) \in B$:
\[
n_1 = \min\{\,n \in \N : f(n) \in B\,\}.
\]
Este mínimo existe porque el conjunto entre llaves es no vacío y está
contenido en $\N$.

Supuesto definido $n_k$ para algún $k \in \N$, definimos $n_{k+1}$ así.
Como $B$ es infinito, el conjunto
\[
B_k := B \setminus \{f(n_1), f(n_2), \dots, f(n_k)\}
\]
no es vacío (si fuera vacío, $B$ tendría a lo sumo $k$ elementos y sería
finito). Entonces existe $b_{k+1} \in B_k$. Nuevamente, como $f$ es
sobreyectiva, existe $m \in \N$ tal que $f(m) = b_{k+1}$. Además, podemos
elegir $m$ de manera que $m > n_k$ (basta tomar algún índice de $b_{k+1}$
mayor que todos $n_1,\dots,n_k$). Definimos
\[
n_{k+1} = \min\{\,n \in \N : n > n_k,\ f(n) \in B_k\,\}.
\]
Este mínimo existe porque el conjunto entre llaves es no vacío y contenido
en los naturales mayores que $n_k$. De la definición se deduce que
\[
n_1 < n_2 < n_3 < \dots,
\]
es decir, $(n_k)$ es estrictamente creciente.

Definimos ahora una función
\[
g : \N \longrightarrow B, \qquad g(k) = f(n_k).
\]

Veamos que $g$ es biyectiva.

\medskip

\noindent\textbf{Inyectividad.}
Sean $k,\ell \in \N$ tales que $g(k) = g(\ell)$, es decir,
$f(n_k) = f(n_\ell)$. Como $f$ es inyectiva, se sigue que $n_k = n_\ell$.
Pero la sucesión $(n_k)$ es estrictamente creciente, luego de
$n_k = n_\ell$ se deduce $k = \ell$. Por lo tanto, $g$ es inyectiva.

\medskip

\noindent\textbf{Sobreyectividad.}
Sea $b \in B$. Como $f$ es sobreyectiva, existe $n \in \N$ tal que
$f(n) = b$. En el proceso de construcción de la sucesión $(n_k)$, en algún
paso $k$ el elemento $b$ aparece por primera vez entre los valores de $f$;
es decir, existe un único $k$ tal que $n_k$ es el mínimo índice con
$f(n_k) = b$ y $n_k > n_{k-1}$ (para $k=1$ entendemos que no hay condición
anterior). Por la definición de $g$, se tiene entonces
\[
g(k) = f(n_k) = b.
\]
De este modo, para todo $b \in B$ existe $k \in \N$ tal que $g(k) = b$, y
por lo tanto $g$ es sobreyectiva.

\medskip

Hemos construido una biyección $g : \N \to B$, lo cual muestra que $B$ es
numerable. Recordando que al principio separamos el caso en que $B$ es
finito, concluimos que, en todos los casos, $B$ es a lo sumo numerable.
\end{proof}

\begin{teo}
Sea $A$ un conjunto infinito. Entonces existe un subconjunto
$B \subseteq A$ tal que $B$ es numerable.
\end{teo}

\begin{proof}
Como $A$ es infinito, en particular es no vacío, de modo que podemos elegir
un elemento $a_1 \in A$.

Supondremos construidos elementos distintos $a_1,\dots,a_n \in A$ para algún
$n \in \N$. Consideremos el conjunto
\[
F_n = \{a_1,\dots,a_n\}.
\]
Si $A \setminus F_n$ fuera vacío, tendríamos $A = F_n$, es decir, $A$
sería finito, lo cual contradice la hipótesis de que $A$ es infinito.
Por lo tanto,
\[
A \setminus F_n \neq \varnothing,
\]
y podemos elegir un elemento
\[
a_{n+1} \in A \setminus F_n.
\]
En particular, $a_{n+1} \in A$ y $a_{n+1} \notin \{a_1,\dots,a_n\}$, por
lo que los elementos $a_1,\dots,a_{n+1}$ siguen siendo todos distintos.

De este modo, por inducción, obtenemos una sucesión $(a_n)_{n\in\N}$ de
elementos de $A$ tales que
\[
a_n \neq a_m \quad \text{si } n \neq m.
\]

Definimos ahora
\[
B = \{a_n : n \in \N\}.
\]
Claramente $B \subseteq A$, por construcción.

Consideremos la función
\[
f : \N \longrightarrow B, \qquad f(n) = a_n.
\]

\emph{Inyectividad.}  
Si $f(n) = f(m)$, entonces $a_n = a_m$, y como la sucesión $(a_n)$ tiene
todos sus términos distintos, se sigue que $n = m$. Por lo tanto, $f$ es
inyectiva.

\emph{Sobreyectividad.}  
Sea $b \in B$. Por definición de $B$, existe $n \in \N$ tal que $b = a_n$.
Entonces $f(n) = a_n = b$, de modo que todo elemento de $B$ es imagen de
algún $n \in \N$. Por lo tanto, $f$ es sobreyectiva.

Concluimos que $f : \N \to B$ es una biyección, es decir, $B$ es numerable.
Como además $B \subseteq A$, hemos encontrado un subconjunto numerable de
$A$, tal como queríamos.
\end{proof}

\subsection{Orden entre cardinales}

Recordemos que, por definición,
\[
\#A = \#B \quad \Longleftrightarrow \quad A \sim B
\]
es decir, si y sólo si existe una biyección \(f : A \to B\).

\begin{defi}
Sean $X$ e $Y$ conjuntos. Decimos que
\[
\#X \le \#Y
\]
si existe una función inyectiva \(f : X \to Y\).
\end{defi}

\begin{defi}
Decimos que
\[
\#X < \#Y
\]
si se cumplen:
\begin{enumerate}[label=(\roman*)]
    \item \(\#X \le \#Y\), es decir, existe una inyección \(f : X \to Y\);
    \item \(X \not\sim Y\), es decir, no existe biyección entre \(X\) e \(Y\).
\end{enumerate}
\end{defi}

\begin{prop}
Sean $X$ e $Y$ conjuntos. Existe una función inyectiva
$f : X \to Y$ si y sólo si existe una función sobreyectiva
$g : Y \to X$.
\end{prop}

\begin{proof}
\textbf{($\Rightarrow$)} Supongamos que existe una función inyectiva
$f : X \to Y$. Distinguimos dos casos.

Si $X = \varnothing$, entonces $f$ es la única función posible
$\varnothing \to Y$. En este caso, la única función de $Y$ a $X$
es la función vacía $g : Y \to \varnothing$, que es sobreyectiva
sólo si $Y = \varnothing$. En muchas aplicaciones se descarta el
caso trivial $X = \varnothing$, así que supongamos ahora que
$X \ne \varnothing$.

Como $X \ne \varnothing$, elegimos un elemento fijo $x_0 \in X$.
Definimos $g : Y \to X$ de la siguiente manera:
\[
g(y) =
\begin{cases}
x & \text{si existe } x \in X \text{ tal que } f(x) = y,\\
x_0 & \text{si no existe tal } x.
\end{cases}
\]
La inyectividad de $f$ garantiza que, cuando $y$ está en la imagen
de $f$, el elemento $x$ tal que $f(x) = y$ es único, de modo que
$g$ está bien definida.

Veamos que $g$ es sobreyectiva. Sea $x \in X$. Como $f$ es función
de $X$ en $Y$, tenemos $f(x) \in Y$, y por definición de $g$,
\[
g(f(x)) = x.
\]
Luego todo $x \in X$ es imagen de algún elemento de $Y$ (por ejemplo,
de $f(x)$), y por lo tanto $g$ es sobreyectiva.

\medskip

\textbf{($\Leftarrow$)} Recíprocamente, supongamos que existe una función
sobreyectiva $g : Y \to X$. Para cada $x \in X$ consideremos el conjunto
de sus preimágenes:
\[
Y_x = \{\, y \in Y : g(y) = x \,\}.
\]
Como $g$ es sobreyectiva, $Y_x$ es no vacío para todo $x \in X$.

Elegimos, para cada $x \in X$, un elemento $y_x \in Y_x$ (es decir,
$g(y_x) = x$), y definimos
\[
f : X \to Y, \qquad f(x) = y_x.
\]

Probemos que $f$ es inyectiva. Sean $x_1,x_2 \in X$ tales que
$f(x_1) = f(x_2)$. Entonces
\[
y_{x_1} = y_{x_2}.
\]
Aplicando $g$ a ambos lados obtenemos
\[
g(y_{x_1}) = g(y_{x_2}),
\]
es decir,
\[
x_1 = x_2,
\]
ya que por definición de $y_x$ se cumple $g(y_x) = x$. Por lo tanto,
$f$ es inyectiva.

\medskip

Hemos probado en un sentido que de una inyectiva $X \to Y$ obtenemos
una sobreyectiva $Y \to X$, y en el otro que de una sobreyectiva
$Y \to X$ obtenemos una inyectiva $X \to Y$. Esto completa la
demostración.
\end{proof}

\subsection{Conjunto de partes y teorema de Cantor}

\begin{defi}
Dado un conjunto $X$, llamamos \emph{conjunto de partes} de $X$ al conjunto
\[
\mathcal{P}(X) = \{ A : A \subseteq X\}.
\]
\end{defi}

\begin{teo}[Cantor]
Sea $X$ un conjunto. Entonces
\[
\#X < \#\mathcal{P}(X).
\]
\end{teo}

\begin{proof}
Recordemos que, por la definición de orden entre cardinales,
\[
\#X < \#\mathcal{P}(X)
\quad\Longleftrightarrow\quad
\begin{cases}
\text{existe una inyección } f : X \to \mathcal{P}(X),\\
\text{no existe biyección entre } X \text{ y } \mathcal{P}(X).
\end{cases}
\]

\textbf{(1) Existe una inyección $X \to \mathcal{P}(X)$.}

Definimos
\[
f : X \longrightarrow \mathcal{P}(X), \qquad f(x) = \{x\}.
\]
Claramente $f(x) \subseteq X$ para todo $x$, luego $f(x) \in \mathcal{P}(X)$.
Si $f(x) = f(y)$, entonces $\{x\} = \{y\}$ y por lo tanto $x = y$.
Así, $f$ es inyectiva, y obtenemos
\[
\#X \le \#\mathcal{P}(X).
\]

\textbf{(2) No existe biyección entre $X$ y $\mathcal{P}(X)$.}

Basta ver que \emph{no existe ninguna función sobreyectiva}
$g : X \to \mathcal{P}(X)$.

Procedemos por absurdo. Supongamos que existe una función
\[
g : X \longrightarrow \mathcal{P}(X)
\]
sobreyectiva. Consideremos el subconjunto
\[
B = \{ x \in X : x \notin g(x) \}.
\]
Por definición, $B \subseteq X$, de modo que $B \in \mathcal{P}(X)$.

Como $g$ es sobreyectiva, debe existir algún elemento $a \in X$ tal que
\[
g(a) = B.
\]

Estudiemos ahora si $a$ pertenece o no a $B$:

- Supongamos que $a \in B$.  
  Por la definición de $B$, esto significa que $a \notin g(a)$. Pero
  $g(a) = B$, luego $a \notin B$, lo que contradice $a \in B$.

- Supongamos que $a \notin B$.  
  Entonces, por la definición de $B$, se tiene $a \in g(a)$. Como
  $g(a) = B$, esto implica $a \in B$, contradiciendo $a \notin B$.

En ambos casos llegamos a una contradicción. Por lo tanto, nuestra
suposición inicial es falsa: no existe función sobreyectiva
$g : X \to \mathcal{P}(X)$.

Concluimos que no existe biyección entre $X$ y $\mathcal{P}(X)$.
Junto con (1), esto implica
\[
\#X < \#\mathcal{P}(X),
\]
como queríamos demostrar.
\end{proof}

\subsection{Suma y resta de conjuntos numerables}

\begin{prop}
Sea $X$ un conjunto infinito. Entonces existe un subconjunto
$Z \subset X$, con $Z$ numerable, tal que
\[
X \sim X \setminus Z.
\]
\end{prop}

\begin{proof}
Como $X$ es infinito, por el teorema anterior existe un subconjunto
numerable infinito $C \subset X$. Como $C$ es numerable, existe una
biyección
\[
\varphi : \N \to C.
\]
Escribimos $c_n = \varphi(n)$ para todo $n \in \N$, de modo que
\[
C = \{c_1, c_2, c_3,\dots\}.
\]

Definimos ahora dos subconjuntos disjuntos de $C$:
\[
Z = \{c_{2n} : n \in \N\}, \qquad
D = \{c_{2n-1} : n \in \N\}.
\]
Entonces $C = D \cup Z$ y $D \cap Z = \varnothing$. Además, tanto $D$
como $Z$ son numerables (son imágenes de $\N$ por las aplicaciones
$n \mapsto c_{2n-1}$ y $n \mapsto c_{2n}$, respectivamente).

Sea
\[
Y = X \setminus C.
\]
Entonces tenemos una partición
\[
X = Y \cup D \cup Z
\quad\text{(unión disjunta)}.
\]
Por otra parte,
\[
X \setminus Z = Y \cup D.
\]

Definimos una aplicación $f : X \to X \setminus Z$ por:
\[
f(x) =
\begin{cases}
x, & \text{si } x \in Y \cup D,\\
c_{2n-1}, & \text{si } x = c_{2n} \in Z \text{ para algún } n \in \N.
\end{cases}
\]

Veamos que $f$ es biyectiva.

\medskip

\noindent\textbf{Inyectividad.}
- Si $x_1,x_2 \in Y \cup D$ y $f(x_1) = f(x_2)$, entonces $x_1 = x_2$
  porque $f$ actúa como la identidad en $Y \cup D$.

- Si $x_1 = c_{2n_1}$ e $x_2 = c_{2n_2}$ pertenecen a $Z$ y
  $f(x_1) = f(x_2)$, entonces
  \[
  c_{2n_1-1} = f(c_{2n_1}) = f(c_{2n_2}) = c_{2n_2-1},
  \]
  de donde $2n_1-1 = 2n_2-1$ y luego $n_1 = n_2$, es decir
  $x_1 = x_2$.

- No puede ocurrir que $x_1 \in Y \cup D$ y $x_2 \in Z$ tengan la
  misma imagen, porque las imágenes de $Y \cup D$ están en $Y \cup D$
  y las de $Z$ están contenidas en $D$; pero $Y$ y $D$ son disjuntos.

En todos los casos, de $f(x_1) = f(x_2)$ se deduce $x_1 = x_2$, luego
$f$ es inyectiva.

\medskip

\noindent\textbf{Sobreyectividad.}
Sea $y \in X \setminus Z = Y \cup D$.

- Si $y \in Y$, entonces $f(y) = y$, así que $y$ es imagen de sí mismo.

- Si $y \in D$, digamos $y = c_{2n-1}$ para algún $n \in \N$, entonces
  $f(c_{2n}) = c_{2n-1} = y$, de modo que $y$ es imagen de $c_{2n} \in Z$.

En consecuencia, todo elemento de $X \setminus Z$ es imagen de algún
elemento de $X$, y $f$ es sobreyectiva.

\medskip

Hemos construido una biyección $f : X \to X \setminus Z$ con
$Z \subset X$ numerable, por lo que $X \sim X \setminus Z$.
\end{proof}

\begin{prop}
Sea $B$ un conjunto y sea $A$ un conjunto numerable.
Suponemos que $B \setminus A$ es infinito. Entonces
\[
B \sim B \setminus A.
\]
\end{prop}

\begin{proof}
Podemos reemplazar $A$ por $A \cap B$, que sigue siendo numerable y
cumple
\[
B \setminus (A \cap B) = B \setminus A.
\]
Por simplicidad, suponemos desde ahora que $A \subseteq B$.

Como $A$ es numerable, existe una biyección
\[
(a_n)_{n\in\N} : \N \to A,
\]
es decir, podemos escribir
\[
A = \{a_1, a_2, a_3,\dots\}.
\]

Por hipótesis, $B \setminus A$ es infinito. Entonces, por el teorema
“conjunto infinito contiene un subconjunto numerable”, existe un
subconjunto numerable infinito
\[
C \subseteq B \setminus A.
\]
Tomamos una biyección
\[
(c_n)_{n\in\N} : \N \to C, \quad C = \{c_1, c_2, c_3,\dots\}.
\]

Definimos ahora una aplicación $f : B \to B \setminus A$ por:
\[
f(x) =
\begin{cases}
c_n, & \text{si } x = a_n \in A \text{ para algún } n \in \N,\\[4pt]
x,   & \text{si } x \in B \setminus A.
\end{cases}
\]

Observemos primero que $f$ está bien definida: si $x \in A$, entonces
$f(x) = c_n \in C \subseteq B \setminus A$; si $x \in B \setminus A$,
entonces $f(x) = x \in B \setminus A$. En cualquier caso,
$f(x) \in B \setminus A$.

\medskip

\noindent\textbf{Inyectividad.}
- Si $x_1, x_2 \in B \setminus A$ y $f(x_1) = f(x_2)$, entonces
  $x_1 = x_2$ porque $f$ actúa como la identidad en $B \setminus A$.

- Si $x_1 = a_n$ y $x_2 = a_m$ pertenecen a $A$ y
  $f(x_1) = f(x_2)$, entonces $c_n = c_m$, y como la sucesión
  $(c_n)$ tiene todos sus términos distintos, se sigue $n = m$ y
  por lo tanto $x_1 = x_2$.

- No puede ocurrir que $x_1 \in A$ y $x_2 \in B \setminus A$ tengan
  la misma imagen, porque $f(x_1) \in C \subseteq B \setminus A$,
  mientras que $f(x_2) = x_2 \in B \setminus A \setminus C$, y
  $C$ es disjunto de $B \setminus A \setminus C$.

En consecuencia, $f$ es inyectiva.

\medskip

\noindent\textbf{Sobreyectividad.}
Sea $y \in B \setminus A$. Distinguimos dos casos:

- Si $y \in B \setminus (A \cup C)$, entonces $f(y) = y$, de modo que
  $y$ es imagen de sí mismo.

- Si $y \in C$, digamos $y = c_n$ para algún $n \in \N$, entonces
  $f(a_n) = c_n = y$, de modo que $y$ es imagen de $a_n \in A$.

Por lo tanto, todo elemento de $B \setminus A$ es imagen de algún
elemento de $B$, y $f$ es sobreyectiva.

\medskip

Hemos construido una biyección $f : B \to B \setminus A$, lo cual prueba
que $B \sim B \setminus A$.
\end{proof}

\begin{prop}
Sea $X$ un conjunto infinito y sea $A$ un conjunto numerable.
Entonces
\[
X \sim X \cup A.
\]
\end{prop}

\begin{proof}
Si $A \subseteq X$, entonces $X \cup A = X$ y la afirmación es trivial.
Supongamos, por lo tanto, que $A$ no está contenido en $X$. Definimos
\[
A_0 = A \setminus X,
\]
que es el conjunto de los elementos de $A$ que no pertenecen a $X$.
Como $A$ es numerable, también $A_0$ es numerable (subconjunto de un
numerable). Además,
\[
X \cup A = X \cup A_0
\]
y la unión es disjunta, ya que $A_0 \cap X = \varnothing$.

Notemos que $X$ es infinito, luego el conjunto $X \cup A_0$ también es
infinito. Consideremos ahora el conjunto
\[
B = X \cup A_0.
\]
Entonces $A_0$ es numerable y
\[
B \setminus A_0 = X.
\]
Como $X$ es infinito, también $B \setminus A_0$ es infinito. Podemos
aplicar la proposición anterior con $A = A_0$ y este conjunto $B$:
obtenemos
\[
B \sim B \setminus A_0.
\]
Pero $B = X \cup A_0$ y $B \setminus A_0 = X$, por lo que
\[
X \cup A_0 \sim X.
\]

Como $X \cup A = X \cup A_0$, concluimos que
\[
X \cup A \sim X.
\]
Por simetría de la relación de equipotencia, también escribimos
$X \sim X \cup A$, como queríamos.
\end{proof}

\begin{cor}
Un conjunto $X$ es infinito si y sólo si es coordinable con un subconjunto
propio suyo, es decir, si y sólo si existe $Y \subsetneq X$ tal que
$X \sim Y$.
\end{cor}

\begin{proof}
($\Rightarrow$) Supongamos que $X$ es infinito.
Por la proposición anterior, existe un subconjunto numerable
$Z \subset X$ tal que $X \sim X \setminus Z$.
Como $Z \ne \varnothing$ (pues es numerable) y $Z \subset X$, se tiene
$X \setminus Z \subsetneq X$.
Por lo tanto, $X$ es coordinable con el subconjunto propio
$X \setminus Z$.

($\Leftarrow$) Recíprocamente, supongamos que existe un subconjunto propio
$Y \subsetneq X$ tal que $X \sim Y$.
Procedamos por absurdo: supongamos que $X$ es finito.
Sea $n = \#X$. Entonces $X \sim I_n$.
Como $Y$ es subconjunto propio de $X$, tiene un número $m$ de elementos
con $m < n$, de modo que $Y \sim I_m$.

Por transitividad de la relación de equipotencia, tendríamos
\[
I_n \sim X \sim Y \sim I_m,
\]
de donde se seguiría $I_n \sim I_m$.
Pero por el teorema anterior, $I_n \sim I_m$ implica $n = m$,
lo cual contradice $m < n$.
Esta contradicción muestra que $X$ no puede ser finito, luego
$X$ es infinito.
\end{proof}

\begin{teo}[Cantor–Schröder–Bernstein]
Sean $X$ e $Y$ dos conjuntos. Si existe una función inyectiva
$f : X \to Y$ y una función inyectiva $g : Y \to X$, entonces
$X$ e $Y$ son coordinables, es decir, existe una biyección
$h : X \to Y$.
\end{teo}

Equivalentemente, en términos de cardinales:
\[
\#X \le \#Y \ \text{y} \ \#Y \le \#X \quad \Longrightarrow \quad \#X = \#Y.
\]

\begin{lema}
El producto cartesiano $\N \times \N$ es numerable.
\end{lema}

\begin{proof}
Consideremos la aplicación $f : \N \times \N \to \N$ dada por
\[
f(m,n) = 2^m 3^n.
\]
Por el teorema fundamental de la aritmética, todo número natural
tiene una única factorización en primos, de modo que distintos pares
$(m,n)$ producen distintos números $2^m 3^n$. Por lo tanto, $f$ es
inyectiva.

Como hemos construido una inyección de $\N \times \N$ en $\N$, se sigue
que $\N \times \N$ es a lo sumo numerable. Además, es infinito, por lo
que es numerable.
\end{proof}

\begin{teo}
Sea $(A_n)_{n\in\N}$ una familia de conjuntos numerables.
Entonces la unión
\[
A = \bigcup_{n\in\N} A_n
\]
es a lo sumo numerable. En particular, si $A$ es infinita, entonces
$A$ es numerable.
\end{teo}

\begin{proof}
Como cada $A_n$ es numerable, para todo $n \in \N$ existe una
biyección
\[
f_n : \N \to A_n.
\]
Escribimos $a_{n,k} = f_n(k)$, de modo que
\[
A_n = \{a_{n,1}, a_{n,2}, a_{n,3}, \dots\}.
\]

Definimos ahora una aplicación
\[
F : \N \times \N \to A, \qquad F(n,k) = a_{n,k}.
\]
Para todo $(n,k) \in \N\times\N$ se tiene $a_{n,k} \in A_n \subseteq A$,
luego $F$ está bien definida. Además, por la definición de $A$,
todo elemento de $A$ es de la forma $a_{n,k}$ para algún $n,k \in \N$,
y por lo tanto $F$ es sobreyectiva.

Por el lema anterior, $\N \times \N$ es numerable. Entonces existe una
biyectiva
\[
g : \N \to \N \times \N.
\]
Consideremos la composición
\[
h = F \circ g : \N \to A.
\]
Como composición de una biyección con una sobreyección, $h$ sigue
siendo sobreyectiva: para todo $a \in A$ existe $(n,k) \in \N \times \N$
tal que $F(n,k) = a$, y como $g$ es biyectiva, existe $m \in \N$ con
$g(m) = (n,k)$; entonces
\[
h(m) = (F \circ g)(m) = F(n,k) = a.
\]

Así hemos construido una función sobreyectiva $h : \N \to A$.
Por la proposición que relaciona inyecciones y sobreyecciones entre
cardinales, esto implica que $A$ es a lo sumo numerable.

Si además $A$ es infinito, por definición de “a lo sumo numerable”
resulta que $A$ es numerable.
\end{proof}

\section{Unidad 3: Espacios métricos}
\subsection{Conjuntos abiertos}

\begin{defi}
Sea $(E,d)$ un espacio métrico. Un subconjunto $U \subseteq E$ se dice
\emph{abierto} si
\[
\forall x \in U\ \exists r > 0 \ \text{tal que} \ B(x,r) \subseteq U,
\]
donde
\[
B(x,r) = \{y \in E : d(x,y) < r\}
\]
es la bola abierta de centro $x$ y radio $r$.
\end{defi}

\begin{prop}
Sea $(E,d)$ un espacio métrico, $x_0 \in E$ y $r>0$.
Entonces la bola abierta $B(x_0,r)$ es un conjunto abierto.
\end{prop}

\begin{proof}
Debemos ver que para todo punto $x \in B(x_0,r)$ existe un radio
$\varepsilon > 0$ tal que
\[
B(x,\varepsilon) \subseteq B(x_0,r).
\]

Sea $x \in B(x_0,r)$. Por definición de bola abierta,
\[
d(x,x_0) < r.
\]
Definimos
\[
\varepsilon = r - d(x,x_0).
\]
Entonces $\varepsilon > 0$ porque $d(x,x_0) < r$.

Tomemos ahora un punto cualquiera $y \in B(x,\varepsilon)$; es decir,
\[
d(x,y) < \varepsilon.
\]
Aplicando la desigualdad triangular, obtenemos
\[
d(y,x_0) \le d(y,x) + d(x,x_0).
\]
Sustituyendo las cotas anteriores,
\[
d(y,x_0) < \varepsilon + d(x,x_0)
= \bigl(r - d(x,x_0)\bigr) + d(x,x_0) = r.
\]
Por lo tanto, $d(y,x_0) < r$, lo que significa que $y \in B(x_0,r)$.

Como $y$ fue elegido arbitrariamente en $B(x,\varepsilon)$, hemos probado
\[
B(x,\varepsilon) \subseteq B(x_0,r).
\]
Y esto vale para todo $x \in B(x_0,r)$, con el $\varepsilon$ definido
como arriba. Por definición de conjunto abierto, $B(x_0,r)$ es abierto.
\end{proof}

\begin{defi}
Sea $(E,d)$ un espacio métrico y sea $A \subseteq E$.
El \emph{interior} de $A$ es el conjunto
\[
A^\circ = \{\,x \in A : \exists r>0 \text{ tal que } B(x,r) \subseteq A\,\}.
\]
Equivalente: $A^\circ$ es el conjunto de los puntos interiores de $A$.
\end{defi}

\begin{prop}
Sea $(E,d)$ un espacio métrico y $A,A_1,A_2 \subseteq E$.
Se tienen las siguientes propiedades:
\begin{enumerate}[label=(\roman*)]
    \item $A^\circ \subseteq A$.
    \item Si $A_1 \subseteq A_2$, entonces $A_1^\circ \subseteq A_2^\circ$.
    \item $A^\circ$ es un conjunto abierto.
    \item Si $G$ es abierto y $G \subseteq A$, entonces $G \subseteq A^\circ$.
\end{enumerate}
\end{prop}

\begin{proof}
(i) Por definición,
\[
A^\circ = \{x \in A : \exists r>0 \text{ tal que } B(x,r) \subseteq A\}.
\]
En particular, todo $x \in A^\circ$ pertenece a $A$, luego
$A^\circ \subseteq A$.

\medskip

(ii) Supongamos que $A_1 \subseteq A_2$ y sea $x \in A_1^\circ$.
Entonces, por definición de interior, existe $r>0$ tal que
\[
B(x,r) \subseteq A_1.
\]
Como $A_1 \subseteq A_2$, se tiene también
\[
B(x,r) \subseteq A_2,
\]
de modo que $x$ es punto interior de $A_2$, es decir, $x \in A_2^\circ$.
Hemos probado que $A_1^\circ \subseteq A_2^\circ$.

\medskip

(iii) Queremos ver que $A^\circ$ es abierto. Sea $x \in A^\circ$.
Entonces existe $r>0$ tal que
\[
B(x,r) \subseteq A.
\]
Por la proposición ya demostrada, $B(x,r)$ es un conjunto abierto.
En particular, como $x \in B(x,r)$ y $B(x,r)$ es abierto, existe
$\varepsilon>0$ tal que
\[
B(x,\varepsilon) \subseteq B(x,r).
\]
Como además $B(x,r) \subseteq A$, obtenemos
\[
B(x,\varepsilon) \subseteq A.
\]
Por definición de interior, esto implica que $x \in A^\circ$ y, de hecho,
para todo $y \in B(x,\varepsilon)$ se cumple $y \in A^\circ$.
Por lo tanto, $B(x,\varepsilon) \subseteq A^\circ$.

Hemos visto que para todo $x \in A^\circ$ existe $\varepsilon>0$ tal que
$B(x,\varepsilon) \subseteq A^\circ$, lo cual significa que $A^\circ$
es abierto.

\medskip

(iv) Sea $G$ un conjunto abierto tal que $G \subseteq A$, y sea
$x \in G$. Como $G$ es abierto, existe $r>0$ tal que
\[
B(x,r) \subseteq G.
\]
De la inclusión $G \subseteq A$ se deduce
\[
B(x,r) \subseteq A.
\]
Luego $x$ es punto interior de $A$, es decir, $x \in A^\circ$.
Como $x$ era un punto arbitrario de $G$, concluimos que $G \subseteq A^\circ$.
\end{proof}

\begin{prop}
Sea $(E,d)$ un espacio métrico. Entonces se verifican:
\begin{enumerate}[label=(\roman*)]
    \item La unión arbitraria de conjuntos abiertos es un conjunto abierto.
    \item La intersección finita de conjuntos abiertos es un conjunto abierto.
\end{enumerate}
\end{prop}

\begin{proof}
(i) Sea $(G_i)_{i\in I}$ una familia cualquiera de conjuntos abiertos en $E$
(indexada por un conjunto $I$ no necesariamente numerable) y definamos
\[
G = \bigcup_{i\in I} G_i.
\]
Queremos ver que $G$ es abierto.

Sea $x \in G$. Entonces existe algún índice $i_0 \in I$ tal que
$x \in G_{i_0}$. Como $G_{i_0}$ es abierto, existe $r>0$ tal que
\[
B(x,r) \subseteq G_{i_0}.
\]
De aquí se deduce
\[
B(x,r) \subseteq G_{i_0} \subseteq G.
\]
Por lo tanto, para todo $x \in G$ hemos encontrado un radio $r>0$ con
$B(x,r) \subseteq G$. Por definición, $G$ es abierto.

\medskip

(ii) Sea $G_1,\dots,G_n$ una familia finita de conjuntos abiertos en $E$
y definamos
\[
H = \bigcap_{k=1}^n G_k.
\]
Probemos que $H$ es abierto.

Tomemos $x \in H$. Entonces $x \in G_k$ para todo $k=1,\dots,n$.
Como cada $G_k$ es abierto, existe $r_k > 0$ tal que
\[
B(x,r_k) \subseteq G_k
\quad \text{para cada } k=1,\dots,n.
\]
Definimos
\[
r = \min\{r_1,\dots,r_n\}.
\]
Entonces $r > 0$ y, para todo $k$,
\[
B(x,r) \subseteq B(x,r_k) \subseteq G_k.
\]
De aquí se sigue
\[
B(x,r) \subseteq \bigcap_{k=1}^n G_k = H.
\]
Por lo tanto, para cada $x \in H$ existe $r>0$ tal que
$B(x,r) \subseteq H$, lo que muestra que $H$ es abierto.
\end{proof}

\begin{obs}
El resultado sobre intersección de abiertos es, en general, válido sólo
para intersecciones finitas. Una intersección infinita de abiertos puede
no ser abierta.

Por ejemplo, en $(\R,d_{\text{eucl}})$, para cada $n \in \N$ el conjunto
\[
G_n = \left(-\frac{1}{n}, \frac{1}{n}\right)
\]
es un abierto. Consideremos la intersección
\[
\bigcap_{n\in\N} G_n
= \bigcap_{n\in\N} \left(-\frac{1}{n}, \frac{1}{n}\right).
\]
Es fácil ver que
\[
\bigcap_{n\in\N} \left(-\frac{1}{n}, \frac{1}{n}\right) = \{0\}.
\]
El conjunto $\{0\}$ no es abierto en $\R$ con la métrica usual, ya que si
tomamos cualquier $r>0$, la bola $B(0,r) = (-r,r)$ contiene puntos distintos
de $0$, de modo que nunca se cumple $B(0,r) \subseteq \{0\}$.
Por lo tanto, la intersección numerable de los abiertos $G_n$ no es abierta.
\end{obs}

\subsection{Conjuntos cerrados}

\begin{defi}
Sea $(E,d)$ un espacio métrico y $A \subseteq E$.
Un punto $x \in E$ se llama \emph{punto de adherencia} (o
\emph{punto de clausura}) de $A$ si
\[
\forall r > 0 \quad B(x,r) \cap A \ne \varnothing.
\]
\end{defi}

\begin{defi}
Sea $(E,d)$ un espacio métrico y $A \subseteq E$.
La \emph{clausura} (o \emph{adherencia}) de $A$ es el conjunto
\[
\overline{A}
= \{x \in E : x \text{ es punto de adherencia de } A\}.
\]
\end{defi}

\begin{defi}
Sea $(E,d)$ un espacio métrico. Un subconjunto $F \subseteq E$ se dice
\emph{cerrado} si su complemento $E \setminus F$ es un conjunto abierto.
\end{defi}

\begin{prop}
Sea $(E,d)$ un espacio métrico, $x_0 \in E$ y $r>0$.
Entonces la bola cerrada
\[
\overline{B}(x_0,r) = \{x \in E : d(x,x_0) \le r\}
\]
es un conjunto cerrado.
\end{prop}

\begin{proof}
Por definición, $\overline{B}(x_0,r)$ es cerrada si su complemento
$E \setminus \overline{B}(x_0,r)$ es abierto. Sea
\[
x \in E \setminus \overline{B}(x_0,r).
\]
Entonces $d(x,x_0) > r$. Definimos
\[
\varepsilon = \frac{d(x,x_0) - r}{2}.
\]
Como $d(x,x_0) > r$, se tiene $\varepsilon > 0$.

Mostremos que
\[
B(x,\varepsilon) \subseteq E \setminus \overline{B}(x_0,r).
\]
Sea $y \in B(x,\varepsilon)$, es decir,
\[
d(x,y) < \varepsilon.
\]
Por la desigualdad triangular,
\[
d(y,x_0) \ge d(x,x_0) - d(x,y).
\]
Luego
\[
d(y,x_0) > d(x,x_0) - \varepsilon
= d(x,x_0) - \frac{d(x,x_0) - r}{2}
= \frac{d(x,x_0) + r}{2}.
\]
Como $d(x,x_0) > r$, se tiene
\[
\frac{d(x,x_0) + r}{2} > \frac{r + r}{2} = r,
\]
de modo que $d(y,x_0) > r$. En particular, $d(y,x_0) \not\le r$, así que
$y \notin \overline{B}(x_0,r)$.

Hemos probado que todo $y \in B(x,\varepsilon)$ pertenece al complemento
de la bola cerrada:
\[
B(x,\varepsilon) \subseteq E \setminus \overline{B}(x_0,r).
\]
Por lo tanto, para cada $x$ del complemento existe un $\varepsilon>0$
tal que la bola $B(x,\varepsilon)$ está contenida en dicho complemento.
Esto significa que $E \setminus \overline{B}(x_0,r)$ es abierto.

Concluimos que $\overline{B}(x_0,r)$ es cerrado.
\end{proof}


\begin{prop}[Caracterización de la clausura mediante cerrados]
Sea $(E,d)$ un espacio métrico y $A \subseteq E$.
Entonces:
\begin{enumerate}[label=(\roman*)]
    \item $A \subseteq \overline{A}$.
    \item $\overline{A}$ es un conjunto cerrado.
    \item Si $F$ es un conjunto cerrado con $A \subseteq F$, entonces
    $\overline{A} \subseteq F$.
\end{enumerate}
En particular, $\overline{A}$ es el menor conjunto cerrado que contiene a $A$,
y se puede escribir
\[
\overline{A} = \bigcap\{F \subseteq E : F \text{ es cerrado y } A \subseteq F\}.
\]
\end{prop}

\begin{proof}
(i) Sea $x \in A$. Entonces, para cualquier $r>0$, se tiene
$x \in B(x,r)$, y por lo tanto $B(x,r) \cap A \ne \varnothing$.
Es decir, $x es$ punto de adherencia de $A$, y por definición
$x \in \overline{A}$. Luego $A \subseteq \overline{A}$.

\medskip

(ii) Probemos que $\overline{A}$ es cerrado mostrando que
$E \setminus \overline{A}$ es abierto.

Sea $x \in E \setminus \overline{A}$. Entonces $x$ no es punto
de adherencia de $A$, es decir, existe $r>0$ tal que
\[
B(x,r) \cap A = \varnothing.
\]
Mostraremos que $B(x,r/2) \subseteq E \setminus \overline{A}$.

Sea $y \in B(x,r/2)$, de modo que $d(x,y) < r/2$.
Tomemos $s = r/2$. Consideremos un punto $z \in B(y,s)$;
entonces $d(y,z) < s = r/2$. Por la desigualdad triangular,
\[
d(x,z) \le d(x,y) + d(y,z) < \frac{r}{2} + \frac{r}{2} = r,
\]
luego $z \in B(x,r)$. Como $B(x,r) \cap A = \varnothing$,
se sigue que $z \notin A$. En consecuencia,
\[
B(y,s) \cap A = \varnothing.
\]
Esto significa que $y$ no es punto de adherencia de $A$,
es decir, $y \notin \overline{A}$.

Como $y$ fue un punto arbitrario de $B(x,r/2)$, hemos probado
que $B(x,r/2) \subseteq E \setminus \overline{A}$. Por lo tanto,
$E \setminus \overline{A}$ es abierto, y en consecuencia
$\overline{A}$ es cerrado.

\medskip

(iii) Sea $F$ un conjunto cerrado tal que $A \subseteq F$.
Debemos probar que $\overline{A} \subseteq F$.

Sea $x \in E \setminus F$. Como $F$ es cerrado, su complemento
$E \setminus F$ es abierto. Entonces existe $r>0$ tal que
\[
B(x,r) \subseteq E \setminus F.
\]
Como $A \subseteq F$, se tiene $A \cap (E \setminus F) = \varnothing$,
y en particular
\[
B(x,r) \cap A = \varnothing.
\]
Esto muestra que $x$ no es punto de adherencia de $A$, es decir,
$x \notin \overline{A}$. Hemos probado
\[
E \setminus F \subseteq E \setminus \overline{A}.
\]
Tomando complementos,
\[
\overline{A} \subseteq F.
\]
Esto vale para todo conjunto cerrado $F$ que contiene a $A$, con lo cual
queda demostrado que $\overline{A}$ es el menor cerrado que contiene a $A$,
y en particular
\[
\overline{A} = \bigcap\{F \subseteq E : F \text{ es cerrado y } A \subseteq F\}.
\]
\end{proof}

\begin{prop}
Un conjunto $F \subseteq E$ es cerrado si y sólo si
\[
F = \overline{F}.
\]
\end{prop}

\begin{proof}
($\Rightarrow$) Si $F$ es cerrado y $F$ contiene a $F$, por la proposición
anterior, $\overline{F} \subseteq F$. Por otra parte, siempre se cumple
$F \subseteq \overline{F}$. De aquí se deduce $F = \overline{F}$.

($\Leftarrow$) Si $F = \overline{F}$, como sabemos que $\overline{F}$ es
cerrado, se sigue que $F$ es cerrado.
\end{proof}

\begin{prop}[Caracterización secuencial de la clausura]
Sea $(E,d)$ un espacio métrico, $A \subseteq E$ y $x \in E$.
Entonces las siguientes afirmaciones son equivalentes:
\begin{enumerate}[label=(\roman*)]
    \item $x \in \overline{A}$.
    \item Existe una sucesión $(a_n)_{n\in\N}$ con $a_n \in A$ para todo $n$ tal que
    \[
    \lim_{n\to\infty} a_n = x.
    \]
\end{enumerate}
\end{prop}

\begin{proof}
$(i) \Rightarrow (ii)$) Supongamos que $x \in \overline{A}$. Por definición
de clausura, esto significa que para todo $r>0$ se cumple
\[
B(x,r) \cap A \ne \varnothing.
\]

Para cada $n \in \N$, consideremos el radio $r_n = \frac{1}{n} > 0$.
Como $B(x,r_n) \cap A \ne \varnothing$, podemos elegir un punto
\[
a_n \in B\Bigl(x,\frac{1}{n}\Bigr) \cap A.
\]
Así definimos una sucesión $(a_n)_{n\in\N}$ de puntos de $A$ que satisface
\[
d(a_n,x) < \frac{1}{n} \quad \text{para todo } n \in \N.
\]

Veamos que $a_n \to x$. Sea $\varepsilon > 0$. Elegimos $N \in \N$ tal que
\[
\frac{1}{N} < \varepsilon.
\]
Si $n \ge N$, entonces
\[
d(a_n,x) < \frac{1}{n} \le \frac{1}{N} < \varepsilon.
\]
Por lo tanto,
\[
\forall \varepsilon>0\ \exists N \in \N\ \forall n \ge N:\ d(a_n,x) < \varepsilon,
\]
lo cual significa que $a_n \to x$. Esto prueba (ii).

\medskip

$(ii) \Rightarrow (i)$) Recíprocamente, supongamos que existe una sucesión
$(a_n)_{n\in\N}$ con $a_n \in A$ para todo $n$ tal que $a_n \to x$.
Debemos probar que $x \in \overline{A}$, es decir, que
\[
\forall r>0 \quad B(x,r) \cap A \ne \varnothing.
\]

Sea $r>0$. Como $a_n \to x$, por definición de límite existe $N \in \N$
tal que, para todo $n \ge N$,
\[
d(a_n,x) < r.
\]
En particular, tomando $n = N$, tenemos $a_N \in B(x,r)$.
Como además $a_N \in A$ por construcción de la sucesión, se obtiene
\[
a_N \in B(x,r) \cap A,
\]
y por lo tanto $B(x,r) \cap A \ne \varnothing$.

Como esto vale para todo $r>0$, concluimos que $x$ es punto de adherencia
de $A$, es decir, $x \in \overline{A}$.
\end{proof}

\begin{prop}
Sea $(E,d)$ un espacio métrico y $F \subseteq E$.
Entonces $F$ es cerrado si y sólo si se verifica:
\[
\text{si } (x_n)_{n\in\N} \subseteq F \text{ y } x_n \to x \text{ en } E,
\text{ entonces } x \in F.
\]
\end{prop}

\begin{proof}
($\Rightarrow$) Supongamos que $F$ es cerrado y sea $(x_n)$ una sucesión
de puntos de $F$ tal que $x_n \to x$ en $E$. Supongamos, por absurdo,
que $x \notin F$. Entonces $x \in E \setminus F$.

Como $E \setminus F$ es abierto (porque $F$ es cerrado), existe $r>0$
tal que
\[
B(x,r) \subseteq E \setminus F.
\]
Por la convergencia de $(x_n)$ a $x$, sabemos que
\[
\forall \varepsilon > 0 \ \exists N \in \N \ \forall n \ge N:
\ d(x_n,x) < \varepsilon.
\]
Tomamos $\varepsilon = r$. Entonces existe $N \in \N$ tal que
si $n \ge N$, se cumple $d(x_n,x) < r$, es decir, $x_n \in B(x,r)$.
Pero $B(x,r) \subseteq E \setminus F$, luego $x_n \notin F$ para
todo $n \ge N$. Esto contradice el hecho de que $x_n \in F$ para
todos los $n$, por hipótesis. Por lo tanto, debe ser $x \in F$.

\medskip

($\Leftarrow$) Supongamos ahora que se cumple la propiedad secuencial:
toda sucesión de puntos de $F$ que converge en $E$ tiene su límite en $F$.
Queremos ver que $F$ es cerrado, es decir, que $E \setminus F$ es abierto.

Tomemos un punto $x \in E \setminus F$. Supongamos, por absurdo, que
$E \setminus F$ no es abierto. Entonces no existe ningún $r>0$ tal que
$B(x,r) \subseteq E \setminus F$. En particular, para todo $n \in \N$,
el radio $r = \tfrac{1}{n}$ no sirve, es decir:
\[
B\Bigl(x,\frac{1}{n}\Bigr) \not\subseteq E \setminus F.
\]
Esto significa que para cada $n \in \N$ existe algún punto
\[
x_n \in B\Bigl(x,\frac{1}{n}\Bigr) \cap F.
\]
Por construcción, $x_n \in F$ y, además,
\[
d(x_n,x) < \frac{1}{n} \quad \text{para todo } n \in \N.
\]
De aquí se deduce que $x_n \to x$ en el espacio métrico $(E,d)$.

Pero, por hipótesis secuencial, el límite de cualquier sucesión de puntos
de $F$ que converge en $E$ debe pertenecer a $F$, de modo que $x \in F$.
Esto contradice que $x \in E \setminus F$.

Por lo tanto, nuestra suposición era falsa: para cada $x \in E \setminus F$
debe existir $r>0$ tal que $B(x,r) \subseteq E \setminus F$, lo que muestra
que $E \setminus F$ es abierto. En consecuencia, $F$ es cerrado.
\end{proof}

\begin{prop}
Sea $(E,d)$ un espacio métrico. Entonces se verifican:
\begin{enumerate}[label=(\roman*)]
    \item $\varnothing$ y $E$ son conjuntos cerrados.
    \item La intersección arbitraria de conjuntos cerrados es un conjunto cerrado.
    \item La unión finita de conjuntos cerrados es un conjunto cerrado.
\end{enumerate}
\end{prop}

\begin{proof}
(i) Notemos que
\[
E \setminus \varnothing = E
\quad\text{y}\quad
E \setminus E = \varnothing.
\]
Como ya sabemos que $E$ y $\varnothing$ son abiertos, por definición
de conjunto cerrado se sigue que $\varnothing$ y $E$ son cerrados.

\medskip

(ii) Sea $(F_i)_{i\in I}$ una familia cualquiera de conjuntos cerrados
en $E$, indexada por un conjunto $I$ arbitrario, y consideremos
\[
F = \bigcap_{i\in I} F_i.
\]
Queremos probar que $F$ es cerrado.

Usamos la relación entre complementos, uniones e intersecciones:
\[
E \setminus F
= E \setminus \bigg(\bigcap_{i\in I} F_i\bigg)
= \bigcup_{i\in I} (E \setminus F_i).
\]
Cada $F_i$ es cerrado, así que $E \setminus F_i$ es abierto para todo $i$.
Por lo tanto, la unión $\bigcup_{i\in I} (E \setminus F_i)$ es abierta
(por la propiedad ya demostrada para uniones arbitrarias de abiertos).
En consecuencia, $E \setminus F$ es abierto, lo que implica que $F$
es cerrado.

\medskip

(iii) Sean $F_1,\dots,F_n$ conjuntos cerrados en $E$ y consideremos
\[
H = \bigcup_{k=1}^n F_k.
\]
Entonces
\[
E \setminus H
= E \setminus \bigg(\bigcup_{k=1}^n F_k\bigg)
= \bigcap_{k=1}^n (E \setminus F_k).
\]
Como cada $F_k$ es cerrado, $E \setminus F_k$ es abierto para todo $k$.
La intersección finita de abiertos es abierta, por lo tanto
$\bigcap_{k=1}^n (E \setminus F_k)$ es abierta. De aquí se deduce
que $E \setminus H$ es abierto y, por definición, $H$ es cerrado.
\end{proof}

\subsection{Puntos de acumulación y frontera}

\begin{defi}
Sea $(E,d)$ un espacio métrico y $A \subseteq E$.
Un punto $x \in E$ se llama \emph{punto de acumulación} (o
\emph{punto límite}) de $A$ si
\[
\forall r>0 \quad \bigl(B(x,r) \setminus \{x\}\bigr) \cap A \neq \varnothing.
\]
Denotamos por $A'$ al conjunto de todos los puntos de acumulación de $A$.
\end{defi}

\begin{prop}[Caracterización secuencial de puntos de acumulación]
Sea $(E,d)$ un espacio métrico, $A \subseteq E$ y $x \in E$.
Entonces las siguientes afirmaciones son equivalentes:
\begin{enumerate}[label=(\roman*)]
    \item $x$ es punto de acumulación de $A$, es decir $x \in A'$.
    \item Existe una sucesión $(a_n)_{n\in\N}$ con $a_n \in A \setminus \{x\}$
    para todo $n$ tal que
    \[
    \lim_{n\to\infty} a_n = x.
    \]
\end{enumerate}
\end{prop}

\begin{proof}
$(i) \Rightarrow (ii)$) Supongamos que $x \in A'$. Entonces, por definición,
para todo $r>0$ se cumple
\[
\bigl(B(x,r) \setminus \{x\}\bigr) \cap A \neq \varnothing.
\]
Para cada $n \in \N$, tomamos $r_n = \frac{1}{n}$ y elegimos
\[
a_n \in \bigl(B(x,r_n) \setminus \{x\}\bigr) \cap A.
\]
Entonces $a_n \in A \setminus \{x\}$ y $d(a_n,x) < \frac{1}{n}$ para todo
$n \in \N$.

Sea $\varepsilon>0$. Elegimos $N \in \N$ tal que
\(
\frac{1}{N} < \varepsilon.
\)
Si $n \ge N$, entonces
\[
d(a_n,x) < \frac{1}{n} \le \frac{1}{N} < \varepsilon.
\]
Por definición de límite, $a_n \to x$.

\medskip

$(ii) \Rightarrow (i)$) Recíprocamente, supongamos que existe una sucesión
$(a_n)$ con $a_n \in A \setminus \{x\}$ para todo $n$ y $a_n \to x$.
Sea $r>0$. Como $a_n \to x$, existe $N \in \N$ tal que, para todo $n \ge N$,
\[
d(a_n,x) < r.
\]
En particular, $a_N \in B(x,r)$, y como $a_N \neq x$ se tiene
$a_N \in B(x,r) \setminus \{x\}$. Además, $a_N \in A$.

Por lo tanto,
\[
\bigl(B(x,r) \setminus \{x\}\bigr) \cap A \neq \varnothing
\]
para todo $r>0$, lo que significa que $x$ es punto de acumulación de $A$,
es decir, $x \in A'$.
\end{proof}

\begin{teo}
Sea $A \subseteq E$ un subconjunto de un espacio métrico $(E,d)$.
Entonces
\[
\overline{A} = A \cup A'.
\]
\end{teo}

\begin{proof}
Primero probamos la inclusión $\overline{A} \subseteq A \cup A'$.
Sea $x \in \overline{A}$. Si $x \in A$, ya está.
Supongamos que $x \notin A$. Como $x \in \overline{A}$, por definición de
clausura se cumple
\[
\forall r>0 \quad B(x,r) \cap A \neq \varnothing.
\]
Pero como $x \notin A$, esto implica automáticamente que
\[
\forall r>0 \quad \bigl(B(x,r) \setminus \{x\}\bigr) \cap A \neq \varnothing,
\]
es decir, $x$ es punto de acumulación de $A$, $x \in A'$. En cualquiera de
los dos casos, $x \in A \cup A'$. Por lo tanto
\[
\overline{A} \subseteq A \cup A'.
\]

Ahora probamos la inclusión inversa $A \cup A' \subseteq \overline{A}$.
Si $x \in A$, ya vimos que siempre se cumple $A \subseteq \overline{A}$,
así que $x \in \overline{A}$.

Si $x \in A'$, entonces para todo $r>0$ se tiene
\[
\bigl(B(x,r) \setminus \{x\}\bigr) \cap A \neq \varnothing.
\]
En particular, esto implica $B(x,r) \cap A \neq \varnothing$ para todo $r>0$,
de modo que $x$ es punto de adherencia de $A$, es decir, $x \in \overline{A}$.

En ambos casos $x \in \overline{A}$, por lo que
\[
A \cup A' \subseteq \overline{A}.
\]

Concluimos que $\overline{A} = A \cup A'$.
\end{proof}

\begin{cor}
Sea $A \subseteq E$. Entonces $A$ es cerrado si y sólo si
\[
A' \subseteq A.
\]
\end{cor}

\begin{proof}
($\Rightarrow$) Si $A$ es cerrado, por definición $A = \overline{A}$.
Por el teorema anterior,
\[
\overline{A} = A \cup A'.
\]
Luego
\[
A = A \cup A'.
\]
Esto sólo puede ocurrir si $A' \subseteq A$.

\medskip

($\Leftarrow$) Recíprocamente, supongamos que $A' \subseteq A$.
Del teorema anterior tenemos
\[
\overline{A} = A \cup A'.
\]
Como $A' \subseteq A$, se obtiene
\[
\overline{A} = A.
\]
Pero un conjunto es cerrado si y sólo si coincide con su clausura, por lo que
$A$ es cerrado.
\end{proof}

\begin{defi}
Sea $(E,d)$ un espacio métrico y $A \subseteq E$.
Un punto $x \in E$ se llama \emph{punto frontera} (o \emph{punto de borde})
de $A$ si para todo $r>0$ se cumple
\[
B(x,r) \cap A \neq \varnothing
\quad\text{y}\quad
B(x,r) \cap (E \setminus A) \neq \varnothing.
\]
El conjunto de todos los puntos frontera de $A$ se denota por $\partial A$
y se llama \emph{frontera} (o \emph{borde}) de $A$.
\end{defi}

\begin{prop}
Para todo $A \subseteq E$ se verifica
\[
\partial A = \overline{A} \setminus A^\circ.
\]
\end{prop}

\begin{proof}
Sea $x \in \partial A$. Entonces, por definición de punto frontera,
para todo $r>0$,
\[
B(x,r) \cap A \neq \varnothing
\quad\text{y}\quad
B(x,r) \cap (E \setminus A) \neq \varnothing.
\]
En particular, la primera condición implica que $x$ es punto de adherencia
de $A$, es decir $x \in \overline{A}$. Por otra parte, la segunda condición
impide que exista algún $r>0$ con $B(x,r) \subseteq A$, luego $x$ no es
punto interior de $A$, es decir $x \notin A^\circ$. Por lo tanto
$x \in \overline{A} \setminus A^\circ$.

Recíprocamente, sea $x \in \overline{A} \setminus A^\circ$. Entonces
$x \in \overline{A}$, de modo que para todo $r>0$,
\[
B(x,r) \cap A \neq \varnothing.
\]
Además, $x \notin A^\circ$, es decir, no existe ningún $r>0$ tal que
$B(x,r) \subseteq A$. Por lo tanto, para todo $r>0$ se tiene
\[
B(x,r) \not\subseteq A,
\]
lo que significa que para todo $r>0$ existe $y \in B(x,r)$ tal que
$y \notin A$, es decir,
\[
B(x,r) \cap (E \setminus A) \neq \varnothing.
\]
Juntando ambas condiciones, $x$ es punto frontera de $A$, es decir
$x \in \partial A$.

Hemos probado ambas inclusiones, así que $\partial A = \overline{A} \setminus A^\circ$.
\end{proof}

\begin{prop}
Sea $A \subseteq E$. Entonces:
\begin{enumerate}[label=(\roman*)]
    \item $\partial A$ es un conjunto cerrado.
    \item Se tiene la descomposición
    \[
    E = A^\circ \,\dot\cup\, \partial A \,\dot\cup\, (E \setminus \overline{A}),
    \]
    donde las uniones son disjuntas.
    \item Se verifica
    \[
    \overline{A} = A \cup \partial A.
    \]
\end{enumerate}
\end{prop}

\begin{proof}
(i) Por la proposición anterior,
\[
\partial A = \overline{A} \setminus A^\circ
= \overline{A} \cap (E \setminus A^\circ).
\]
Sabemos que $\overline{A}$ es cerrado. Como $A^\circ$ es abierto,
su complemento $E \setminus A^\circ$ es cerrado. La intersección de
conjuntos cerrados es cerrada, luego $\partial A$ es cerrado.

\medskip

(ii) Por definición de interior, clausura y frontera, se tiene
\[
A^\circ \subseteq A \subseteq \overline{A}.
\]
Además, por la proposición anterior,
\[
\partial A = \overline{A} \setminus A^\circ.
\]
Entonces
\[
\overline{A} = A^\circ \,\dot\cup\, \partial A.
\]
Por otra parte, el complemento de $\overline{A}$ es
$E \setminus \overline{A}$, y es abierto. Así, todo punto de $E$
pertenece o bien a $\overline{A}$ o bien a $E \setminus \overline{A}$,
y dentro de $\overline{A}$ está exactamente en $A^\circ$ o en
$\partial A$. Esto da la descomposición
\[
E = A^\circ \,\dot\cup\, \partial A \,\dot\cup\, (E \setminus \overline{A}).
\]

\medskip

(iii) De la igualdad del teorema anterior
\(
\overline{A} = A \cup A'
\)
y del hecho de que $\partial A \subseteq \overline{A}$, se deduce
en particular que
\[
A \subseteq \overline{A}
\quad\text{y}\quad
\partial A \subseteq \overline{A}.
\]
Por lo tanto,
\[
A \cup \partial A \subseteq \overline{A}.
\]

Recíprocamente, sea $x \in \overline{A}$. Si $x \in A$, ya está.
Si $x \notin A$, como $x \in \overline{A}$, para todo $r>0$ se cumple
$B(x,r) \cap A \neq \varnothing$. Además, como $x \notin A$, se tiene
$x \in E \setminus A$, luego para todo $r>0$ se cumple
$x \in B(x,r) \cap (E \setminus A)$, o sea
\[
B(x,r) \cap (E \setminus A) \neq \varnothing.
\]
Por definición, esto significa que $x \in \partial A$.
En cualquier caso, $x \in A \cup \partial A$, con lo cual
\[
\overline{A} \subseteq A \cup \partial A.
\]
Concluimos que $\overline{A} = A \cup \partial A$.
\end{proof}

\subsection{Métricas equivalentes}

Sea $E$ un conjunto y $d_1,d_2$ dos métricas en $E$.

\begin{defi}
Decimos que las métricas $d_1$ y $d_2$ en $E$ son
\emph{equivalentes} si inducen los mismos conjuntos abiertos, es decir,
si para todo $U \subseteq E$ se cumple:
\[
U \text{ es abierto en } (E,d_1)
\quad \Longleftrightarrow \quad
U \text{ es abierto en } (E,d_2).
\]
\end{defi}

\begin{prop}[Caracterización secuencial]
Sean $d_1,d_2$ dos métricas en $E$. Son equivalentes si y sólo si
para toda sucesión $(x_n)_{n\in\N}$ en $E$ y todo $x \in E$ se verifica
\[
x_n \to x \text{ en } (E,d_1)
\quad \Longleftrightarrow \quad
x_n \to x \text{ en } (E,d_2).
\]
\end{prop}

\begin{proof}
($\Rightarrow$) Supongamos que $d_1$ y $d_2$ son equivalentes, es decir,
tienen los mismos conjuntos abiertos.

Recordemos que, en un espacio métrico, una sucesión $(x_n)$ converge a $x$
si y sólo si se verifica la siguiente propiedad topológica:
\[
\forall U \text{ abierto con } x \in U,\ \exists N \in \N\ \forall n \ge N:
\ x_n \in U.
\]

Sea $(x_n)$ una sucesión en $E$ y $x \in E$.
Supongamos que $x_n \to x$ en $(E,d_1)$.
Entonces, para todo abierto $U$ de $(E,d_1)$ con $x \in U$, existe
$N$ tal que, si $n \ge N$, se cumple $x_n \in U$.

Como $d_1$ y $d_2$ tienen los mismos abiertos, ese mismo conjunto $U$
es abierto también en $(E,d_2)$. La condición anterior es exactamente
la definición de convergencia de $(x_n)$ a $x$ en $(E,d_2)$. Por lo tanto,
$x_n \to x$ en $(E,d_2)$.

La implicación recíproca se demuestra igual, intercambiando el rol de
$d_1$ y $d_2$.

\medskip

($\Leftarrow$) Supongamos ahora que las sucesiones convergentes son las
mismas para $d_1$ y $d_2$.

Queremos ver que los conjuntos abiertos también coinciden.
Sea $U \subseteq E$ abierto en $(E,d_1)$, y probemos que es abierto en
$(E,d_2)$.

Para eso, basta ver que si $(x_n)$ es una sucesión en $E \setminus U$
que converge a algún $x \in E$ respecto de $d_2$, entonces
$x \in E \setminus U$ (es decir, $x \notin U$). En efecto, esta propiedad
caracteriza a los cerrados, y al tomar complementos caracteriza a los
abiertos.

Sea entonces $(x_n)$ una sucesión con $x_n \in E \setminus U$ para todo $n$
y $x_n \to x$ en $(E,d_2)$. Por hipótesis, las sucesiones convergentes son
las mismas en ambas métricas, así que también $x_n \to x$ en $(E,d_1)$.

Como $U$ es abierto en $(E,d_1)$, su complemento $E \setminus U$ es cerrado
en $(E,d_1)$. Por la caracterización secuencial de cerrados,
si una sucesión de puntos de $E \setminus U$ converge (en $d_1$),
su límite debe pertenecer a $E \setminus U$. En particular,
$x \in E \setminus U$, es decir, $x \notin U$.

Hemos probado que el complemento de $U$ es cerrado también con la métrica
$d_2$, luego $U$ es abierto en $(E,d_2)$. El mismo argumento, cambiando
el rol de $d_1$ y $d_2$, muestra la implicación inversa. Por lo tanto,
los abiertos de $d_1$ coinciden con los de $d_2$, y las métricas son
equivalentes.
\end{proof}

\begin{defi}
Decimos que dos métricas $d_1,d_2$ en $E$ son \emph{fuertemente
equivalentes} si existen constantes $c_1,c_2 > 0$ tales que
\[
c_1\, d_1(x,y) \le d_2(x,y) \le c_2\, d_1(x,y)
\quad \text{para todo } x,y \in E.
\]
\end{defi}

\begin{prop}
Si $d_1$ y $d_2$ son fuertemente equivalentes, entonces son
equivalentes (en el sentido de que inducen los mismos conjuntos abiertos).
\end{prop}

\begin{proof}
Sea $U$ un abierto de $(E,d_1)$ y tomemos $x \in U$.
Por definición de abierto, existe $r>0$ tal que
\[
B_{d_1}(x,r) \subseteq U,
\]
donde $B_{d_1}(x,r) = \{y : d_1(x,y) < r\}$.

Consideremos ahora la bola en la métrica $d_2$
\[
B_{d_2}\bigl(x, c_1 r\bigr)
= \{y \in E : d_2(x,y) < c_1 r\}.
\]
Si $y \in B_{d_2}(x,c_1 r)$, entonces $d_2(x,y) < c_1 r$, y usando
la desigualdad $c_1 d_1(x,y) \le d_2(x,y)$ obtenemos
\[
c_1 d_1(x,y) \le d_2(x,y) < c_1 r
\quad\Longrightarrow\quad
d_1(x,y) < r.
\]
Luego $y \in B_{d_1}(x,r)$, y por lo tanto
\[
B_{d_2}(x,c_1 r) \subseteq B_{d_1}(x,r) \subseteq U.
\]
Hemos mostrado que para todo $x \in U$ existe un radio $c_1 r>0$ tal que
la bola $d_2$-abierta correspondiente está contenida en $U$, de modo que
$U$ es abierto en $(E,d_2)$.

La implicación recíproca (todo abierto $d_2$-abierto es $d_1$-abierto)
se prueba exactamente igual usando la otra desigualdad
$d_2(x,y) \le c_2 d_1(x,y)$. Concluimos que $d_1$ y $d_2$ tienen los mismos
abiertos, es decir, son métricas equivalentes.
\end{proof}

\begin{obs}
En espacios de dimensión finita (por ejemplo, $\R^n$), todas las normas
son fuertemente equivalentes. En particular, las métricas inducidas
por las normas $d_1,d_2,d_\infty$ en $\R^n$ son equivalentes, y por lo tanto
tienen los mismos abiertos, los mismos cerrados, las mismas sucesiones
convergentes, etc.
\end{obs}

\begin{teo}
Sean $d$ y $d'$ dos métricas sobre un mismo conjunto $E$.
Las métricas $d$ y $d'$ son (topológicamente) equivalentes si y sólo si
para todo $x \in E$ y todo $r>0$ existen $r_1,r_2>0$ tales que
\[
B_{d'}(x,r_1) \subseteq B_d(x,r)
\quad\text{y}\quad
B_d(x,r_2) \subseteq B_{d'}(x,r),
\]
donde
\[
B_d(x,r) = \{y \in E : d(x,y) < r\},
\qquad
B_{d'}(x,r) = \{y \in E : d'(x,y) < r\}.
\]
\end{teo}

\begin{proof}
($\Rightarrow$) Supongamos que $d$ y $d'$ son métricas equivalentes, es decir,
inducen los mismos conjuntos abiertos.

Sea $x \in E$ y $r>0$ arbitrarios. Entonces $B_d(x,r)$ es un abierto en
$(E,d)$; como los abiertos de $(E,d)$ y $(E,d')$ coinciden, $B_d(x,r)$
es también abierto en $(E,d')$.

Por definición de abierto en la métrica $d'$, existe $r_1>0$ tal que
\[
B_{d'}(x,r_1) \subseteq B_d(x,r).
\]

Para obtener la otra inclusión, usamos el mismo argumento intercambiando
el rol de $d$ y $d'$: como $B_{d'}(x,r)$ es abierto en $(E,d')$ y las
familias de abiertos coinciden, $B_{d'}(x,r)$ es abierto en $(E,d)$.
Luego existe $r_2>0$ tal que
\[
B_d(x,r_2) \subseteq B_{d'}(x,r).
\]
Así se verifica la condición del enunciado.

\medskip

($\Leftarrow$) Recíprocamente, supongamos que para todo $x \in E$ y
todo $r>0$ existen $r_1,r_2>0$ tales que
\[
B_{d'}(x,r_1) \subseteq B_d(x,r)
\quad\text{y}\quad
B_d(x,r_2) \subseteq B_{d'}(x,r).
\]

Queremos probar que los abiertos de $(E,d)$ y $(E,d')$ coinciden.

Sea $U \subseteq E$ un conjunto abierto en $(E,d)$. Sea $x \in U$.
Como $U$ es abierto para $d$, existe $r>0$ tal que
\[
B_d(x,r) \subseteq U.
\]
Por hipótesis, existe $r_2>0$ tal que
\[
B_d(x,r_2) \subseteq B_{d'}(x,r).
\]
Aplicando de nuevo la hipótesis, pero ahora a la bola $d'$-abierta
$B_{d'}(x,r)$, podemos elegir $r_3>0$ con
\[
B_{d'}(x,r_3) \subseteq B_d(x,r_2).
\]
Por lo tanto,
\[
B_{d'}(x,r_3) \subseteq B_d(x,r_2) \subseteq B_d(x,r) \subseteq U.
\]
Hemos encontrado, para cada $x \in U$, un radio $r_3>0$ tal que
$B_{d'}(x,r_3) \subseteq U$, lo que significa que $U$ es abierto en
$(E,d')$.

El argumento recíproco (si $U$ es abierto en $(E,d')$ entonces lo es
en $(E,d)$) se obtiene exactamente igual, usando de nuevo la hipótesis
para pasar de bolas $d'$-abiertas a bolas $d$-abiertas. Concluimos que
las familias de conjuntos abiertos de $(E,d)$ y $(E,d')$ coinciden, es
decir, $d$ y $d'$ son métricas equivalentes.
\end{proof}

\subsection{Sucesiones de Cauchy y espacios métricos completos}

\begin{defi}
Sea $(E,d)$ un espacio métrico y $A \subseteq E$.
Decimos que $A$ es \emph{acotado} si existen $x \in E$ y $r>0$ tales que
\[
A \subset B(x,r),
\]
donde $B(x,r) = \{y \in E : d(x,y) < r\}$ es la bola abierta de centro $x$
y radio $r$.
\end{defi}

\begin{defi}
Sea $(E,d)$ un espacio métrico y $(x_n)_{n\in\N} \subseteq E$ una sucesión.
Decimos que $(x_n)$ es \emph{acotada} si existen $x \in E$ y $r>0$ tales que
\[
x_n \in B(x,r) \quad \text{para todo } n \in \N.
\]
Equivalente: el conjunto $\{x_n : n \in \N\}$ es acotado en $E$.
\end{defi}

\begin{defi}
Sea $(E,d)$ un espacio métrico y $(x_n)_{n\in\N} \subseteq E$ una sucesión.
Decimos que $(x_n)$ es una \emph{sucesión de Cauchy} si
\[
\forall \varepsilon>0\ \exists n_0 \in \N\ \text{tal que}\ \forall n,m \ge n_0:
\ d(x_n,x_m) < \varepsilon.
\]
\end{defi}

\begin{teo}
Sea $(E,d)$ un espacio métrico y $(x_n)_{n\in\N} \subseteq E$. Entonces:
\begin{enumerate}[label=(\arabic*)]
    \item Si $(x_n)$ es de Cauchy, entonces es acotada.
    \item Si $(x_n)$ es convergente en $E$, entonces es de Cauchy.
    \item Si $(x_n)$ es de Cauchy y tiene alguna subsucesión convergente,
    entonces $(x_n)$ es convergente (en $E$) y converge al mismo límite
    que esa subsucesión.
\end{enumerate}
\end{teo}

\begin{proof}
(1) Supongamos que $(x_n)$ es de Cauchy. Tomamos $\varepsilon = 1$ en la
definición. Entonces existe $n_0 \in \N$ tal que, si $n,m \ge n_0$, se cumple
\[
d(x_n,x_m) < 1.
\]
En particular, para todo $n \ge n_0$,
\[
d(x_n,x_{n_0}) < 1.
\]

Sea ahora
\[
M = \max\bigl\{d(x_1,x_{n_0}), d(x_2,x_{n_0}), \dots, d(x_{n_0-1},x_{n_0}), 1\bigr\}
\]
(si $n_0=1$, simplemente tomamos $M=1$). Entonces $M>0$ y, para todo
$n \in \N$,
\[
d(x_n,x_{n_0}) \le M.
\]
Por lo tanto,
\[
x_n \in B(x_{n_0},M) \quad \text{para todo } n \in \N,
\]
lo que muestra que la sucesión $(x_n)$ es acotada.

\medskip

(2) Supongamos que $(x_n)$ converge en $E$ a algún $x \in E$; es decir,
\[
\forall \varepsilon>0\ \exists n_0 \in \N\ \forall n \ge n_0:
\ d(x_n,x) < \varepsilon/2.
\]
(Escribimos $\varepsilon/2$ para simplificar las cuentas que siguen.)

Sea ahora $\varepsilon>0$ fijo. Tomamos $n_0$ como arriba.
Si $n,m \ge n_0$, por la desigualdad triangular:
\[
d(x_n,x_m) \le d(x_n,x) + d(x,x_m) < \frac{\varepsilon}{2} + \frac{\varepsilon}{2}
= \varepsilon.
\]
Hemos mostrado que para todo $\varepsilon>0$ existe $n_0$ tal que,
si $n,m \ge n_0$, se cumple $d(x_n,x_m) < \varepsilon$; es decir,
$(x_n)$ es de Cauchy.

\medskip

(3) Supongamos que $(x_n)$ es de Cauchy y que existe una subsucesión
$(x_{n_k})_{k\in\N}$ tal que $x_{n_k} \to x \in E$.

Debemos probar que $x_n \to x$.

Sea $\varepsilon>0$. Como $(x_{n_k})$ converge a $x$, existe $K \in \N$ tal que,
si $k \ge K$,
\[
d(x_{n_k},x) < \varepsilon/2.
\]
Como $(x_n)$ es de Cauchy, existe $n_0 \in \N$ tal que, si $n,m \ge n_0$,
\[
d(x_n,x_m) < \varepsilon/2.
\]

Definimos
\[
N = \max\{n_0, n_K\}.
\]
Sea ahora $n \ge N$. Entonces $n \ge n_0$ y $n_K \ge n_0$, por lo que
aplicando la propiedad de Cauchy con $m = n_K$ obtenemos:
\[
d(x_n,x_{n_K}) < \varepsilon/2.
\]
Por otra parte, como $n_K \ge K$, tenemos
\[
d(x_{n_K},x) < \varepsilon/2.
\]

Usando la desigualdad triangular,
\[
d(x_n,x) \le d(x_n,x_{n_K}) + d(x_{n_K},x)
< \frac{\varepsilon}{2} + \frac{\varepsilon}{2} = \varepsilon.
\]

Hemos probado que
\[
\forall \varepsilon>0\ \exists N \in \N\ \forall n \ge N:\ d(x_n,x) < \varepsilon,
\]
lo cual significa que $x_n \to x$ en $(E,d)$.
\end{proof}

\begin{defi}
Un espacio métrico $(E,d)$ se dice \emph{completo} si toda sucesión de
Cauchy en $E$ es convergente a un punto de $E$.
\end{defi}

% (Opcional, pero muy usado en ejercicios)
\begin{prop}
Sea $(E,d)$ un espacio métrico completo y sea $F \subseteq E$ un subconjunto
cerrado. Entonces el subespacio métrico $(F,d)$ es completo.
\end{prop}

\begin{proof}
Sea $(x_n)_{n\in\N}$ una sucesión de Cauchy en $F$ (con la métrica $d$
restringida). Como $F \subseteq E$, también es una sucesión de Cauchy en $E$.
Dado que $(E,d)$ es completo, existe $x \in E$ tal que $x_n \to x$ en $E$.

Como $F$ es cerrado en $E$ y todos los $x_n$ están en $F$, por la
caracterización secuencial de cerrados se tiene necesariamente $x \in F$.
Por lo tanto, $(x_n)$ converge a un punto de $F$, y así $(F,d)$ es completo.
\end{proof}

\section{Unidad 4: Funciones continuas}
\subsection{Definición y caracterizaciones de continuidad}

Sea $(E,d_E)$ y $(F,d_F)$ espacios métricos, y sea $A \subseteq E$.
Consideramos funciones $f : A \to F$.

\begin{defi}[Continuidad en un punto]
Sea $x_0 \in A$. Decimos que $f$ es \emph{continua en $x_0$} si
\[
\forall \varepsilon > 0\ \exists \delta > 0\ \text{tal que}\ 
\forall x \in A,\ d_E(x,x_0) < \delta \ \Rightarrow\ d_F\bigl(f(x),f(x_0)\bigr) < \varepsilon.
\]
\end{defi}

\begin{defi}[Continuidad en un conjunto]
Decimos que $f$ es \emph{continua en $A$} si es continua en todo punto
$x_0 \in A$. En particular, cuando $A = E$, diremos simplemente que
$f : E \to F$ es \emph{continua}.
\end{defi}

\begin{defi}[Continuidad secuencial en un punto]
Sea $x_0 \in A$. Decimos que $f$ es \emph{secuencialmente continua en $x_0$}
si para toda sucesión $(x_n)_{n\in\N} \subseteq A$ tal que
\[
x_n \to x_0 \text{ en } (E,d_E),
\]
se cumple
\[
f(x_n) \to f(x_0) \text{ en } (F,d_F).
\]
\end{defi}

\begin{prop}[Equivalencia $\varepsilon$--$\delta$ / sucesiones]
Sea $f : A \to F$ y $x_0 \in A$. Entonces las siguientes afirmaciones
son equivalentes:
\begin{enumerate}[label=(\roman*)]
    \item $f$ es continua en $x_0$ (en el sentido $\varepsilon$--$\delta$).
    \item $f$ es secuencialmente continua en $x_0$, es decir:
    para toda sucesión $(x_n) \subseteq A$ con $x_n \to x_0$ en $(E,d_E)$,
    se tiene $f(x_n) \to f(x_0)$ en $(F,d_F)$.
\end{enumerate}
\end{prop}

\begin{proof}
$(i) \Rightarrow (ii)$) Supongamos que $f$ es continua en $x_0$ en el sentido
$\varepsilon$--$\delta$. Sea $(x_n)_{n\in\N} \subseteq A$ una sucesión tal que
$x_n \to x_0$ en $(E,d_E)$.

Sea $\varepsilon>0$. Por continuidad en $x_0$, existe $\delta>0$ tal que
\[
d_E(x,x_0) < \delta \ \Rightarrow\ d_F\bigl(f(x),f(x_0)\bigr) < \varepsilon
\quad \text{para todo } x \in A.
\]
Como $x_n \to x_0$, existe $N \in \N$ tal que, si $n \ge N$, se cumple
\[
d_E(x_n,x_0) < \delta.
\]
Aplicando la condición de continuidad a cada $x_n$ con $n \ge N$, obtenemos
\[
d_F\bigl(f(x_n),f(x_0)\bigr) < \varepsilon
\quad \text{para todo } n \ge N.
\]
Por definición de límite de sucesión en $(F,d_F)$, esto significa que
$f(x_n) \to f(x_0)$. Luego (ii) es verdadera.

\medskip

$(ii) \Rightarrow (i)$) Supongamos ahora que $f$ es secuencialmente continua
en $x_0$ y probemos que es continua en $x_0$ en el sentido $\varepsilon$--$\delta$.

Razonamos por absurdo. Supongamos que $f$ \emph{no} es continua en $x_0$.
Entonces existe algún $\varepsilon_0 > 0$ tal que, para todo $\delta > 0$,
existe $x \in A$ con
\[
d_E(x,x_0) < \delta
\quad \text{pero} \quad
d_F\bigl(f(x),f(x_0)\bigr) \ge \varepsilon_0.
\]

En particular, para cada $n \in \N$, aplicamos esta propiedad con
$\delta = \frac{1}{n}$ y obtenemos un punto $x_n \in A$ tal que
\[
d_E\Bigl(x_n,x_0\Bigr) < \frac{1}{n}
\quad \text{y} \quad
d_F\bigl(f(x_n),f(x_0)\bigr) \ge \varepsilon_0.
\]
De aquí se ve que $x_n \to x_0$ en $(E,d_E)$, porque para todo
$\varepsilon>0$ basta tomar $N$ con $\frac{1}{N} < \varepsilon$;
entonces si $n \ge N$,
\[
d_E(x_n,x_0) < \frac{1}{n} \le \frac{1}{N} < \varepsilon.
\]

Sin embargo, la sucesión $(f(x_n))$ no puede converger a $f(x_0)$ en
$(F,d_F)$, ya que para todo $n$ se tiene
\[
d_F\bigl(f(x_n),f(x_0)\bigr) \ge \varepsilon_0,
\]
y esto contradice la definición de límite. Esto contradice la hipótesis
de continuidad secuencial en $x_0$.

Por lo tanto, nuestra suposición era falsa y $f$ debe ser continua
en $x_0$ en el sentido $\varepsilon$--$\delta$.
\end{proof}

\begin{defi}[Imagen inversa]
Sea $f : A \to F$ y $B \subseteq F$. Definimos la \emph{imagen inversa}
de $B$ por $f$ como
\[
f^{-1}(B) = \{x \in A : f(x) \in B\}.
\]
\end{defi}

\begin{teo}[Continuidad y abiertos]
Sea $f : A \to F$ una función entre espacios métricos. Entonces las
siguientes afirmaciones son equivalentes:
\begin{enumerate}[label=(\roman*)]
    \item $f$ es continua en $A$.
    \item Para todo abierto $G \subseteq F$, el conjunto
    $f^{-1}(G)$ es abierto en el subespacio $A$ (es decir,
    $f^{-1}(G) = A \cap U$ para algún abierto $U$ de $E$).
    \item Si $A = E$, entonces (ii) dice simplemente:
    para todo abierto $G \subseteq F$, $f^{-1}(G)$ es abierto en $E$.
\end{enumerate}
\end{teo}

\begin{proof}
$(i) \Rightarrow (ii)$) Supongamos que $f$ es continua en $A$ y sea
$G \subseteq F$ un conjunto abierto. Consideremos $f^{-1}(G) \subseteq A$.

Tomemos $x_0 \in f^{-1}(G)$. Entonces $f(x_0) \in G$. Como $G$ es abierto
en $(F,d_F)$, existe $\varepsilon > 0$ tal que
\[
B_{d_F}\bigl(f(x_0),\varepsilon\bigr) \subseteq G.
\]
Como $f$ es continua en $x_0$, existe $\delta > 0$ tal que para todo
$x \in A$,
\[
d_E(x,x_0) < \delta \ \Rightarrow\ d_F\bigl(f(x),f(x_0)\bigr) < \varepsilon,
\]
es decir,
\[
x \in B_{d_E}(x_0,\delta) \cap A \ \Rightarrow\ f(x) \in
B_{d_F}\bigl(f(x_0),\varepsilon\bigr) \subseteq G.
\]
Por lo tanto,
\[
B_{d_E}(x_0,\delta) \cap A \subseteq f^{-1}(G).
\]

Esto muestra que, en el subespacio $A$, el punto $x_0$ tiene una “bola”
(dada por $B_{d_E}(x_0,\delta) \cap A$) contenida en $f^{-1}(G)$.
Por definición, $f^{-1}(G)$ es abierto en $A$.

\medskip

$(ii) \Rightarrow (i)$) Supongamos ahora que para todo abierto $G \subseteq F$,
el conjunto $f^{-1}(G)$ es abierto en el subespacio $A$.

Sea $x_0 \in A$ y probemos que $f$ es continua en $x_0$.

Sea $\varepsilon>0$. Consideramos el abierto
\[
G = B_{d_F}\bigl(f(x_0),\varepsilon\bigr) \subseteq F.
\]
Por hipótesis, $f^{-1}(G)$ es abierto en $A$. En particular,
$x_0 \in f^{-1}(G)$ (porque $f(x_0) \in G$), y como $f^{-1}(G)$ es abierto
en el subespacio $A$, existe $\delta>0$ tal que
\[
B_{d_E}(x_0,\delta) \cap A \subseteq f^{-1}(G).
\]

Entonces, si $x \in A$ y $d_E(x,x_0) < \delta$, se tiene $x \in A$ y
$x \in B_{d_E}(x_0,\delta)$, por lo que $x \in f^{-1}(G)$; esto significa
que $f(x) \in G$, es decir,
\[
d_F\bigl(f(x),f(x_0)\bigr) < \varepsilon.
\]

Hemos probado exactamente la condición $\varepsilon$--$\delta$ de continuidad
de $f$ en $x_0$. Como $x_0$ es arbitrario en $A$, $f$ es continua en $A$.

\medskip

La equivalencia con (iii) es sólo una simplificación de notación cuando
$A=E$, ya que en ese caso “abierto en $A$” coincide con “abierto en $E$”.
\end{proof}

\begin{teo}[Continuidad y cerrados]
Sea $f : A \to F$ una función entre espacios métricos. Son equivalentes:
\begin{enumerate}[label=(\roman*)]
    \item $f$ es continua en $A$.
    \item Para todo cerrado $C \subseteq F$, el conjunto
    $f^{-1}(C)$ es cerrado en el subespacio $A$.
\end{enumerate}
\end{teo}

\begin{proof}
$(i) \Rightarrow (ii)$) Supongamos $f$ continua en $A$ y sea $C \subseteq F$
un cerrado. Su complemento $F \setminus C$ es abierto en $F$.

Por el teorema anterior, $f^{-1}(F \setminus C)$ es abierto en $A$.
Notemos que
\[
A \setminus f^{-1}(C) = f^{-1}(F \setminus C).
\]
Entonces el complemento de $f^{-1}(C)$ en $A$ es abierto en $A$, por lo que
$f^{-1}(C)$ es cerrado en $A$.

\medskip

$(ii) \Rightarrow (i)$) Recíprocamente, supongamos que la preimagen de
todo cerrado en $F$ es cerrada en $A$.

Sea $G \subseteq F$ un abierto. Entonces $F \setminus G$ es cerrado en $F$,
y por hipótesis
\[
f^{-1}(F \setminus G)
\]
es cerrado en $A$. Su complemento en $A$,
\[
A \setminus f^{-1}(F \setminus G),
\]
es abierto en $A$. Pero
\[
A \setminus f^{-1}(F \setminus G) = f^{-1}(G).
\]
Concluimos que $f^{-1}(G)$ es abierto en $A$ para todo abierto $G$ de $F$.
Por el teorema anterior, esto implica que $f$ es continua en $A$.
\end{proof}

\begin{teo}[Continuidad y clausura de la imagen]
Sea $f : E \to E'$ una función entre espacios métricos. Entonces las
siguientes afirmaciones son equivalentes:
\begin{enumerate}[label=(\roman*)]
    \item $f$ es continua en $E$.
    \item Para todo subconjunto $A \subseteq E$ se cumple
    \[
    f(\overline{A}) \subseteq \overline{f(A)}.
    \]
\end{enumerate}
\end{teo}

\begin{proof}
$(i) \Rightarrow (ii)$) Supongamos que $f$ es continua en $E$
(en cada punto de $E$).

Sea $A \subseteq E$ y sea $x \in \overline{A}$. Por la caracterización
secuencial de la clausura, existe una sucesión $(a_n)_{n\in\N} \subseteq A$
tal que
\[
a_n \to x \quad \text{en } E.
\]
Como $f$ es continua en $x$, se tiene
\[
f(a_n) \to f(x) \quad \text{en } E'.
\]

Además, cada $f(a_n)$ pertenece a $f(A)$, luego por la caracterización
secuencial de la clausura en $E'$ concluimos que
\[
f(x) \in \overline{f(A)}.
\]

Esto vale para todo $x \in \overline{A}$, por lo tanto
\[
f(\overline{A}) \subseteq \overline{f(A)}.
\]

\medskip

$(ii) \Rightarrow (i)$) Supongamos ahora que para todo $A \subseteq E$
se cumple
\[
f(\overline{A}) \subseteq \overline{f(A)}.
\]
Queremos probar que $f$ es continua en $E$.

Usaremos el criterio de continuidad en términos de cerrados:
$f$ es continua si y sólo si, para todo conjunto cerrado
$C \subseteq E'$, la preimagen $f^{-1}(C)$ es cerrada en $E$.

Sea entonces $C \subseteq E'$ un cerrado y definamos
\[
A = f^{-1}(C) = \{x \in E : f(x) \in C\}.
\]
Tenemos $A \subseteq E$ y, en particular, $f(A) \subseteq C$.
De la hipótesis aplicada a este conjunto $A$ se obtiene
\[
f(\overline{A}) \subseteq \overline{f(A)}.
\]
Como $f(A) \subseteq C$, se cumple
\[
\overline{f(A)} \subseteq \overline{C} = C
\]
(porque $C$ es cerrado). Por lo tanto,
\[
f(\overline{A}) \subseteq C.
\]

Tomando preimagen por $f$ de ambos lados,
\[
f^{-1}\bigl(f(\overline{A})\bigr) \subseteq f^{-1}(C) = A.
\]
En particular, como $\overline{A} \subseteq f^{-1}\bigl(f(\overline{A})\bigr)$
(siempre se cumple $B \subseteq f^{-1}(f(B))$ para cualquier función y
cualquier conjunto $B$), obtenemos
\[
\overline{A} \subseteq A.
\]

Pero siempre se tiene $A \subseteq \overline{A}$, por definición de clausura.
Luego
\[
A \subseteq \overline{A} \subseteq A,
\]
de donde se deduce $\overline{A} = A$. Es decir, $A$ es cerrado en $E$.

Hemos probado que, para todo cerrado $C \subseteq E'$, la preimagen
$f^{-1}(C)$ es cerrada en $E$. Por el criterio de continuidad mediante
cerrados, esto implica que $f$ es continua en $E$.

Así queda demostrada la equivalencia entre (i) y (ii).
\end{proof}

\subsection{Continuidad uniforme}

Sea $f : E \to E'$ entre espacios métricos $(E,d)$ y $(E',d')$.

\begin{defi}[Continuidad uniforme, versión con bolas]
Decimos que $f$ es \emph{uniformemente continua} en $E$ si
\[
\forall \varepsilon>0\ \exists \delta>0\ \text{tal que}\ 
f\bigl(B_d(x,\delta)\bigr) \subseteq B_{d'}(f(x),\varepsilon)
\quad \text{para todo } x \in E.
\]
Es decir: dado $\varepsilon>0$ se puede elegir un mismo $\delta>0$
(válido para todos los puntos $x$) de modo que, siempre que $y$ esté
$\delta$-cerca de $x$, las imágenes $f(y)$ y $f(x)$ estén
$\varepsilon$-cerca.
\end{defi}

\begin{defi}[Definición equivalente, versión $\varepsilon$--$\delta$]
Equivalente y más usada:
$f$ es \emph{uniformemente continua} en $E$ si
\[
\forall \varepsilon>0\ \exists \delta>0\ \text{tal que}\ 
\forall x,y \in E,\ d(x,y) < \delta \ \Rightarrow\ d'\bigl(f(x),f(y)\bigr) < \varepsilon.
\]
Aquí $\delta$ depende sólo de $\varepsilon$, no de los puntos $x,y$.
\end{defi}

\begin{prop}
Las dos definiciones anteriores de continuidad uniforme son equivalentes.
\end{prop}

\begin{proof}
Supongamos primero la definición con bolas. Sea $\varepsilon>0$ y
sea $\delta>0$ el correspondiente en esa definición. Si $x,y \in E$
y $d(x,y) < \delta$, entonces $y \in B_d(x,\delta)$ y, por la hipótesis,
\[
f(y) \in f\bigl(B_d(x,\delta)\bigr)
\subseteq B_{d'}(f(x),\varepsilon),
\]
es decir, $d'(f(x),f(y)) < \varepsilon$. Esto da la definición
$\varepsilon$--$\delta$.

Recíprocamente, supongamos que vale la definición $\varepsilon$--$\delta$.
Fijamos $\varepsilon>0$ y tomamos el correspondiente $\delta>0$.
Sea $x \in E$ y $y \in B_d(x,\delta)$, entonces $d(x,y) < \delta$ y por
la hipótesis
\[
d'(f(x),f(y)) < \varepsilon,
\]
es decir, $f(y) \in B_{d'}(f(x),\varepsilon)$. Por lo tanto,
\[
f\bigl(B_d(x,\delta)\bigr) \subseteq B_{d'}(f(x),\varepsilon)
\quad \text{para todo } x \in E,
\]
que es la definición con bolas.
\end{proof}

\begin{prop}[Criterio secuencial para \emph{no} continuidad uniforme]
Sea $f : E \to E'$. Entonces $f$ \textbf{no} es uniformemente continua en $E$
si y sólo si existen $\varepsilon_0>0$ y sucesiones
$(x_n)_{n\in\N}$, $(y_n)_{n\in\N}$ en $E$ tales que
\[
d(x_n,y_n) \xrightarrow[n\to\infty]{} 0
\quad \text{pero} \quad
d'\bigl(f(x_n),f(y_n)\bigr) \ge \varepsilon_0
\quad \text{para todo } n \in \N.
\]
\end{prop}

\begin{proof}
($\Rightarrow$) Supongamos que $f$ no es uniformemente continua.
Entonces existe $\varepsilon_0>0$ tal que \emph{para todo} $\delta>0$
no se cumple la condición de continuidad uniforme, es decir:
\[
\forall \delta>0\ \exists x,y \in E\ \text{con}\ d(x,y)<\delta
\ \text{y}\ d'\bigl(f(x),f(y)\bigr) \ge \varepsilon_0.
\]

Para cada $n \in \N$, aplicamos esto con $\delta = \frac{1}{n}$, obteniendo
puntos $x_n,y_n \in E$ tales que
\[
d(x_n,y_n) < \frac{1}{n}
\quad \text{y} \quad
d'\bigl(f(x_n),f(y_n)\bigr) \ge \varepsilon_0.
\]
Entonces $d(x_n,y_n) \to 0$ cuando $n \to \infty$, mientras que
las distancias entre las imágenes están siempre acotadas inferiormente
por $\varepsilon_0$. Esto da las sucesiones pedidas.

\medskip

($\Leftarrow$) Recíprocamente, supongamos que existen $\varepsilon_0>0$
y sucesiones $(x_n)$, $(y_n)$ en $E$ tales que
\[
d(x_n,y_n) \to 0
\quad \text{y} \quad
d'\bigl(f(x_n),f(y_n)\bigr) \ge \varepsilon_0\ \forall n.
\]

Supongamos, por absurdo, que $f$ fuera uniformemente continua.
Tomemos $\varepsilon = \varepsilon_0$ en la definición de continuidad
uniforme. Entonces existiría $\delta>0$ tal que, para todo $x,y \in E$,
\[
d(x,y) < \delta \ \Rightarrow\ d'\bigl(f(x),f(y)\bigr) < \varepsilon_0.
\]

Como $d(x_n,y_n) \to 0$, existe $N \in \N$ tal que, para todo $n \ge N$,
\[
d(x_n,y_n) < \delta.
\]
Aplicando la condición de uniformidad, tendríamos
\[
d'\bigl(f(x_n),f(y_n)\bigr) < \varepsilon_0
\quad \text{para todo } n \ge N,
\]
lo que contradice la hipótesis
$d'(f(x_n),f(y_n)) \ge \varepsilon_0$ para todo $n$.

Por lo tanto, $f$ no puede ser uniformemente continua.
\end{proof}

\begin{defi}[Aplicación de Lipschitz]
Sea $C>0$. Decimos que $f : E \to E'$ es \emph{Lipschitz} (o
\emph{$C$-Lipschitz}) si
\[
d'\bigl(f(x),f(y)\bigr) \le C\, d(x,y)
\quad \text{para todo } x,y \in E.
\]
\end{defi}

\begin{teo}
Sea $f : E \to E'$. Si existe $C>0$ tal que
\[
d'\bigl(f(x),f(y)\bigr) \le C\, d(x,y)
\quad \text{para todo } x,y \in E,
\]
entonces $f$ es uniformemente continua en $E$.
\end{teo}

\begin{proof}
Sea $\varepsilon>0$. Definimos
\[
\delta = \frac{\varepsilon}{C} > 0.
\]
Si $x,y \in E$ satisfacen $d(x,y) < \delta$, entonces
\[
d'\bigl(f(x),f(y)\bigr) \le C\, d(x,y) < C \delta = \varepsilon.
\]
Esto es exactamente la definición de continuidad uniforme
(con $\delta = \varepsilon/C$). Por lo tanto $f$ es uniformemente
continua.
\end{proof}

\subsubsection{Isometrías}

\begin{defi}
Sea $f : E \to E'$ entre espacios métricos $(E,d)$ y $(E',d')$.
Decimos que $f$ es una \emph{isometría} si
\[
d(x,y) = d'\bigl(f(x),f(y)\bigr)
\quad \text{para todo } x,y \in E.
\]
\end{defi}

\begin{prop}
Sea $f : E \to E'$ una isometría. Entonces:
\begin{enumerate}[label=(\roman*)]
    \item $f$ es inyectiva.
    \item $f$ es $1$-Lipschitz, luego uniformemente continua.
\end{enumerate}
\end{prop}

\begin{proof}
(i) Si $f(x) = f(y)$, entonces
\[
d(x,y) = d'\bigl(f(x),f(y)\bigr) = d'(f(x),f(x)) = 0,
\]
lo que implica $x=y$. Por lo tanto, $f$ es inyectiva.

(ii) De la definición de isometría se tiene directamente
\[
d'\bigl(f(x),f(y)\bigr) = d(x,y) \le 1 \cdot d(x,y)
\quad \forall x,y \in E,
\]
es decir, $f$ es $1$-Lipschitz. Por el teorema anterior, toda
aplicación Lipschitz es uniformemente continua, así que $f$ lo es.
\end{proof}

\begin{obs}
Si $f : E \to E'$ es una isometría \emph{biyectiva}, entonces su inversa
$f^{-1} : E' \to E$ también es una isometría: para $u,v \in E'$,
escribiendo $u = f(x)$ y $v = f(y)$ con $x,y \in E$, se tiene
\[
d\bigl(f^{-1}(u),f^{-1}(v)\bigr) = d(x,y) = d'\bigl(f(x),f(y)\bigr)
= d'(u,v).
\]
En particular, $f^{-1}$ también es uniformemente continua.
\end{obs}

\subsubsection{Homeomorfismos}

\begin{defi}
Sea $f : E \to E'$ una función entre espacios métricos.
Decimos que $f$ es un \emph{homeomorfismo} si:
\begin{enumerate}[label=(\roman*)]
    \item $f$ es biyectiva,
    \item $f$ es continua,
    \item la inversa $f^{-1} : E' \to E$ es continua.
\end{enumerate}
\end{defi}

\begin{defi}
Dos espacios métricos $(E,d)$ y $(E',d')$ se dicen \emph{homeomorfos}
si existe un homeomorfismo $f : E \to E'$.
\end{defi}

\begin{prop}
Si $f : E \to E'$ es un homeomorfismo, entonces:
\begin{enumerate}[label=(\roman*)]
    \item $G \subseteq E$ es abierto si y sólo si $f(G)$ es abierto en $E'$.
    \item $F \subseteq E$ es cerrado si y sólo si $f(F)$ es cerrado en $E'$.
\end{enumerate}
\end{prop}

\begin{proof}
(i) Si $G$ es abierto en $E$ y $f$ es continua, entonces
$f(G)$ es abierto en $E'$ si y sólo si $f^{-1}$ es continua,
porque $G = f^{-1}(f(G))$ y $f^{-1}$ es continua por hipótesis.

Formalmente: como $f$ es biyectiva, $G = f^{-1}(H)$ para $H = f(G)$.
La continuidad de $f^{-1}$ implica que, si $H$ es abierto en $E'$,
entonces $G$ es abierto en $E$. Inversamente, dado $G$ abierto en $E$,
$H=f(G)$ es abierto en $E'$ porque $H$ es la imagen inversa de un
abierto por $f^{-1}$ (que es continua).

(ii) Se deduce de (i) aplicando el resultado a los complementos:
$F$ es cerrado en $E$ si y sólo si $E \setminus F$ es abierto, y
la imagen por $f$ de $E \setminus F$ es el complemento de $f(F)$ en $E'$.
\end{proof}

\begin{obs}
En general, una aplicación biyectiva y continua \emph{no} tiene por qué
tener inversa continua, es decir, no todo biectivo continuo es un
homeomorfismo. El requisito $f^{-1}$ continua es esencial en la definición.
\end{obs}

\subsubsection{Conjuntos densos}

\begin{defi}
Sea $(E,d)$ un espacio métrico. Un subconjunto $D \subseteq E$ se dice
\emph{denso en $E$} si
\[
\overline{D} = E.
\]
Equivalente: todo punto de $E$ es límite de una sucesión de puntos de $D$.
\end{defi}

\begin{prop}
Sea $(E,d)$ un espacio métrico y $D \subseteq E$. Son equivalentes:
\begin{enumerate}[label=(\roman*)]
    \item $D$ es denso en $E$, es decir, $\overline{D} = E$.
    \item Para todo abierto no vacío $G \subseteq E$ se cumple
    \[
    G \cap D \neq \varnothing.
    \]
\end{enumerate}
\end{prop}

\begin{proof}
$(i) \Rightarrow (ii)$) Supongamos $\overline{D} = E$. Sea $G$ un abierto
no vacío en $E$. Como $G \subseteq E = \overline{D}$, y $G$ es abierto,
se cumple $G \cap D \neq \varnothing$ (porque todo abierto que intersecta
la clausura de un conjunto debe intersectar al conjunto mismo).

Más formalmente: si $G \cap D = \varnothing$, entonces $G \subseteq E\setminus D$,
y el complemento $E\setminus D$ sería un cerrado que contiene a $G$,
contradiciendo que $G \subseteq \overline{D}$.

\medskip

$(ii) \Rightarrow (i)$) Supongamos que todo abierto no vacío $G$ verifica
$G \cap D \neq \varnothing$. Si existiera $x \in E \setminus \overline{D}$,
entonces $x$ estaría en el abierto $E \setminus \overline{D}$, que por
definición de clausura no intersecta a $D$. Esto contradice (ii).
Por lo tanto, $E \setminus \overline{D} = \varnothing$, es decir,
$\overline{D} = E$.
\end{proof}
\section{Unidad 5: Compacidad}
\section{Unidad 5: Compacidad}

\subsection{Definiciones básicas}

\begin{defi}[Conjunto compacto]
Sea $(E,d)$ un espacio métrico y $K \subseteq E$.
Decimos que $K$ es \emph{compacto} si para toda familia de abiertos
$\{G_i\}_{i \in I}$ de $E$ tal que
\[
K \subseteq \bigcup_{i \in I} G_i,
\]
existe un subíndice finito $i_1,\dots,i_n \in I$ con
\[
K \subseteq \bigcup_{k=1}^n G_{i_k}.
\]
Es decir, todo recubrimiento abierto de $K$ admite un subcubrimiento
finito.
\end{defi}

\begin{defi}[Compacidad secuencial]
Sea $(E,d)$ un espacio métrico y $K \subseteq E$.
Decimos que $K$ es \emph{secuencialmente compacto} si para toda sucesión
$(x_n)_{n\in\N}$ con $x_n \in K$ para todo $n$, existe una subsucesión
$(x_{n_j})_{j\in\N}$ y un punto $x \in K$ tales que
\[
x_{n_j} \xrightarrow[j\to\infty]{} x.
\]
\end{defi}

\begin{defi}[Conjunto totalmente acotado]
Sea $(E,d)$ un espacio métrico y $A \subseteq E$.
Decimos que $A$ es \emph{totalmente acotado} (T.T.A.) si
\[
\forall \varepsilon>0\ \exists x_1,\dots,x_{n(\varepsilon)} \in E
\ \text{tales que}\ 
A \subseteq \bigcup_{k=1}^{n(\varepsilon)} B(x_k,\varepsilon),
\]
donde $B(x_k,\varepsilon)=\{y\in E : d(x_k,y)<\varepsilon\}$.
\end{defi}

\begin{defi}[Punto de acumulación]
Sea $(E,d)$ un espacio métrico y $A \subseteq E$.
Un punto $x \in E$ se llama \emph{punto de acumulación} (o punto límite)
de $A$ si
\[
\forall r>0 \quad \bigl(B(x,r)\setminus\{x\}\bigr) \cap A \neq \varnothing.
\]
Denotamos por $A'$ al conjunto de puntos de acumulación de $A$.
\end{defi}

\begin{prop}[Caracterización secuencial de puntos de acumulación]
Sea $A \subseteq E$ y $x \in E$. Entonces
\[
x \in A' \iff \exists (a_n)_{n\in\N} \subseteq A\setminus\{x\}
\text{ tal que } a_n \to x.
\]
\end{prop}

\subsection{Compacidad y puntos de acumulación}

\begin{teo}[Equivalencias de compacidad en espacios métricos]
Sea $(E,d)$ un espacio métrico y $K \subseteq E$. Son equivalentes:
\begin{enumerate}[label=(\roman*)]
    \item $K$ es compacto.
    \item $K$ es secuencialmente compacto.
    \item Todo subconjunto infinito $A \subseteq K$ tiene al menos un punto
    de acumulación en $K$, es decir, $A' \cap K \neq \varnothing$.
\end{enumerate}
\end{teo}

\begin{proof}
$(i) \Rightarrow (ii)$.
Sea $(x_n)$ una sucesión en $K$. Si el conjunto de valores
$A=\{x_n:n\in\N\}$ es finito, alguna de sus constantes se repite
infinitas veces y obtenemos una subsucesión constante, luego convergente.

Si $A$ es infinito, como $K$ es compacto, también lo es $\overline{A}$,
y por tanto $A$ tiene un punto de acumulación $x\in K$.
Por la caracterización secuencial de puntos de acumulación, existe una
subsucesión $(x_{n_j})$ con $x_{n_j}\to x\in K$.
En ambos casos, $(x_n)$ admite una subsucesión convergente en $K$.

\medskip

$(ii) \Rightarrow (iii)$.
Sea $A\subseteq K$ infinito. Elegimos una sucesión de puntos distintos
$(x_n)\subseteq A$ (por ejemplo, una enumeración inyectiva de $A$).
Por (ii), existe una subsucesión $(x_{n_j})$ y un punto $x\in K$ tales
que $x_{n_j}\to x$. Como todos los $x_{n_j}\in A$, la caracterización
secuencial de puntos de acumulación dice que $x\in A'$. Por tanto
$A'\cap K\neq \varnothing$.

\medskip

$(iii) \Rightarrow (ii)$.
Sea $(x_n)$ una sucesión en $K$ y sea $A=\{x_n:n\in\N\}\subseteq K$.
Si $A$ es finito, como antes obtenemos una subsucesión constante y
convergente.

Si $A$ es infinito, por (iii) existe $x\in A'\cap K$. Entonces, por la
caracterización secuencial de puntos de acumulación, existe una
subsucesión $(x_{n_j})$ de $(x_n)$ tal que $x_{n_j}\to x\in K$.
En todo caso, $(x_n)$ admite una subsucesión convergente en $K$.

\medskip

$(ii) \Rightarrow (i)$.
Probamos la contrarrecíproca. Supongamos que $K$ no es compacto.
Entonces existe un recubrimiento abierto $\{G_i\}_{i\in I}$ de $K$
que no admite subcubrimiento finito.

Construimos inductivamente una sucesión $(x_n)$ en $K$ del siguiente modo:

- Elegimos $x_1\in K$ y un índice $i_1$ con $x_1\in G_{i_1}$.
- Supuesto elegido $x_1,\dots,x_n$ con índices $i_1,\dots,i_n$ tales
  que $x_k\in G_{i_k}$, como
  \(
    K \nsubseteq \bigcup_{k=1}^n G_{i_k}
  \)
  existe
  \(
    x_{n+1}\in K \setminus \bigcup_{k=1}^n G_{i_k}.
  \)
  Elegimos $i_{n+1}$ con $x_{n+1}\in G_{i_{n+1}}$.

Entonces cada $x_n$ pertenece a un abierto nuevo de la familia, distinto
de los anteriores. Sea $(x_{n_j})$ una subsucesión cualquiera.
Si fuera convergente, digamos $x_{n_j}\to x\in K$, algún abierto $G_i$
contendría a $x$ y a todos los términos suficientemente avanzados de
la subsucesión; en particular, ese solo $G_i$ cubriría casi todos los
$x_n$, y un número finito de abiertos de la familia cubriría $K$,
contradiciendo la elección de $\{G_i\}$ sin subcubrimiento finito.
Luego $(x_n)$ no posee subsucesión convergente, y $K$ no es secuencialmente
compacto.

La contrarrecíproca muestra que (ii) implica (i).
\end{proof}

\subsection{Compacidad, completitud y total acotación}

\begin{teo}
Sea $(E,d)$ un espacio métrico y $K \subseteq E$. Entonces:
\begin{enumerate}[label=(\roman*)]
    \item Si $K$ es compacto, entonces $K$ es completo y totalmente acotado.
    \item Si $K$ es completo y totalmente acotado, entonces $K$ es compacto.
\end{enumerate}
En particular,
\[
K \text{ compacto} \iff
K \text{ completo y totalmente acotado}.
\]
\end{teo}

\begin{proof}
(i) Si $K$ es compacto, por el teorema anterior es secuencialmente
compacto. Toda sucesión de Cauchy en $K$ admite una subsucesión convergente;
un argumento estándar muestra que la sucesión completa converge al mismo
límite, que pertenece a $K$, de modo que $K$ es completo.

Para ver que $K$ es T.T.A., sea $\varepsilon>0$ y consideremos la familia
de bolas $\{B(x,\varepsilon):x\in K\}$, que es un recubrimiento abierto
de $K$. Por compacidad, existe un subcubrimiento finito
$K\subset \bigcup_{k=1}^n B(x_k,\varepsilon)$, que es precisamente la
definición de T.T.A.

\medskip

(ii) Supongamos que $K$ es completo y totalmente acotado.
Sea $(x_n)$ una sucesión en $K$. Por total acotación, usando el clásico
método de “subsucesiones encajadas”, se puede extraer una subsucesión
$(x_{n_j})$ que resulta ser de Cauchy. Como $K$ es completo, esa
subsucesión converge en $K$. Así, toda sucesión en $K$ tiene una
subsucesión convergente en $K$, es decir, $K$ es secuencialmente compacto.
Por el teorema anterior, $K$ es compacto.
\end{proof}

\begin{prop}
Si $(E,d)$ es un espacio métrico y $K \subseteq E$ es compacto, entonces
$K$ es cerrado y acotado.
\end{prop}

\begin{proof}
Ya vimos que todo compacto es totalmente acotado, luego es acotado.

Resta ver que es cerrado. Sea $\overline{K}$ la clausura de $K$.
Si $x\in \overline{K}$, existe una sucesión $(x_n)\subseteq K$
tal que $x_n\to x$. Como $K$ es compacto, es secuencialmente compacto,
por lo que $(x_n)$ admite una subsucesión convergente con límite en $K$;
la unicidad del límite en espacios métricos fuerza que ese límite sea $x$,
por lo que $x\in K$. Se obtiene $\overline{K}\subseteq K$, y como siempre
$K\subseteq\overline{K}$, concluimos $\overline{K}=K$, es decir, $K$ es
cerrado.
\end{proof}

\begin{teo}[Heine--Borel en $\R^m$]
Sea $K \subseteq \R^m$ con la métrica euclídea usual. Entonces
\[
K \text{ es compacto} \iff K \text{ es cerrado y acotado}.
\]
\end{teo}

\subsection{Propiedades estructurales de compactos y T.T.A.}

\begin{prop}[Subconjunto cerrado de un compacto]
Sea $(E,d)$ un espacio métrico y $K \subseteq E$ compacto.
Si $F \subseteq K$ es cerrado (en $E$), entonces $F$ es compacto.
\end{prop}

\begin{proof}
Sea $\{G_\alpha\}_{\alpha\in A}$ un recubrimiento abierto de $F$ en $E$.
Como $F$ es cerrado, $E\setminus F$ es abierto. Entonces
\[
\{G_\alpha : \alpha\in A\} \cup \{E\setminus F\}
\]
es un recubrimiento abierto de $K$. Por compacidad, existe una subfamilia
finita que recubre $K$, y por lo tanto $F$ queda cubierto por una subfamilia
finita de los $G_\alpha$. Así, $F$ es compacto.
\end{proof}

\begin{cor}
La intersección arbitraria de subconjuntos compactos de un espacio
métrico es un conjunto compacto (en particular, $\varnothing$ es compacto).
\end{cor}

\begin{proof}
Si $\{K_i\}_{i\in I}$ son compactos, cada uno es cerrado en $E$.
La intersección $K = \bigcap_{i\in I} K_i$ es cerrada y satisface
$K\subseteq K_{i_0}$ para cualquier índice fijo $i_0$. Como $K$ es un
subconjunto cerrado de $K_{i_0}$, por la proposición anterior $K$ es compacto.
El caso $K=\varnothing$ también es compacto por definición.
\end{proof}

\begin{prop}[Propiedades de conjuntos totalmente acotados]
Sea $(E,d)$ un espacio métrico.
\begin{enumerate}[label=(\roman*)]
    \item Si $A \subseteq B \subseteq E$ y $B$ es T.T.A., entonces $A$ es T.T.A.
    \item Si $A$ es T.T.A., entonces $\overline{A}$ es T.T.A.
    \item Si $A$ y $B$ son T.T.A., entonces $A \cup B$ es T.T.A.
    (en general, la unión finita de conjuntos T.T.A. es T.T.A.).
    \item Si $A$ es T.T.A., entonces $A$ es acotado.
\end{enumerate}
\end{prop}

\begin{proof}
(i) y (iii) se obtienen directamente de la definición observando que
un recubrimiento finito de $B$ o de $A$ y $B$ provee uno de $A$ o
de $A\cup B$.

(ii) Dado $\varepsilon>0$, se cubre $A$ con finitas bolas
$B(x_k,\varepsilon/2)$; se comprueba que la clausura de cada una de ellas
queda incluida en $B(x_k,\varepsilon)$, y se deduce que
$\overline{A}$ queda cubierta por la unión finita
\(\bigcup_k B(x_k,\varepsilon)\).

(iv) Tomando $\varepsilon=1$, se ve que $A$ queda contenido en la unión
finita de bolas de radio $1$, por lo que todo punto de $A$ está a distancia
acotada de cualquiera de sus centros; esto da una cota global para $A$.
\end{proof}

\begin{prop}[Imagen de T.T.A. por función uniformemente continua]
Sean $(E,d)$ y $(E',d')$ espacios métricos, $A \subseteq E$
totalmente acotado y $f : E \to E'$ uniformemente continua.
Entonces $f(A)$ es totalmente acotado en $(E',d')$.
\end{prop}

\begin{proof}
Sea $\varepsilon>0$. Por uniformidad de $f$, existe $\delta>0$ tal que
\[
d(x,y)<\delta \Rightarrow d'(f(x),f(y))<\varepsilon
\quad \forall x,y\in E.
\]
Como $A$ es T.T.A., existen $x_1,\dots,x_n$ con
\[
A \subseteq \bigcup_{k=1}^n B(x_k,\delta).
\]
Entonces
\[
f(A) \subseteq \bigcup_{k=1}^n f(B(x_k,\delta))
\subseteq \bigcup_{k=1}^n B_{d'}(f(x_k),\varepsilon).
\]
Hemos cubierto $f(A)$ con finit­as bolas de radio $\varepsilon$; como
$\varepsilon$ era arbitrario, $f(A)$ es T.T.A.
\end{proof}

\subsection{Funciones continuas sobre compactos}

\begin{teo}[Imagen continua de un compacto]
Sean $(E,d)$ y $(E',d')$ espacios métricos y sea
$f : E \to E'$ continua. Si $K \subseteq E$ es compacto, entonces
$f(K)$ es compacto en $E'$.
\end{teo}

\begin{proof}
Sea $\{V_\alpha\}_{\alpha\in A}$ un recubrimiento abierto de $f(K)$ en $E'$.
Para cada $\alpha$ definimos $U_\alpha = f^{-1}(V_\alpha)$, que es abierto
en $E$ por continuidad de $f$. Además,
\[
K \subseteq \bigcup_{\alpha\in A} U_\alpha,
\]
pues si $x\in K$ entonces $f(x)\in f(K)\subseteq \bigcup V_\alpha$.

Como $K$ es compacto, existe un subcubrimiento finito
$K \subseteq \bigcup_{j=1}^n U_{\alpha_j}$. Aplicando $f$,
\[
f(K) \subseteq \bigcup_{j=1}^n f(U_{\alpha_j})
\subseteq \bigcup_{j=1}^n V_{\alpha_j},
\]
que es un subcubrimiento finito de $f(K)$. Así, $f(K)$ es compacto.
\end{proof}

\begin{cor}[Teorema de Weierstrass]
Sea $K \subseteq E$ compacto y $f : K \to \R$ continua.
Entonces:
\begin{enumerate}[label=(\roman*)]
    \item $f$ es acotada en $K$: existe $c>0$ tal que
    $|f(x)|\le c$ para todo $x\in K$;
    \item $f$ alcanza su máximo y su mínimo en $K$, es decir,
    existen $x_{\min},x_{\max}\in K$ con
    \[
    f(x_{\min}) \le f(x) \le f(x_{\max}) \quad \forall x\in K.
    \]
\end{enumerate}
\end{cor}

\begin{proof}
Por el teorema anterior, $f(K)$ es compacto en $\R$. Todo compacto
no vacío de $\R$ es cerrado y acotado, de modo que admite supremo e
ínfimo que pertenecen al conjunto; eso da el máximo y el mínimo.
La acotación se deduce de la existencia de supremo y de ínfimo.
\end{proof}

\begin{teo}[Continuidad uniforme en compactos]
Sean $(E,d)$ y $(E',d')$ espacios métricos y sea
$f : E \to E'$ continua. Si $E$ es compacto, entonces $f$ es
uniformemente continua.
\end{teo}

\begin{proof}
Sea $\varepsilon>0$. Para cada $x\in E$, por continuidad de $f$ en $x$
existe $\delta_x>0$ tal que
\[
d(x,y)<\delta_x \Rightarrow d'(f(x),f(y))<\varepsilon
\quad \forall y\in E.
\]
La familia de bolas $B(x,\delta_x)$ (o $B(x,\delta_x/2)$) es un
cubrimiento abierto de $E$. Por compacidad, existe un subcubrimiento
finito
\[
E \subseteq \bigcup_{k=1}^n B\bigl(x_k,\delta_{x_k}\bigr).
\]
Definimos
\[
\delta = \min_{1\le k\le n} \delta_{x_k} > 0.
\]

Si $x,y\in E$ satisfacen $d(x,y)<\delta$, entonces ambos pertenecen a
alguna de las bolas del subcubrimiento (por ejemplo, a la de centro
$x_k$ que contiene a $x$), y como en esa bola vale la condición
de continuidad con $\delta_{x_k}\ge\delta$, obtenemos
$d'(f(x),f(y))<\varepsilon$. Por lo tanto,
\[
\forall \varepsilon>0\ \exists \delta>0\ \forall x,y\in E:
d(x,y)<\delta \Rightarrow d'(f(x),f(y))<\varepsilon,
\]
y $f$ es uniformemente continua.
\end{proof}

\begin{cor}[Funciones continuas sobre compacto son cerradas]
Sean $(E,d)$ y $(E',d')$ espacios métricos, $E$ compacto y
$f : E \to E'$ continua. Entonces $f$ es una aplicación cerrada:
si $G \subseteq E$ es cerrado, entonces $f(G)$ es cerrado en $E'$.
\end{cor}

\begin{proof}
Si $G$ es cerrado en $E$, entonces $G$ es compacto (cerrado en un
compacto). Por el teorema de la imagen de un compacto, $f(G)$ es
compacto en $E'$. En un espacio métrico, todo compacto es cerrado,
así que $f(G)$ es cerrado.
\end{proof}

\begin{cor}[Homeomorfismos desde compactos]
Sean $(E,d)$ y $(E',d')$ espacios métricos, con $E$ compacto, y sea
$f : E \to E'$ continua y biyectiva. Entonces $f$ es un homeomorfismo,
es decir, $f^{-1} : E' \to E$ es continua.
\end{cor}

\begin{proof}
Del corolario anterior, $f$ es una aplicación cerrada.  
Sea $F' \subseteq E'$ cerrado. Como $f$ es biyectiva, se tiene
\[
f^{-1}(F') \subseteq E.
\]
Además
\[
E' \setminus F' \text{ es abierto} \Rightarrow
f^{-1}(E'\setminus F') = E \setminus f^{-1}(F')
\text{ es abierto en }E,
\]
por continuidad de $f$. Por lo tanto $f^{-1}(F')$ es cerrado en $E$.
La preimagen por $f^{-1}$ de todo cerrado de $E$ es cerrada, lo que
equivale a la continuidad de $f^{-1}$. Luego $f$ es un homeomorfismo.
\end{proof}

\section{Unidad 6: Espacios normados}
\section{Unidad 7: Sucesiones de funciones}
\subsection{Convergencia puntual y uniforme}

\begin{defi}[Convergencia puntual]
Sean $(E,d)$ y $(E',d')$ espacios métricos y sea
$(f_n)_{n\in\N}$ una sucesión de funciones $f_n : E \to E'$.
Decimos que $(f_n)$ \emph{converge puntualmente} a una función
$f : E \to E'$ si
\[
\forall x \in E:\quad \lim_{n\to\infty} d'\bigl(f_n(x),f(x)\bigr) = 0.
\]
Equivalentemente,
\[
\forall x \in E,\ \forall \varepsilon>0\ \exists N\in\N\ \forall n\ge N:
\ d'\bigl(f_n(x),f(x)\bigr) < \varepsilon.
\]
En este caso escribimos $f_n(x) \to f(x)$ puntualmente, o simplemente
$f_n \to f$ puntualmente en $E$.
\end{defi}

\begin{defi}[Convergencia uniforme]
Con la notación anterior, diremos que $(f_n)$ \emph{converge uniformemente}
a $f$ en $E$ si
\[
\forall \varepsilon>0\ \exists N\in\N\ \forall n\ge N\ \forall x\in E:
\ d'\bigl(f_n(x),f(x)\bigr) < \varepsilon.
\]
En este caso escribimos $f_n \rightrightarrows f$ en $E$.
\end{defi}

\subsection{Límite uniforme de funciones continuas}

\begin{teo}
Sea $(E,d)$, $(E',d')$ espacios métricos y sea
$(f_n)_{n\in\N}$ una sucesión de funciones continuas
$f_n : E \to E'$, que converge uniformemente a
$f : E \to E'$. Entonces $f$ es continua.
\end{teo}

\begin{proof}
Sea $x_0 \in E$ y sea $\varepsilon>0$ dado.  
Como $f_n \rightrightarrows f$, existe $N \in \N$ tal que
\[
d'\bigl(f_n(x),f(x)\bigr) < \frac{\varepsilon}{3}
\quad \forall x\in E,\ \forall n\ge N.
\]
En particular,
\[
d'\bigl(f_N(x_0),f(x_0)\bigr) < \frac{\varepsilon}{3}.
\]

Como $f_N$ es continua en $x_0$, existe $\delta>0$ tal que
\[
d(x,x_0) < \delta \Rightarrow
d'\bigl(f_N(x),f_N(x_0)\bigr) < \frac{\varepsilon}{3}.
\]

Tomemos ahora un $x\in E$ con $d(x,x_0)<\delta$. Entonces
\[
\begin{aligned}
d'\bigl(f(x),f(x_0)\bigr)
&\le d'\bigl(f(x),f_N(x)\bigr)
   + d'\bigl(f_N(x),f_N(x_0)\bigr)
   + d'\bigl(f_N(x_0),f(x_0)\bigr) \\
&< \frac{\varepsilon}{3} + \frac{\varepsilon}{3} + \frac{\varepsilon}{3}
= \varepsilon.
\end{aligned}
\]
Como $\varepsilon>0$ era arbitrario, esto prueba que $f$ es continua en
$x_0$. Dado que $x_0$ era un punto cualquiera de $E$, $f$ es continua
en todo $E$.
\end{proof}

\subsection{Pasaje al límite bajo el signo integral}

\begin{prop}
Sea $[a,b] \subset \R$ con $a<b$ y sean $f_n,f : [a,b]\to\R$ funciones
continuas tales que $f_n \rightrightarrows f$ en $[a,b]$. Entonces
\[
\lim_{n\to\infty} \int_a^b f_n(t)\,dt
= \int_a^b f(t)\,dt.
\]
\end{prop}

\begin{proof}
Sea $\varepsilon>0$ arbitrario. Como $f_n \rightrightarrows f$ en $[a,b]$,
por definición de convergencia uniforme existe $N\in\N$ tal que
\[
\forall n \ge N\ \forall x\in[a,b]:
\ |f_n(x)-f(x)| < \frac{\varepsilon}{b-a}.
\]

Fijemos $n \ge N$. Entonces, para todo $t\in[a,b]$,
\[
|f_n(t)-f(t)| < \frac{\varepsilon}{b-a}.
\]
Integrando en el intervalo $[a,b]$ y usando la desigualdad
triangular para integrales, obtenemos
\[
\begin{aligned}
\left|\int_a^b f_n(t)\,dt - \int_a^b f(t)\,dt\right|
&= \left|\int_a^b (f_n(t)-f(t))\,dt\right| \\
&\le \int_a^b |f_n(t)-f(t)|\,dt \\
&\le \int_a^b \frac{\varepsilon}{b-a}\,dt \\
&= \frac{\varepsilon}{b-a}\,(b-a) = \varepsilon.
\end{aligned}
\]

Hemos probado que
\[
\forall \varepsilon>0\ \exists N\in\N\ \forall n\ge N:
\left|\int_a^b f_n(t)\,dt - \int_a^b f(t)\,dt\right| < \varepsilon,
\]
lo cual es precisamente
\[
\lim_{n\to\infty} \int_a^b f_n(t)\,dt
= \int_a^b f(t)\,dt.
\]
\end{proof}

\subsection{Convergencia de derivadas}

\begin{prop}
Sean $f_n : [a,b]\to\R$ funciones de clase $C^1$ en $[a,b]$, tales que
\begin{itemize}
    \item $f_n \to f$ puntualmente en $[a,b]$;
    \item $f_n' \rightrightarrows g$ en $[a,b]$.
\end{itemize}
Entonces $f$ es derivable en $[a,b]$ y
\[
f' = g.
\]
\end{prop}

\begin{proof}
Sea $x_0 \in [a,b]$ fijo. Por el Teorema Fundamental del Cálculo aplicado
a cada $f_n$, para todo $x\in[a,b]$ se cumple
\[
f_n(x) - f_n(x_0) = \int_{x_0}^x f_n'(t)\,dt.
\]

Tomando límite cuando $n\to\infty$ en ambos lados:

- Por la convergencia puntual $f_n(x)\to f(x)$ y $f_n(x_0)\to f(x_0)$,
  el lado izquierdo converge a $f(x)-f(x_0)$.

- Por la proposición anterior (pasaje al límite bajo el integral) y la
  convergencia uniforme de $f_n'$ a $g$, el lado derecho converge a
  \(
  \displaystyle \int_{x_0}^x g(t)\,dt.
  \)

Por lo tanto,
\[
f(x) - f(x_0) = \int_{x_0}^x g(t)\,dt
\quad\text{para todo } x\in[a,b].
\]

Definamos
\[
F(x) := f(x_0) + \int_{x_0}^x g(t)\,dt.
\]
La función $F$ es de clase $C^1$ en $[a,b]$ y satisface $F' = g$.
Además, la igualdad anterior muestra que $f(x)=F(x)$ para todo $x$.
Luego $f$ es derivable y $f' = g$.
\end{proof}

\subsection{Sucesiones uniformemente de Cauchy}

\begin{defi}[Sucesión uniformemente de Cauchy]
Sea $(E,d)$ un espacio métrico, $A\subseteq E$ y
$(E',d')$ otro espacio métrico. Una sucesión
$(f_n)_{n\in\N}$ de funciones $f_n : A \to E'$ se dice
\emph{uniformemente de Cauchy} si
\[
\forall \varepsilon>0\ \exists n_0\in\N\ \forall m,n\ge n_0\
\forall x\in A:\ d'\bigl(f_n(x),f_m(x)\bigr) < \varepsilon.
\]
\end{defi}

\begin{teo}
Sea $A\subseteq E$ y $(f_n)_{n\in\N}$ una sucesión de funciones
$f_n : A \to \R$ uniformemente de Cauchy. Entonces existe una función
$f : A \to \R$ tal que $f_n \rightrightarrows f$ en $A$.
\end{teo}

\begin{proof}
Fijemos $x\in A$. Consideremos la sucesión numérica
\[
\bigl(f_n(x)\bigr)_{n\in\N} \subset \R.
\]
De la definición de sucesión uniformemente de Cauchy se deduce en
particular que, para todo $\varepsilon>0$, existe $n_0$ tal que
\[
|f_n(x) - f_m(x)| < \varepsilon
\quad\text{para todo } m,n\ge n_0.
\]
Es decir, para cada $x\in A$, la sucesión $(f_n(x))$ es de Cauchy en $\R$.
Como $\R$ es completo, existe el límite
\[
f(x) := \lim_{n\to\infty} f_n(x) \in \R.
\]
Así definimos una función $f : A \to \R$.

Resta ver que $f_n \rightrightarrows f$ en $A$. Sea $\varepsilon>0$.
Por ser $(f_n)$ uniformemente de Cauchy, existe $n_0 \in \N$ tal que
\[
|f_n(x) - f_m(x)| < \varepsilon
\quad\forall m,n\ge n_0,\ \forall x\in A.
\]

Fijemos $n\ge n_0$ y $x\in A$ arbitrarios. Tomando el límite cuando
$m\to\infty$ en la desigualdad anterior, obtenemos
\[
|f_n(x) - f(x)| \le \varepsilon,
\]
ya que $f_m(x)\to f(x)$ para cada $x$.

Como la cota $\varepsilon$ es independiente de $x$ y vale para todo
$n\ge n_0$, concluimos que
\[
\sup_{x\in A} |f_n(x) - f(x)| \le \varepsilon
\quad\text{para todo } n\ge n_0.
\]
Esto es precisamente $f_n \rightrightarrows f$ en $A$.
\end{proof}
\section{Unidad 8: Medida de Lebesgue}
\subsection{Conjuntos nulos}

\begin{defi}[Conjunto nulo]
Sea $A \subseteq \R$. Decimos que $A$ es un \emph{conjunto nulo}
si para todo $\varepsilon > 0$ existen intervalos abiertos
contables $(U_n)_{n \in \N}$ tales que
\[
A \subseteq \bigcup_{n\in\N} U_n
\quad\text{y}\quad
\sum_{n\in\N} \operatorname{long}(U_n) < \varepsilon.
\]
\end{defi}

\subsection{$\sigma$-álgebras y conjuntos medibles de Lebesgue}

\begin{defi}[$\sigma$-álgebra]
Sea $X$ un conjunto y sea $\mathcal{A} \subseteq \mathcal{P}(X)$
una familia de subconjuntos de $X$. Decimos que $\mathcal{A}$ es
una \emph{$\sigma$-álgebra} si se verifica:
\begin{enumerate}[label=(\roman*)]
    \item $X \in \mathcal{A}$;
    \item si $A \in \mathcal{A}$, entonces $A^c = X \setminus A \in \mathcal{A}$;
    \item si $(A_n)_{n\in\N} \subseteq \mathcal{A}$, entonces
    \[
    \bigcup_{n\in\N} A_n \in \mathcal{A}
    \quad\text{y}\quad
    \bigcap_{n\in\N} A_n \in \mathcal{A}.
    \]
\end{enumerate}
\end{defi}

\begin{defi}[Conjuntos medibles de Lebesgue]
Sea $\mathcal{M}$ la $\sigma$-álgebra generada por los intervalos abiertos
y los conjuntos nulos de $\R$. A $\mathcal{M}$ la llamamos
\emph{$\sigma$-álgebra de conjuntos medibles de Lebesgue} en $\R$.

Si $I$ es un intervalo de $\R$, denotamos por $\mathcal{M}(I)$ a la
$\sigma$-álgebra de subconjuntos medibles de Lebesgue de $I$.
\end{defi}

\subsection{Medida de Lebesgue}

\begin{teo}[Existencia y unicidad de la medida de Lebesgue]
Existe una única función
\[
\mu : \mathcal{M} \longrightarrow [0,+\infty]
\]
tal que:
\begin{enumerate}[label=(\roman*)]
    \item si $A = (a,b)$ es un intervalo abierto acotado, entonces
    \[
    \mu(A) = b-a;
    \]
    \item si $(A_n)_{n\in\N} \subseteq \mathcal{M}$, entonces
    \[
    \mu\Bigl(\bigcup_{n\in\N} A_n\Bigr)
    \le \sum_{n\in\N} \mu(A_n);
    \]
    \item si, además, los $A_n$ son dos a dos disjuntos, entonces
    \[
    \mu\Bigl(\bigcup_{n\in\N} A_n\Bigr)
    = \sum_{n\in\N} \mu(A_n);
    \]
    \item para todo $A \in \mathcal{M}$ se cumple la propiedad de
    \emph{regularidad exterior}:
    \[
    \mu(A) = \inf\{\mu(U) : A \subseteq U,\ U \text{ abierto}\}.
    \]
\end{enumerate}
La función $\mu$ se llama \emph{medida de Lebesgue}.
\end{teo}

\subsection{Propiedades básicas de la medida de Lebesgue}

En esta subsección trabajamos, salvo aclaración en contrario,
en el intervalo $I = [0,1]$ con la medida de Lebesgue, y escribimos
$\mathcal{M}(I)$ para la $\sigma$-álgebra de subconjuntos medibles de $I$.

\begin{teo}[Propiedades básicas]
Sea $\mu : \mathcal{M}(I) \to [0,+\infty]$ la medida de Lebesgue. Entonces:
\begin{enumerate}[label=(\roman*)]
    \item \emph{Monotonía:} si $A,B \in \mathcal{M}(I)$ y $A \subseteq B$,
    entonces
    \[
    \mu(A) \le \mu(B).
    \]
    \item \emph{Conjuntos nulos:} si $A \subseteq \R$ es un conjunto nulo,
    entonces $A \in \mathcal{M}$ y $\mu(A) = 0$. Recíprocamente,
    si $A \in \mathcal{M}$ y $\mu(A) = 0$, entonces $A$ es un conjunto nulo.
    \item \emph{Invariancia por traslaciones:} dados $A \in \mathcal{M}$
    y $c \in \R$, se tiene $A+c := \{x+c : x\in A\} \in \mathcal{M}$ y
    \[
    \mu(A+c) = \mu(A).
    \]
\end{enumerate}
\end{teo}

\begin{proof}[Idea de la demostración]
La monotonía se obtiene escribiendo $B$ como unión disjunta de $A$ y
$B \setminus A$ y usando la $\sigma$-aditividad. Las afirmaciones sobre
conjuntos nulos se deducen de la relación entre definición de conjunto
nulo y la regularidad exterior. La invariancia por traslaciones se
prueba primero en intervalos (donde es obvia) y luego se extiende a
$\mathcal{M}$ usando que ésta es la $\sigma$-álgebra generada por
intervalos y conjuntos nulos, y que la traslación preserva nulos.
\end{proof}

\begin{prop}
Sea $I = [0,1]$ y sea $\mu$ la medida de Lebesgue en $\mathcal{M}(I)$.
Si $A,B \in \mathcal{M}(I)$, entonces:
\begin{enumerate}[label=(\roman*)]
    \item $A \setminus B \in \mathcal{M}(I)$;
    \item
    \[
        \mu(A \cup B) = \mu(A \setminus B) + \mu(B).
    \]
\end{enumerate}
En particular,
\[
\mu(A^c) = 1 - \mu(A), \qquad A^c = I \setminus A.
\]
\end{prop}

\begin{proof}
(i) Como $\mathcal{M}(I)$ es una $\sigma$-álgebra,
es cerrada por complementos e intersecciones. Observamos que
\[
A \setminus B = A \cap B^c,
\]
por lo que $A \setminus B \in \mathcal{M}(I)$.

\medskip

(ii) Notamos que
\[
A \cup B = (A \setminus B) \cup B
\]
y que $(A \setminus B)$ y $B$ son disjuntos. Por $\sigma$-aditividad,
\[
\mu(A \cup B) = \mu(A \setminus B) + \mu(B).
\]

La igualdad $\mu(A^c) = 1 - \mu(A)$ se obtiene aplicando esta fórmula
con $A$ y $A^c$ observando que $I = A \cup A^c$ y $\mu(I)=1$.
\end{proof}

\subsection{Regularidad de la medida de Lebesgue}

\begin{prop}[Regularidad exterior]
Sea $A \in \mathcal{M}(I)$. Entonces
\[
\mu(A) = \inf\{\mu(U) : A \subseteq U,\ U \text{ abierto en } I\}.
\]
\end{prop}

\begin{proof}[Esbozo]
Esta propiedad forma parte de la construcción misma de la medida de Lebesgue
(en el teorema de existencia). La desigualdad
\[
\mu(A) \le \inf\{\mu(U)\}
\]
se sigue de la monotonía: si $A \subseteq U$, entonces $\mu(A) \le \mu(U)$.
En la construcción de la medida se verifica además que para todo
$\varepsilon>0$ existe un abierto $U \supseteq A$ tal que
$\mu(U) < \mu(A) + \varepsilon$, lo que da la igualdad.
\end{proof}

\begin{prop}[Regularidad interior]
Sea $A \in \mathcal{M}(I)$. Entonces
\[
\mu(A) =
\sup\{\mu(F) : F \subseteq A,\ F \text{ cerrado en } I\}.
\]
\end{prop}

\begin{proof}
Por monotonía, si $F \subseteq A$ entonces $\mu(F) \le \mu(A)$,
de donde
\[
\sup\{\mu(F) : F \subseteq A,\ F \text{ cerrado}\} \le \mu(A).
\]

Para la otra desigualdad, sea $\varepsilon>0$.  
Por regularidad exterior aplicada a $A^c$, existe un abierto
$U \supseteq A^c$ tal que
\[
\mu(U) < \mu(A^c) + \varepsilon.
\]
Tomando complementos en $I$, el conjunto
\[
F := U^c = I \setminus U
\]
es cerrado y satisface $F \subseteq A$.

Además, por la proposición anterior,
\[
\mu(F) = \mu(I) - \mu(U).
\]
Como $\mu(I)=1$ y $\mu(A^c) = 1 - \mu(A)$, obtenemos
\[
\mu(F) = 1 - \mu(U)
> 1 - (\mu(A^c) + \varepsilon)
= \mu(A) - \varepsilon.
\]
Por lo tanto, para todo $\varepsilon>0$ existe un cerrado $F \subseteq A$
tal que $\mu(F) > \mu(A) - \varepsilon$, lo que implica
\[
\mu(A) \le
\sup\{\mu(F) : F \subseteq A,\ F \text{ cerrado}\}.
\]
\end{proof}

\begin{prop}[Regularidad fuerte]
Sea $A \in \mathcal{M}(I)$ y $\varepsilon>0$. Entonces existen
un cerrado $C$ y un abierto $U$ tales que
\[
C \subseteq A \subseteq U
\quad\text{y}\quad
\mu(A) - \varepsilon < \mu(C) \le \mu(A) \le \mu(U) < \mu(A) + \varepsilon.
\]
Además, $U$ puede elegirse como una unión numerable de intervalos
abiertos dos a dos disjuntos.
\end{prop}

\begin{proof}
Sea $\varepsilon>0$.  
Por regularidad interior, existe un cerrado $C \subseteq A$ tal que
\[
\mu(C) > \mu(A) - \frac{\varepsilon}{2}.
\]
Por regularidad exterior, existe un abierto $U \supseteq A$ tal que
\[
\mu(U) < \mu(A) + \frac{\varepsilon}{2}.
\]
De aquí se obtiene
\[
\mu(A) - \varepsilon
< \mu(A) - \frac{\varepsilon}{2}
< \mu(C) \le \mu(A) \le \mu(U)
< \mu(A) + \frac{\varepsilon}{2}
< \mu(A) + \varepsilon.
\]

El hecho de que cualquier abierto $U \subseteq I$ puede escribirse
como unión numerable de intervalos abiertos dos a dos disjuntos es
un resultado clásico de análisis real (se prueba usando que $U$ es
una unión numerable de componentes conexas, que en $\R$ son intervalos).
Aplicándolo a este $U$, se obtiene la última afirmación.
\end{proof}

\subsection{Continuidad de la medida}

\begin{teo}[Continuidad de la medida]
Sea $\{A_n\}_{n\in\N} \subseteq \mathcal{M}(I)$. Entonces:
\begin{enumerate}[label=(\roman*)]
    \item (Continuidad desde abajo) Si
    \[
    A_1 \subseteq A_2 \subseteq \cdots \subseteq A_n \subseteq \cdots
    \]
    y $A = \bigcup_{n\in\N} A_n$, entonces
    \[
    \mu(A) = \lim_{n\to\infty} \mu(A_n).
    \]
    \item (Continuidad desde arriba) Si
    \[
    B_1 \supseteq B_2 \supseteq \cdots \supseteq B_n \supseteq \cdots
    \]
    y $B = \bigcap_{n\in\N} B_n$, con $\mu(B_1)<\infty$, entonces
    \[
    \mu(B) = \lim_{n\to\infty} \mu(B_n).
    \]
\end{enumerate}
\end{teo}

\begin{proof}
(i) Definimos
\[
C_1 = A_1, \qquad
C_n = A_n \setminus A_{n-1} \quad (n \ge 2).
\]
Entonces los conjuntos $C_n$ son dos a dos disjuntos y
\[
A = \bigcup_{n\in\N} A_n = \bigcup_{n\in\N} C_n.
\]
Por $\sigma$-aditividad,
\[
\mu(A) = \sum_{n=1}^\infty \mu(C_n).
\]
Además, para cada $n$,
\[
\mu(A_n) = \mu\Bigl(\bigcup_{k=1}^n C_k\Bigr)
= \sum_{k=1}^n \mu(C_k).
\]
La sucesión de sumas parciales converge a la suma infinita, luego
\[
\lim_{n\to\infty} \mu(A_n)
= \sum_{k=1}^\infty \mu(C_k)
= \mu(A).
\]

\medskip

(ii) Definimos
\[
A_n = B_1 \setminus B_n \quad (n\in\N).
\]
Entonces $A_1 \subseteq A_2 \subseteq \cdots$ y
\[
\bigcup_{n\in\N} A_n = B_1 \setminus \bigcap_{n\in\N} B_n
= B_1 \setminus B.
\]
Aplicando (i) a la familia $(A_n)$,
\[
\mu(B_1 \setminus B)
= \lim_{n\to\infty} \mu(A_n)
= \lim_{n\to\infty} \bigl(\mu(B_1) - \mu(B_n)\bigr),
\]
donde usamos que $A_n = B_1 \setminus B_n$ y $\mu(B_1)<\infty$.
Entonces
\[
\mu(B_1) - \mu(B)
= \lim_{n\to\infty} \bigl(\mu(B_1) - \mu(B_n)\bigr).
\]
Restando $\mu(B_1)$ en ambos lados se obtiene
\[
\mu(B) = \lim_{n\to\infty} \mu(B_n).
\]
\end{proof}
\section{Unidad 9: Funciones medibles}
Sea $(X,\mathcal A)$ un espacio medible (es decir, $X$ es un conjunto y
$\mathcal A$ una $\sigma$-álgebra de subconjuntos de $X$).

\begin{defi}[Función medible real]
Una función $f : X \to \R$ se llama \emph{(Lebesgue) medible} si para
todo $a \in \R$ se cumple
\[
\{x \in X : f(x) < a\} \in \mathcal A.
\]
\end{defi}

\begin{teo}[Caracterizaciones de función medible]
Sea $(X,\mathcal A)$ un espacio medible y $f : X \to \R$ una función.
Son equivalentes las siguientes afirmaciones:
\begin{enumerate}[label=(\roman*)]
    \item Para todo $a \in \R$ se tiene
    \[
    \{x \in X : f(x) < a\} \in \mathcal A.
    \]
    \item Para todo $a \in \R$ se tiene
    \[
    \{x \in X : f(x) \le a\} \in \mathcal A.
    \]
    \item Para todo $a \in \R$ se tiene
    \[
    \{x \in X : f(x) > a\} \in \mathcal A.
    \]
    \item Para todo $a \in \R$ se tiene
    \[
    \{x \in X : f(x) \ge a\} \in \mathcal A.
    \]
\end{enumerate}
En particular, cualquiera de estas condiciones puede tomarse como
definición de función medible.
\end{teo}

\begin{proof}
$(i) \Rightarrow (ii)$.
Sea $a \in \R$. Mostramos que
\[
\{f \le a\} = \bigcap_{n=1}^{\infty} \{f < a + \tfrac{1}{n}\}.
\]

Primero, si $x \in \{f \le a\}$, entonces $f(x) \le a < a + \tfrac{1}{n}$
para todo $n \in \N$, y por lo tanto $x \in \{f < a + 1/n\}$ para todo $n$.
Esto prueba la inclusión
\[
\{f \le a\} \subseteq \bigcap_{n=1}^{\infty} \{f < a + \tfrac{1}{n}\}.
\]

Recíprocamente, sea $x$ tal que $x \in \{f < a + 1/n\}$ para todo $n$.
Entonces
\[
f(x) < a + \frac{1}{n} \quad \text{para todo } n \in \N.
\]
Supongamos, por absurdo, que $f(x) > a$. Entonces $f(x) - a > 0$ y
podemos definir
\[
\varepsilon = \frac{f(x) - a}{2} > 0.
\]
Tomamos $n$ suficientemente grande tal que $\tfrac{1}{n} < \varepsilon$.
Entonces
\[
a + \frac{1}{n} < a + \varepsilon = \frac{a + f(x)}{2} < f(x),
\]
lo cual contradice $f(x) < a + 1/n$. Por lo tanto no puede ser
$f(x) > a$, y forzosamente $f(x) \le a$, es decir $x \in \{f \le a\}$.

Concluimos la igualdad de conjuntos. Cada conjunto
$\{f < a + 1/n\}$ es medible por hipótesis (i), y las intersecciones
numerables de conjuntos medibles pertenecen a $\mathcal A$. Por lo tanto
$\{f \le a\}$ es medible. Así, (ii) se cumple.

\medskip

$(ii) \Rightarrow (iii)$.
Sea $a \in \R$. Observamos que
\[
\{f > a\} = X \setminus \{f \le a\}.
\]
En efecto, si $f(x) > a$ entonces $f(x)$ no puede satisfacer
$f(x) \le a$, y viceversa. Como $\mathcal A$ es una $\sigma$-álgebra,
es estable por complementos; de la medibilidad de $\{f \le a\}$ se
deduce la medibilidad de $\{f > a\}$. Luego (iii) se verifica.

\medskip

$(iii) \Rightarrow (iv)$.
Sea $a \in \R$. Mostramos que
\[
\{f \ge a\}
= \bigcap_{n=1}^{\infty} \{f > a - \tfrac{1}{n}\}.
\]

Si $x \in \{f \ge a\}$, entonces $f(x) \ge a > a - \tfrac{1}{n}$ para
todo $n$, y en particular $x \in \{f > a - 1/n\}$ para todo $n$.
Esto prueba la inclusión
\[
\{f \ge a\} \subseteq \bigcap_{n=1}^{\infty} \{f > a - \tfrac{1}{n}\}.
\]

Para la inclusión recíproca, sea $x$ tal que
$x \in \{f > a - 1/n\}$ para todo $n$, es decir,
\[
f(x) > a - \frac{1}{n} \quad \forall n \in \N.
\]
Supongamos, por absurdo, que $f(x) < a$. Entonces $a - f(x) > 0$ y
podemos definir
\[
\varepsilon = \frac{a - f(x)}{2} > 0.
\]
Elegimos $n$ suficientemente grande tal que $\tfrac{1}{n} < \varepsilon$.
Entonces
\[
a - \frac{1}{n} > a - \varepsilon = \frac{a + f(x)}{2} > f(x),
\]
lo cual contradice $f(x) > a - 1/n$. Por lo tanto no puede ser
$f(x) < a$, y debe cumplirse $f(x) \ge a$, es decir $x \in \{f \ge a\}$.

Hemos probado la igualdad. Cada conjunto $\{f > a - 1/n\}$ es medible
por hipótesis (iii), y la intersección numerable de medibles también
lo es; por lo tanto $\{f \ge a\}$ es medible. Se verifica (iv).

\medskip

$(iv) \Rightarrow (i)$.
Sea $a \in \R$. Notamos que
\[
\{f < a\} = X \setminus \{f \ge a\}.
\]
En efecto, si $f(x) < a$ entonces no puede ser $f(x)\ge a$, y si
$f(x)\ge a$ entonces no puede ser $f(x)<a$. Si $\{f \ge a\}$ es medible
por (iv), su complemento también pertenece a $\mathcal A$, de modo que
$\{f < a\}$ es medible.

Con esto cerramos el ciclo
\[
(i) \Rightarrow (ii) \Rightarrow (iii) \Rightarrow (iv) \Rightarrow (i),
\]
y las cuatro condiciones son equivalentes.
\end{proof}

\end{document}
